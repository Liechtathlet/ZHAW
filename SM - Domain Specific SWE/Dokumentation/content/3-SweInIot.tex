\chapter{Software Engineering in der Domäne "`Internet of Things"'}


\section{Programmiersprachen}

\subsection{Assembler}
Eine Implementation in Assembler kann unter bestimmten Voraussetzungen die beste Lösung für ein bestimmtes Problem sein. In der Regel wird der Assembler-Ansatz gewählt, wenn das Programm so effizient und sparsam wie möglich ablaufen und mit so wenig Ressourcen als möglich auskommen soll.



\subsection{C / C++}


\subsection{Java / .NET}
Für die Implementation kann es durchaus sinnvoll sein eine Hochsprache, wie Java oder .NET C\# einzusetzen. 

\section{Protokolle und Standards}

\subsection{Thread}
Protokolle: Thread, Netzwerk-Protokoll, Fokus: Security, Low Energy, notwendiger Chip, schon in vielen Geräten vorhanden, gestützt auf 6LoWPAN, IPv6 over Low power Wireless Personal Area Network, 


\subsection{AllJoyn}
Qualcomm entwickelt, anschliessend: Linux Foundation, AllSeen Alliance (Cisco, Microsoft, LG, HTC, ...)
Verbindung, Wartung Geräte in WLAN-Netzwerk, Kontrolle, Benachrichtungs Service, 

MQTT: Collect device data
CoAP
XBEE
XMPP: Access device data
DDS: Distribute Device Data
HTTP
AMQP (IOT-Client-Seitig?)

https://www.sparkfun.com/news/1705
http://micrium.com/iot/internet-protocols/
http://electronicdesign.com/embedded/understanding-protocols-behind-internet-things

\section{Frameworks}
In diesem Kapitel werden einige Frameworks vorgestellt, welche ....


\subsection{Kommunikation zwischen \gls{acr:IOT}-Geräten}
AllSeen
IOTIVITY
Z-Wave
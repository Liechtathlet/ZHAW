% !TeX encoding=utf8
% !TeX spellcheck = de_CH_frami

\chapter{Einleitung}

\section{Hintergrund}
Im Rahmen meines Bachelor-Studiums in Informatik an der \gls{acr:ZHAW} muss im 6. Semester eine Seminararbeit zu einem vorgegebenen Themenbereich erarbeitet werden. Ich habe mich für den Themenbereich "`Domain Specific Software Engineering"' entschieden.

Aus einem Themenkatalog konnte ein spezifisches Thema im Bereich "`Domain Specific Software Engineering"' ausgewählt werden. Ich habe mich für das Thema "`Internet of Things"' entschieden.

Für die Arbeit sollen circa 50 Arbeitsstunden aufgewendet werden. Dies entspricht etwa einem Umfang von 15 bis 20 Seiten. Zusätzlich gelten die Rahmenbedingungen gemäss dem Reglement zur Verfassung einer Seminararbeit (\cite{ZHAW:2012:Seminararbeit:Reglemente})

\section{Aufgabenstellung}
Es soll ein Dokument zum Thema Domain Specific Software Engineering im Bereich Internet of Things erstellt werden. Das Papier soll die Schwierigkeiten der Software-Entwicklung in diesem Bereich aufzeigen und einen groben Überblick über das Thema eben.

\section{Abgrenzung}
\todo{Abgrenzung}

\section{Motivation}
\todo{motivation}

\section{Struktur}
Diese Arbeit gliedert sich in folgende Hauptteile:
\begin{itemize}
\item 
\end{itemize}
\todo{ Struktur}
Im ersten Kapitel werden die Details zur Ausgangslage und die Hintergründe der Arbeit aufgezeigt. Im zweiten Kapitel wird mit Hilfe einer Umfrage innerhalb des Turnvereins eine Analyse erstellt. Aus dieser Analyse gehen die Randbedingungen, Ziele und Anforderungen an das Mini  System hervor. Diese Randbedingungen, Ziele und Anforderungen werden im Kapitel 'Evaluation' als Kriterien für die Vorselektion, Selektion und anschliessenden die Evaluation der Produkte verwendet. Im letzten Kapitel wird ein Fazit gezogen, eine Empfehlung an den  abgegeben und über die gesamte Arbeit reflektiert.
\chapter{Angriffe, Incident Detection \& Incident Response} \label{chap:AIDIR}
In diesem Kapitel wird erläutert, wie typische Angriffe ablaufen, wie diese erkannt und anschliessend entsprechend reagiert werden kann.


\section{Angriffe}
Für die Reaktion auf Angriffe und die Untersuchung von Angriffen ist es wichtig die grundlegenden Angriffsverfahren, -methoden und -techniken zu verstehen. In diesem Kapitel werden die wichtigsten Punkte erläutert.


\subsection{Angriffstypen}
Grundsätzlich können zwei Angriffstypen unterschieden werden. Auf der einen Seite stehen Massenangriffe, so genannte "`un-targeted attacks"', deren Ziel es ist so viele Geräte oder Services als möglich zu treffen. Das einzelne Opfer spielt dabei eine untergeordnete Rolle. Phising und Malware sind zwei Beispiele für solche Massenangriffe. Ausgenutzt wird hier grundsätzlich immer die Offenheit des Internets.

Auf der anderen Seite stehen gezielte Angriffe, so genannte "`targeted attacks"'. Diese Attacken sind in der Regel auf das Ziel oder das spezifische Szenario, massgeschneidert. Solche Angriffe werden über mehrere Monate hinweg geplant und vorbereitet. Oft sind diese Codes spezifisch entwickelt worden und können somit von Intrusion-Detection-Systemen und Anti-Viren-Software nicht oder nur sehr schwer erkannt werden. Ein Beispielt für eine solche Attacke wäre Spear-Phising.

Bei den gezielten Angriffen hat sich in den letzten Jahren eine neue Unterkategorie, die Kategorie der "`advanced persistent threats"'. Ziel dieser Angriffe ist es, möglichst lange unerkannt zu bleiben und den Einbruch zu vertuschen. Dabei werden gerade so viele Daten gesammelt, bzw. Aktionen durchgeführt, dass der Täter noch unerkannt bleibt. Ein solcher Angriff wird über mehrere Monate, wenn nicht sogar Jahre, hinweg vorbereitet und anschliessend Schritt für Schritt umgesetzt. Auch der eingesetzte Schadcode wird so gebaut, dass dieser möglichst lange unterkannt bliebt, aber trotzdem so viel Nutzen als möglich erbringen kann.


\subsection{Kategorien von Schwachstellen}
Bei einem Angriff werden immer vorhandene Schwachstellen ausgenutzt. Diese Schwachstellen können in drei Kategorien unterteilt werden.

\begin{itemize}
\item \textbf{Flaws (Fehler / Mängel)} \\
Bei einem Flaw handelt es sich um eine unbeabsichtigte Funktionalität der Anwendung. Dieser kann entweder durch schlechtes Design oder simpel und einfach durch einen Implementierungsfehler entstehen.
\item \textbf{Features (Funktionalitäten)} \\
Hier wird eine vorhandene Funktionalität für andere Zwecke missbraucht. Dabei handelt es sich um keinen Fehler in der Anwendungen, sondern um eine Funktionalität, welche entsprechend spezifiziert wurde.
\item \textbf{User Errors (Benutzer Fehler)} \\
User Errors werden durch den Benutzer verursacht. Zum Beispiel könnte ein unerfahrener Systemadministrator unwissentlich Schwachstellen im System freischalten.
\end{itemize}


\subsection{Komplexität}
Durch die vorherrschende Monokultur von Betriebsystemen, Anwendungen und Komponenten werden die Anforderungen an Hacker immer grösser. Mit den steigenden Anforderungen werden auch die Angriffe und die Angriffstechniken immer komplexer.

\todo{Bild CF Seite 13}


\subsection{Täter}
Die Motivation von Tätern sind sehr unterschiedlich. Dies reicht von Sozielen, politischen, finanziellen, staatlich-politischen Motivationen über technische Ambitionen bis hin zu Regierungen oder Gruppierungen wie Anonymous. Neben der Motivation können die Täter nach Aussen- und Innentätern unterschieden werden. Innentäter verfügen über Insider-Wissen und arbeiten in der Regel für das angegriffene Unternehmen oder die angegriffene Organisation. Der Anteil an Innentätern am gesamten Tätervolumen ist sehr hoch und wächst stetig. Unternehmen und Organisationen sind sich dessen aber nicht immer bewusst und wähnen sich in falscher Sicherheit.

Die "`Berufsbezeichnungen"' der Täter sind sehr unterschiedlich und vielfältig. Nachfolgend sind einige der gängigsten Bezeichnungen und deren Bedeutung aufgelistet.

\todo{Vervollständigen}
\begin{itemize}
\item \textbf{Elite} \\
\item \textbf{Hacker} \\
Neutraler Begriff
\item \textbf{Cracker} \\
Negativ
\item \textbf{Script Kiddy} \\
\item \textbf{White-Hat} \\
Berücksichtigung Hackerethik, Penetrationstests
\item \textbf{Gray-Hats} \\
\item \textbf{Black-Hats} \\
...
\end{itemize}


\subsection{Typischer Ablauf}
Ein Angriff kann in die nachfolgenden Phasen gegliedert werden. Diese können je nach Angriff in unterschiedlichen Ausprägungen vorkommen.

\subsubsection{Survey (Untersuchung)}
In dieser Phase werden so viele Informationen wie möglich gesammelt. Dazu gehören Informationen über die Organisation, die eingesetzte Hard- und Software und Prozesse. Anschliessend wird versucht so viele Schwachstellen wie möglich zu ermitteln. Zum einen wird ein Footprinting durchgeführt, welches so viele Informationen wie möglich über die Systeme zu Tage befördern soll. Zum Footprinting gehören unter anderem Port- und Protokollscanns und DNS- und WHOIS-Abfragen. Zum anderen werden mit Hilfe von Social Engeineering und Commodity-Toolkits und -Techniken weitere Schwachstellen ermittelt.


\subsubsection{Delivery (Positionierung)}
Diese Phase beschäftigt sich mit den expliziten Vorbereitungen für die Ausnutzung der Schwachstellen. Der Angreifer versucht das für dieses Szenario am besten geeignete Vorgehen zu ermitteln und bringt sich anschliessend in Position um die Schwachstellen auszunutzen. Eine typische Aktion in dieser Phase wäre zum Beispiel der Versand einer infizierten E-Mail oder das Unterjubeln eines inifizierten USB-Sticks.


\subsubsection{Breach (Ausnutzung)}
In dieser Phase wird die Schwachstelle ausgenutzt, um dem Angreifer Zugang zum gewünschten System zu verschaffen.

\subsubsection{Affect (Beeinträchtigung / Infizierung)}
Nach dem der Angreifer Zugang zum System erlangt hat, unternimmt er weitere Schritte um sein eigentliches Ziel zu erreichen. Dies kann zum Beispiel die Erweiterung seiner Zugriffsrechte, die Einrichtung von Hintertüren, die Sammlung von Daten oder der Angriff eines weiteren Systemes sein.


\subsubsection{Clean Up (Aufräumen)}
Je nach Ziel und Zweck des Angreifers verwischt er seine Spuren und räumt auf, damit er unerkannt bleibt oder allenfalls zu einem späteren Zeitpunkt nochmals zurückkehren kann.

\section{Incident Detection (Erkennung eines Vorfalls)} \label{sec:Angriffe:IncidentDetection}
Bevor auf einen Angriff, beziehungsweise auf einen Sicherheitsvorfall, reagiert werden kann muss dieser zuerst bemerkt werden. Bleibt der Vorfall unterkannt, wird es nie zu einer Untersuchung kommen.
Ein Angriff kann durch verschiedenste Indikatoren erkannt und zum Teil sogar vorausgesagt werden. Nachfolgend werden einige dieser Indikatoren aufgelistet.

\subsection{Hinweise Netzwerkseitig}
\begin{itemize}
  \item Ungewöhnlich hohe Netzwerklast
  \item Ungewähnliche Anzahl Firewall-Regelverstösse
\end{itemize}

\subsection{Hinweise Serverseitig}
\begin{itemize}
  \item Unbekannte Prozesse
  \item Unbekannte / Neue User
  \item Unbekannte Dateien
  \item Ungewöhnliche Systemlast
  \item Dienste laufen nicht mehr
  \item Ungewöhnliche Systemanmeldungen
  \item Systemabsturz
  \item Kleiner werdende Log-Files
  \item Bestehende Dateien werden grösser (Beispiel: Ausführbare Datei wächst um mehrere kB)
  \item Versuch Berechtigungen zu verändern
  \item Schlechte Performance
\end{itemize}

\subsection{Hinweise durch Intrusion-Detection-Systeme}
Intrusion-Detection-Systeme sind dazu da Angriffe möglichst früh zu erkennen und die entsprechenden Stellen zu informieren. Ist das Intrusion-Detection-System gut konfiguriert, kann dieses Angriffe anhand von Strategien und Mustern erkennen.

\todo{Data Loss Prevention Technology}


\subsection{Weitere Hinweise}
Weitere Hinweise können durch Kunden, Partner, Mitarbeiter, Strafverfolgungsbehörden oder die Presse erfolgen.
\todo{Intrustion-Mapping-Systeme}

\subsection{Meldung eines Vorfalles}\label{subsec:Angriff:MeldungVorfall}
Wurde ein möglicher Sicherheitsvorfall oder ein Angriff gemeldet, ist es wichtig, dass die Person, welche die Meldung entgegen nimmt korrekt und schnell reagiert. Personen welche solche Meldungen entgegen nehmen könntetn (z.B. Mitarbeiter des Service Desks) sollten geschult und mit einem entsprechenden Merkblatt und einer Checkliste / Formular ausgestattet werden. Die entgegenehmende Person muss vom Melder so viele Informationen wie möglich erfragen, damit anschliessend schnellere und effizientere Entscheidungen getroffen werden können. Dabei sind sowohl Informationen zum Melder, als auch über die Symptome und den Zustand des Systemes von Interesse.

\todo{Verweis Formular}

Nachdem ein Vorfall gemeldet wurde, ist unverzüglich das zuständige Incident Response Team zu informieren und aufzubieten. Gibt es in der Organisation kein Incident Response Team und keinen Incident Response Plan ist das weitere Vorgehen mit dem Vorgesetzten und allenfalls einem Mitglied des höheren Managements abzustimmen. Übereilte Reaktionen sollten vermieden werden, da dadurch Beweisspuren verwischt oder vernichtet werden können. 

\section{Incident Response Team}
Das Incident Response Team ist die Eingreiftruppe beim Eintretten eines Sicherheitsvorfalles. Die Aufgabe dieses Team ist es im Falle eines Incidents auf Basis der vorhandenen Informationen eine Lagebeurteilung und Risikoeinschätzung durchzuführen und anschliessend entsprechende Massnahmen einzuleiten.

In einem Incident Response Team sollten folgende Rollen besetzt werden.

\begin{itemize}
\item \textbf{Kern-Team} \\
\begin{itemize}
\item Koordinator / Leiter mit direktem Zugang zum Management
\item Kontaktstelle zur Entgegennahme von Verdachtsmeldungen
\item Incident-Spezialist oder einen Ermittler aus dem Bereich der Computer Forensik
\end{itemize}
\item \textbf{Erweitertes Team} \\
\begin{itemize}
\item Juristischer Berater
\item Auditor
\item Mitarbeiter der physikalischen Sicherheit
\item HR-Mitarbeiter
\item Fachspezialisten (z.B. Netzwerk-, Sicherheits- oder Datenbankadministratoren)
\end{itemize}
\end{itemize}

Die Mitarbeiter dieses Teams sollten über längere Erfahrung in ihrem Tätigkeitsbereich verfügen, gute Kommunikationsfähigkeiten besitzen, teamfähig sein und gut integriert und zuverlässig sein. Darüber hinaus müssen sie in der Lage sein unter Stress effiziente und akzeptable Entscheide zu treffen, sich an vorgegebene Regeln und Prozeduren zu halten und in sicherheitsrelevanten Aspekten als Vorbild dienen. Sie müssen in der Lage sein sich unter Stress an vorgegebene Regeln und Prozeduren zu halten.

Bei grossen Organisationen kann das Incident Response Team als Dauerhaftes Team vorhanden ist, welches auch noch andere Aufgaben im Sicherheitsbereich wahrnimmt. Bei kleineren Organisationen kann es sich um ein Team mit Mitgliedern aus mehreren Organisationseinheiten handeln, welche im Notfall zusammengerufen werden können. Denkbar ist es auch, dass das ganze Incident Response Team oder einen Teil davon (z.B. den Incident-Spezialisten) durch eine externe spezialisierte Unternehmung wahrgenommen wird.


\section{Incident Response}
Die Incident Response hat zum Ziel bei einem Sicherheitsvorfall so rasch als möglich den entstandenen Schaden, die verwendeten Angriffsmethoden und die Auswirkungen für die Organisation zu beurteilen und anschliessend entsprechende Massnahmen umzusetzen. Die Computer Forensik ist ein essentieller Bestandteil des Incident Response Prozesses.




Notwendig: guter Beweissicherungsmassnahmen im Prozess etablieren

Guter Incident Response Prozess / erfolgreicher Ablauf Incident Response: Basis für juristische Verfolgung


Wichtig: Ermittlung Ursaxche, grundlage für zukünftige Handlungsempfehlungen
Infos: Business Impact, ...

\subsection{Organisatorische Vorbereitung}
organisatorische Vorarbeit notwendig, um korrekt reagieren zu können. wenn nicht: im Entscheidenden Moment keine Ressourcen
-Incident Awareness: Beteiligte MA, Bewusstsein
-Grobes Konzept Sicherheitsvorfallbehandlung (Eskalations- / Alarmierungsregelung, Weisungskompetenzen)
-Security-Monitoring- und Alarmierungskonzept (Einbezug: Personalvertreter, Datenschutzbeauftragter für Datenauswertung)
-Weiterbildungen: Incident-Detection / Response
-Kontakt zu Security-Spezialisten / Ermittlungsbehörden aufbauen
-....

Vorbereitung ("`Readiniess"'): Autorisierung, wichtig bei nicht polizeilichen Ermittlern, keine Aktionen auf eigene Faust, mehr Schaden als Nutzen, Incident-Response-Plan
Organisatorische Vorbereitung: Rollen, Verantowrtlichkeiten, Policies, Vorbereitende / Unterstützende Massnahmen im Rahmen System Life Cycle (Zentralisierte Logs, Auditing für Server, Arbeitsplatzcomputer, ..., file hashes für verbreitete Betriebssysteme und Installationen, File integrity checking software, data retention policies, etc. ), Guidelines, Step-By-Step- Procedures

\subsection{Incident Response Prozess}
Wurde ein Vorfall gemeldet gilt es zuerst zu beurteilen, ob es sich um einen wirklichen Sicherheitsvorfalle handelt, oder ob es sich um eine Betriebsstörung handelt.

Handelt es sich um einen Sicherheitsvorfall muss auf Basis der vorhandenen Informationen eine erste Einschätzung durchgeführt werden. Um für die Einschätzung alle relevanten Informationen zur Verfügung zu haben, ist es essentiell, dass bei der Entgegennahme der Meldung die entsprechenden Informationen erfragt werden (Siehe dazu Kapitel \ref{subsec:Angriff:MeldungVorfall}). Sind zu wenig Informationen vorhanden, kann bereits eine erste Analyse durchgeführt werden. 

Es ist jedoch darauf zu achten, das....keine Beweise zerstören, nicht bemerkt werden, jur. Verwendbarkeit..


-Honeypots


-Nach Abschluss:
--Ermittlunsvorgang analysieren und verbessern
--Reaktionszeit, Wirksamkeit, Tätermotivation, Kosten
--Schlüsse ziehen, permanente Massnahmen etablieren.

Strategie, zwei Aspekte berücksichtigen: direkten / indirekten Schaden minimieren, Tathergang möglichst umfassend rekonstruieren zur Identifikation Tatverdächtige, jeder Sicherheitsvorfall erfordert andere Strategie
-Kritkialität System im Bezug auf Unternehmensprozesse
-Kritikalität / Wichtigkeit gestohlene Daten
-Täger-Vermutung?
-Vermutung Fähigkeiten / Wissen beim Täter
-Vorfall an Öffentlichkeit gelangt?
-Wie weit ist der Täter gekommen
-Verkraftbare Downtime?
-Vermuteter finanzieller Gesamtverlust

-Kein unüberlegter Gegenangriff, Angriffer bemerkt, evtl. zerstörung,



Beurteilung Vorfall / Störung: nicht zu lange warten, Durchleuchtug / Ausschlussverfahren

Bei Einbruch: erste Risikoabschützung für möglcihe Abschaltzung / Netzdekonnektion, Berücksichtigung weiterer Ermittlungsschritte

Entscheid Abschaltung / Dekonnektion: Management der Systemeigentümer, Basis: Empfehlung ISR

Klassifizierung  des Vorfalls: Probing  / Portscanning, Denial-of--service-Angriff, Unberechtigter Zugriff auf User-Account, ... Admin-Account, Datendiebstahl / -manipulation,


\begin{itemize}
  \item Härtung der Systeme, Abwehr des Angriffes
  \item Abwarten, Beobachten, Informationen sammeln.
\end{itemize}


\subsection{Reaktionsarten}
Bei der Auswahl von geeigneten Massnahmen ist immer auch der Zeitpunkt der Angriffes zu beachten.

\begin{itemize}
  \item Angriff in der Zukunft
  \item Angriff ist am Laufen
  \item Angriff ist schon vorbei
\end{itemize}

\subsubsection{Abwarten, Beobachten, Informationen sammeln}

\begin{itemize}
  \item Eigentlicher Angriff noch ausstehend, Härtung und Abwehr
  \item Eigentlicher Angriff noch ausstehend, Abwarten, Beobachten, Informationen sammeln, Backtracing
  \item Angriff am Laufen, Abwehr (Härtung)
  \item Angriff am Laufen, Abwarten, Beobachten, Informationen
\end{itemize}


\section{Ablauf}
\begin{enumerate}
\item \textbf{Identify (Identifizierung)}\\
\begin{enumerate}
\item Eingang eines Hinweises für einen Verdachtsmoment (Siehe dazu Kapitel \ref{sec:Angriffe:IncidentDetection})
\item Handlet es sich um einen Sicherheitsvorfall oder eine Betriebsstörung?
\end{enumerate}
\item \textbf{Assess (Beurteilung)}\\
\begin{enumerate}
\item Durchführen einer ersten Analyse / Sicherstellung von Spuren
\item Einschätzung auf Basis der vorhandenen Informationen.
\item Bestätigung des Verdachtes.
\end{enumerate}
\item \textbf{Respond (Reagieren)}\\
\begin{enumerate}
\item ...Initial Response (Determine origin, identify compromised systems, disconnect, untersuchung, anzeige)
\item ...Procedure
\item ...Recovery
\end{enumerate}
\item \textbf{Report (Bericht)}\\
\begin{enumerate}
\item ...
\end{enumerate}
\item \textbf{Review (Rückblick)}\\
\begin{enumerate}
\item ...
\end{enumerate}
\end{enumerate}




\begin{itemize}
  \item Sicherstellung elektronische Beweise
  \item Beweisspuren identifizieren
  \item Beweisspuren analysieren
  \item Analyseergebnisse interpretieren / verifizieren
  \item Analyseergebnisse in Bericht zusammenfassen / präsentieren.
\end{itemize}

\chapter{Die Domäne "`Internet of Things"'}
In diesem Kapitel wird die Domäne "`Internet of Things"' näher beschrieben und die Eigen- und Besonderheiten im Bereich Software Engineering aufgezeigt.

\section{Ziel}
Das primäre Ziel des \gls{acr:IOT} ist die Vernetzung von Dingen mit der realen Welt und mit anderen Dingen. Die Dinge sollen intelligent miteinander kommunizieren können um so einen Mehrwert zu schaffen. 

\section{Anwendungsbereich}
Das \gls{acr:IOT} könnte grundsätzlich überall eingesetzt werden. Sei es in der Landwirtschaft, in der Pflege, in der Medizin oder in der Industrie. Das \gls{acr:IOT} ist mehr ein Konzept, beziehungsweise eine Architektur, als ein konkretes Produkt, Dadurch lässt es sich auf die jeweiligen Bedürfnisse der verschiedenen Unternehmensbereiche individuell anpassen. Neben der Anwendung im geschäftlichen Umfeld, findet das \gls{acr:IOT} auch im privaten Umfeld Anwendung. Das primäre Ziel ist dabei die Heimautomation, das heisst die intelligente Vernetzung verschiedener Geräte und Gegenstände im Haushalt. Nachfolgend einige Beispiele für konkrete Anwendungen des \gls{acr:IOT}

\begin{itemize}
\item Smart City
\item ÖL-Pipeline
\item Pflege
\item Medizin
\item Energie
\item Verteidigung
\item Kommunikation
\item Industrie
\end{itemize}

Auch im Bereich Big Data stellt \gls{acr:IOT} ein weiterer Meilenstein dar. Durch die Vernetzung von Millionen von Geräten können riesige Datenmengen gesammelt, analysiert und ausgewertet werden.



\section{Architektur \& Aufbau}


\subsection{Das "`Thing"'}
Die Definition eines "`Thing"', beziehungsweise eines "`Dinges"', ist nicht ganz einfach. Grundsätzlich handelt es sich um ein "`Embedded System"' welches Informationen über ein Netzwerk versendet und empfängt. Ein "`Embedded System"' basiert auf einem Mikrocontroller, verfügt relativ gesehen nur über einen geringe Ressourcen (Prozessor und Arbeitsspeicher) und ist in der Regel auf Energieeffizienz ausgelegt. Das "`Embedded System"' verfügt über einen Kommunikationsstack mit 1 bis n Kommunikationsprotokollen welcher entweder direkt oder über ein Gateway mit einem Netzwerk kommuniziert. Die gesammelten Daten werden entweder auf dem Gerät oder auf einem Gateway gefiltert. 

Solche "`Embedded Systems"' sind bereits heute überall gegenwärtig. Sei es in Autos, Wearables, Kühlschränken, Kaffeemschinen oder Smartphones.


\subsection{Gateway}
Filtern, managedn Daten, Verbindungshub

Gemäss Intel WhitePaper: 85\% der Geräte nicht für Internet-Kommunikation ausgelegt (Requirements Stack)
Intermediär Netzwerk und Legacy Things


Netzwerk und Cloud: Datanalyse der Rohdaten
Lokales Netzwerk
Back-End Services (Enterprise Data System, PC, mobile Device)



\section{Unterschied zum Konzept des klassischen Internet}
Im Gegensatz zum 
Paradigmenwechsel: Früher Server stellen Daten zur Verfügung.
Das Konzept verändert sich auch dahingehend, dass nicht mehr nur Server miteinander kommunizieren, sondern auch die Endgeräte.

\section{Effekte \& Auswirkungen}
Durch \gls{acr:IOT} kann Mehrwert auf verschiedenen Ebenen geschaffen werden. Zum Beispiel in Industriebetrieben können Produkte schneller und besser entwickelt und produziert werden, was sich am Ende auf die Kosten auswirken wird.

-Besseres Shopping vergnügen
-Optimierung Energienutzung / -konsum

\section{Subdomänen}
\subsection{Wireless Sensor Network}
\gls{acr:WSN} sind Netzwerke von verteilten Sensoren, welche gewisse physikalische Zustände oder Zustände in der Umgebung und Umwelt überwachen. Ein \gls{acr:WSN}-Node ist ein billig produzierbares Gerät, welches nur sehr wenig Strom benötigt. Idealerweise wird dieses über eine Batterie oder eine autonome Energiequelle (zum Beispiel ein Solar-Panel) betrieben. Ein \gls{acr:WSN}-Node besitzt nur eine einzelne Funktion.

Ein \gls{acr:WSN}-Edge-Nodes verbindet mehrere \gls{acr:WSN} mit einem Netzwerk, Gateway

\todo{image: http://micrium.com/iot/devices/}	
Vorwiegend industrie

WSN-Technologien: Wi-Fi (hohe Verbreitung, hoher Stromverbrauch), Low-Power-Solutions (Low-Power and efficient radios, energy harvesting, mesh networking, long term operation, new protocols and data formats), IEEE 802.15.4: Radio Standard, base low power system, power reduction, 6LoWPAN: As small messages as possible, IPv6 over Low power Wireless Personal Area Networks, Kapselung und Kompressionsmechnismen für kürzere Übermittlungszeiten

\subsection{Kommerzielle und Heimnutzung}
Bluetooth, Ethernet (wired or not wired), nur lokale services











































--
Anforderungen allgemein:
-Geschwindigkeit, Skalierung, Fähigkeiten
-Datensicherheit und Datenschutz, Management von Geräten, Data analytics
-Grosses Ökosystem: Regulatorische / Rechtliche Schwierigkeiten
-Unterschiedlich je Subdomäne: http://micrium.com/iot/iot-rtos/
\section{Anforderungen an Software}
Stark von Einsatzgebiet abhängig (Kühlschrank vs. Low Energy ohne Stromversorgung)


\section{Anforderungen an die Softwarearchitektur}

\section{Anforderungen an die Programmiersprache}

----

\section{Anforderungen / Unterschiede zur "`Standard-Domäne"'}
In diesem Kapitel werden für die einzelnen Abschnitte des Software Engineering Prozesses die Anforderungen, beziehungsweise die Unterschiede, zum Software Engeineering Prozess in der "`Standard-Domäne"' aufgezeigt. Mit "`Standard-Domäne"' wird in dieser Arbeit Software Engineering im Bereich von Enterprise-Anwendungen bezeichnet.

http://ercim-news.ercim.eu/en98/special/internet-of-things-a-challenge-for-software-engineering
http://link.springer.com/chapter/10.1007%2F978-3-642-31479-7_47#page-1

Das der Begriff "`Internet of Things"' eigentlich nur ein Oberbegriff ist....
deckt dieser Begriff auch ein sehr groses Spektrum an verschiedensten Anwendungsmöglichkeiten ab.

\subsection{Software Requirements}
eindeutig, Komplexität schwierig, nachhaltig (schwieriger Update)

non-consumer-market: regulatorisches Schwierigkeiten, Remote Tracking, Kabellose Implantate, --> Requirements-Phase: Verfolgbarkeit und Prüfbarkeit, Geschickte Verisonierung, Reviews --> Sicherstellung Compliance und Erfolg
Wetter, Physische Einflüsse, Durch das Gerät erzeugte Hitze, Lange Lebensdauer

SW Muss: Skalierbar für breite Palette an verschiedenen Gerätekategorien
Modular sein --> nur Auswahl (RAM-Footprint)
Verbunden (Daten rein/raus)
Verlässlich:Zertifizierung für kritische Applikationen
\subsection{Software Design}
Schlank, Energieeffizient, einfach wartbar?

Berücksichtigung: Protokolle, Standards, Last-Anforderungen

Machine To Machine Connectivity, Wireless, API-Evolution in MInd
\subsection{Software Construction}
Je nach Anwendungsbereich: Hardwarenahe, Optimierung
Für die Entwicklung von Anwendungen für \gls{acr:IOT}-Geräte können
https://www.rti.com/company/careers/software-engineer.html grundsätzlich


Firmware für Spezifische Hardware, Adressierung Netzwerk- und Verbindungsprobleme, Security


\subsection{Software Testing}
Wenn bei Consumer: Update evtl. schwierig, Fehlfunktion schwerwiegend,

Test: Nachbildung physische Umgebung, komplexe Szenarien, Netzwerk-Anforderungen, 

\subsection{Software Maintenance}
Schwierig,...
Immer connected --> regular updates, wenn nicht regelmässig upgedatet --> Verlust kritischer Funktionalität, Continuous Delivery

Nicht überall möglich, Low-Bandwith, Hardware-Near Devices


\subsection{Software Configuration Management}


\subsection{Software Engineering Management}


\subsection{Software Engineering Process}
Defect Tracking, small scale projects, fast product turnaroudn, 

\subsection{Software Engineering Tools and Methods}


\subsection{Software Quality}
Höhere Qualität notwendig




Layer: Transport-Layer: TCP zum Teil overkill für IOT-Device --> UDP
UDP: beser geeignet für relatime-data, TCPS Acknowledgment and retransmission unnützer overhad für solche Anwendungen (Stück sprache nicht rechtzeitig übermittelt -> Retransmission sinnlos), 


Klassische Ansätze nicht alle geeignet für IoT, 

Service-based Application: composition and orchestration of Services
% !TeX encoding=utf8
% !TeX spellcheck = de_CH_frami

\chapter{Schlusswort} \label{chap:Finish}
In diesem Kapitel wird ein Fazit zu dieser Arbeit und den einzelnen betrachteten Bereichen gezogen und über die gesamte Arbeit reflektiert.

%---------------------------------------------------------
\section{Fazit}
Aufgrund der Ergebnisse der Analyse kann gesagt werden, dass aktuell im Bereich der Modellierung und Implementierung von Business Prozessen ein Wandel stattfindet. Mit diesem Wandel rückt der Einsatz des \gls{acr:IOT} etwas mehr in den Vordergrund. Dies ist vor allem im geschäftlichen Umfeld wahrzunehmen. Daher gibt es im geschäftlichen Breich bereits erste \gls{acr:BPM}-\gls{acr:IOT} Lösungen, welche die beiden Bereiche miteinander verknüpfen. Im Bereich Home Automation ist dies eher nicht der Fall, die betrachteten Lösungen haben zwar zum grössten Teil Möglichkeiten zur Automatisierung von Abläufen, aber nicht mit \gls{acr:BPMN} oder \gls{acr:BPEL}.

Aktuell müssen im Bereiche Home Automation noch Kombinationen von mehreren Produkten und Lösungen eingesetzt werden, um Prozesse mit Hilfe einer formalen und standardisierten Notation zu definieren.

\subsection{BPM in der Domäne "`Internet of Things"'}
BPM ist im \gls{acr:IOT} definitiv ein Thema und es gibt auch bereits erste Hersteller, welche mit Ihren BPM Lösungen Möglichkeiten anbieten um Aspekte des \gls{acr:IOT} abzudecken, beziehungsweise zu integrieren. Auch gibt es zahlreiche \gls{acr:IOT} Plattformen, welche entsprechende Mechanismen zur Abbildung von Prozessen und Workflows Out-of-the-Box anbieten.

Gerade für Unternehmen, welche das \gls{acr:IOT} für das Kerngeschäft relevant ist oder relevant werden könnte, ist die Integration in (bestehende) \gls{acr:BPM} Lösungen ein wichtiger Aspekt. Grund dafür ist die dadurch erreichten Effizienzsteigerungen und Kostenreduktionen. Ebenfalls stellt die \gls{acr:BPM} Integration sicher, dass auch grosse Volumen korrekt und effizient verarbeitet werden können und dadurch ein Teil der in Kapitel \ref{sec:AnalyseIot:ChallangesAndProblems} \nameref{sec:AnalyseIot:ChallangesAndProblems} aufgezeigten Punkte abgedeckt, beziehungsweise entschärft werden können.


\subsection{BPM in der Domäne "`Home Automation"'}
Die durchgeführten Recherchen haben gezeigt, dass es im Bereich "`Home Automation"' heute eigentlich keine Lösung gibt, bei welcher Prozesse über \gls{acr:BPMN} oder \gls{acr:BPEL} modelliert werden können. Die meisten Softwarelösungen und auch Kombi-Produkte aus Hard- und Software basieren auf dem Konzept von Auslösern (Triggern) und Aktionen (Actions). Die Verwendung oder Integration von \gls{acr:BPMN} oder \gls{acr:BPEL} gestaltet sich aktuell noch als sehr schwierig.

Nichts desto trotz hätte die Verwendung von \gls{acr:BPM} auch im Bereich Home Automation einige nicht zu vernachlässigende Vorteile. Dem Gegenüber steht jedoch das dafür erforderliche technische Know-How bei den Endanwendern. Es ist daher davon auszugehen, dass auch in Zukunft die Verwendung formaler und standardisierten Sprachen eher nicht im Fokus von Home Automation Lösungen stehen werden. Viel mehr wird es darum gehen dem Benutzer eine einfache und effektive Möglichkeit zu bieten Abläufe zu definieren, konfigurieren und zu automatisieren.

\subsection{BPM auf dem "`Raspberry Pi"' in der Domäne "`Home Automation"'}\label{subsec:Fazit:BPMN:RPI:HA}
Bereits bei der Analyse zu "`BPM in der Domäne "`Home Automation"'"' hat sich gezeigt, dass es praktisch keine Lösung gibt, welche direkt \gls{acr:BPMN} und \gls{acr:BPEL} unterstützten. Aufgrund dessen musste der Ziel-Lösungsraum angepasst werden, sodass auch eine Kombination von Produkten verwendet werden kann um den Ziel-Setup zu realisieren. Um einen Setup für den zu realisierenden Beispielprozess zu finden, wurden mehrere Home Automation Lösungen auf dem Raspberry Pi installiert und getestet. Überzeugen konnten diese nur bedingt.

Die wichtigste Erkenntnis dabei ist, dass für alle Lösungen vertieftes technisches Know-How notwendig ist, um diese im Eigenheim zu betreiben. Dies steht im Gegensatz zu den ersten Grobeinschätzung aus dem Kapitel \ref{sec:Analyse:HA:LPF} \nameref{sec:Analyse:HA:LPF}. Von daher konnte aktuell keine der Lösungen vollumfänglich überzeugen. Die einzelnen Ansätze und umgesetzten Funktionalitäten sind jedoch vielversprechend und funktionieren soweit einwandfrei. 

Da es sich um Open Source Lösungen handelt, wird es noch eine Weile dauern, bis sich diese auch von Endanwendern mit wenig oder keinem technischen Fachwissen verwendet werden können.

\subsection{Beispielsetup}
Am Ende konnte ein guter Beispielsetup realisiert werden. Der Weg dahin war jedoch nicht einfach und hat noch viele Schwächen und Unschönheiten der betrachteten Produkte aufgezeigt.

Die Installation, Konfiguration und Benutzung von openHAB 2 ist grundsätzlich relativ gut, aber teilweise nur mit zusätzlichen Erweiterungen benutzbar. Da es sich bei der Version 2 um eine stark überarbeitete Version der Version 1 handelt, ist nicht immer ganz klar, was wie funktioniert. Dies ist vor allem der Dokumentation geschuldet, welche die Unterschiede und die Best-Practices der beiden Versionen nur sehr beschränkt aufzeigen. Ohne technisches Fachwissen und rudimentäre Programmierkenntnisse ist openHAB praktisch nicht zu verwenden, da die interessanten Anwendungsfälle nicht ohne diese nicht realisiert werden können. Die verwendeten und getesteten Erweiterungen (Bindings, Actions, Peristence, etc.) haben mehrheitlich gut funktioniert. Bei einigen war jedoch auch hier die Handhabung nicht ganz klar, beziehungsweise nicht intuitiv.

Für die activiti \gls{acr:BPM} Plattform gibt es diverse fixfertige Beispiele. Darunter ein Web-Explorer und \gls{acr:REST} Services. Auf Basis dieser beiden Beispiele wurde eine Kombination der beiden zusammengestellt und einige kleinere Konfigurationsanpassungen vorgenommen. Anschliessend war die Benutzung sehr intuitiv und selbsterklärend. Die Definition der Prozesse mit \gls{acr:BPMN} gestaltete sich dank des Eclipse-Plugins ebenfalls als sehr einfach und intuitiv. Es waren keine zusätzlichen Recherchen für die Benutzung von \gls{acr:BPMN} notwendig. Standardmässig sind jedoch sehr wenige spezifische Task verfügbar, sehr vieles muss über eigene Implementationen oder Scripts gelöst werden. Zum Beispiel gibt es keine Möglichkeit um direkt \gls{acr:REST} Services anzusteuern. Dies macht das ganze etwas unhandlich und reduziert die Portabilität und Plattformneutralität sehr stark.


%---------------------------------------------------------
\section{Reflexion}
Bereits seit längerer Zeit wollte ich aus Eigeninteresse einen Blick in die Welt von Smart Home und Home Automation werfen. Mit diesem Seminar hat sich mir nun eine gute Gelegenheit geboten. Dementsprechend hatte ich zu Beginn viele Ideen und Vorstellungen, was alles realisiert werden könnte.

Es hat sich dann schnell herausgestellt, dass das Ganze nicht so einfach werden wird. Bereits die ersten Recherchen haben gezeigt, dass im Open Source Heimanwenderbereich die Lösungen noch nicht so gut entwickelt sind, wie ich mir das vorgestellt hatte. Dadurch musste ich viel Zeit und Energie in den Setup und die Konfiguration der einzelnen Elemente des Beispielsetups investieren. Die grössten Schwierigkeiten bereiteten die unvollständigen oder mangelhaften Dokumentationen der Komponenten. 

Alles in allem waren die Recherchen und insbesondere die Realisierung des Demo-Setups sehr interessant und lehrreiche. Für den einen oder anderen Anwendungsfall werde ich wahrscheinlich in Zukunft auf das gesammelte Wissen und die gemachten Erfahrungen zurückgreifen können. Aktuell würde ich für eine Hom Automation Lösung eher auf eine Kombination aus einer Hard- und Software Komponente zurückgreifen. Grund dafür ist, dass diese im Moment ausgereifter sind als die verfügbaren Open Source Lösungen.




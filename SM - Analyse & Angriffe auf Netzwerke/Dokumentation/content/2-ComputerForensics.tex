\chapter{Computer Forensik}

Dieses Kapitel definiert den Begriff der Computer Forensik und beschreibt das Themengebiet im Allgemeinen. 

\section{Einbettung und Definition}
\subsection{Forensik}
\subsubsection{Ursprung}
Der Begriff "`Forensik"' stammt aus den Zeiten des antiken Roms. Damals wurden Gerichtsverfahren, Untersuchungen, Urteilsverkündungen und der Vollzug von Strafen öffentlich auf dem Marktplatz abgehalten. Marktplatz (oder auch Forum) wird im lateinischen mit \textit{forum} bezeichnet. Die Plural-Form von \textit{forum} ist \textit{foren}. Aus dieser Plural-Form hat sich der Begriff "`Forensik"' entwickelt.

\subsubsection{Bedeutung}
Die Forensik ist ein Wissenschaftszweig, welche sich mit dem Nachweis, Beweis und der Aufklärung von kriminellen, oder allgemein strafbaren, Handlungen beschäftigt. Die forensische Untersuchung ist eine systematische Analyse mit dem Ziel strafbare Handlungen zu identifizieren, analysieren und rekonstruieren.

Der "`Guide to Integrating Forensic Techniques into Incident Response"' des \gls{acr:NIST} beinhaltet eine kurze und prägnante Definition für den Begriff der "`Forensik"'.

\begin{center}
\textbf{"`Forensic science is generally defined as the application of science to the law"'} \cite[S. ES-1]{E:2006:NIST:Guide:IncidentResponse}
\end{center}

Übersetzt bedeutet dies so viel wie "`Forensische Wissenschaft ist allgemein definiert, als die Anwendung der Wissenschaft für das Gesetz"'.


\subsubsection{Teilbereiche}
Wie in der vorangehenden Definition bereits angedeutet, gibt es grundsätzlich für jeden Wissenschaftszweig einen entsprechenden Wissenschaftszweig in der Forensik. Nachfolgend sind einige für die Strafverfolgung bedeutensten Teilbereiche der Forensik aufgelistet.

\begin{itemize}
\item Forensische Pathologie
\item Forensische Kriminaltechnik
\item Forensische Phsychiatrie und Psychologie
\item Forensische Toxikologie
\item Ballistik
\item Computer-Forensik
\end{itemize}


\subsection{IT- / Digitale Forensik}
Die IT-, bzw. Digitale, Forensik beschäftigt sich mit der Auffindung, Untersuchung und Wiederherstellung von Material, bzw. Daten, auf elektronischen, bzw. digitalen, Geräten. Dabei kann es sich zum Beispiel sowohl um verlorene Daten, als auch um explizites oder nicht explizites Beweismaterial handeln.

\subsubsection{Teilbereiche}
Die Unterteilung der IT- / Digitalen Forensik in ihre Teilgebiete ist nicht offiziell definiert. Nachfolgend wird eine mögliche Unterteilung aufgezeigt. Diese Unterteilung ist nicht vollständig und nicht abschliessend.

\begin{itemize}
\item Computer Forensik
\item Forensische Datananalyse
\item Datenbank Forensik
\item Mobile Device Forensik
\item Netzwerk Forensik
\item Forensische Videoanalyse
\item Forensische Audioanalyse

\end{itemize}


\subsection{Computer Forensik}
Für die Definition der Computer Forensik existieren zum heutigen zwei verschiedene Ansätze. Ein Ansatz sieht die Computer Forensik als Teilgebiet der IT-, bzw. der Digitalen Forensik. Der andere Ansatz betrachtet den Begriff Computer Forensik als Synonym zu den Begriffen IT- und Digitale Forensik.

Diese Arbeit richtet sich nach dem ersten Ansatz, bei dem die Computer Forensik ein Teilgebiet der Digitalen Forensik ist.


\todo{Definition Computer Forensik}

\todo{Grafik Einbettung}


\section{Einführung}
Die Computer Forensik kann in verschiedenen Kontexten zum Einsatz kommen. Zum einen erfolgt während, bzw. nach einem Sicherheitsvorfall (Incident), z.B. Systeminbruch eine forensische Untersuchung (Mehr dazu im Kapitel \ref{chap:AIDI}). 

Im Kontext der Incident Response ist es das Ziel der Computerforensik die ausgenutzte Schwachstelle zu finden, den Schaden zu beziffern, den Angreifer zu Identifizieren und die Beweise für allfällige juristsiche Schritte zu sichern.

Im Kontext der Untersuchung von Straftaten ist es das Ziel, 

.....

\section{Themengebiete und Teilbereiche}
-HW, SW, AW

\section{Anwendungsbereich}
Die Computer Forensik findet unter anderem in folgenden Bereichen Anwendung:

\begin{itemize}
\item Strafuntersuchungen
\item Incident Response / Incident Handlung
\item Log Monitoring
\item Datenwiederherstellung
\item Datenbeschaffung
\end{itemize}

\section{Ziele}

\section{Ausbildung \& Zertifizierung}

\section{Hinweise für die juristische Verwertbarkeit}
Sollen die sichergestellten Daten und Informationen juristisch verwertbar sein, zum Beispiel als Beweise in einem Strafprozess müssen einige zusätzliche Punkte beachtet werden. Grundsätzlich ist es sinnvoll die folgenden Punkte bei jeder Untersuchung zu berücksichtigen.

\subsection{Methoden, Techniken und Programme}
Die angewendeten Methoden und eingesetzten Techniken und Programme sollten in der Fachwelt akzeptiert und beschrieben sein. Neue Tools und Verfahren haben in der Regel einen schweren Stand, bis diese allgemein akzeptiert wurden.

\subsection{Glaubwürdigkeit und Reproduzierbarkeit}
Um die Glaubwürdigkeit der Ergebnisse sicherzustellen müssen sämtliche Schritte und die resultierenden Ergebnisse von Laien nachvollzogen werden können. Zusätzlich müssen die Ergebnisse durch einen anderen Experten reproduziert werden können. Der Ermittler, bzw. die Person, welche die forensische Untersuchung durchgeführt hat, muss in der Lage sein den gesamten Ablauf im Detail zu erklären. Erklärungen im Stiel von "`Diese Information wurde vom eingesetzten Analyse-Programm automatisch gefunden"' sind nicht gern gesehen und können die Glaubwürkdigkeit der gesamten Untersuschung in Frage stellen.


\subsection{Integrität}
Während der gesamten Ermittlung (und auch darüber) hinaus muss die Integrität der untersuchten Daten und gefundenn Informationen, Daten und Beweise lückenlos sichergestellt werden. Die Integrität muss jederzeit vollständig belegt werden können.

\subsection{Präsentation und Dokumentation}
Die Ergebnisse müssen angemessen dokumentiert und präsentiert werden. Am geeignetsten ist es, wenn die Ergebnisse in Form von Ursache - Wirkung aufgezeigt werden. Die Beweisspuren, Ereignisse und Personen sollen möglichst logisch und nachvollziehbar in Relation zu einander gebracht werden.





Unvoreingenommenheit
Gewisse Beweise kurze Halbwertszeit (meist sehr spannend), erfordern besonnenes / koordiniertes erfassen, Bewusstsein, dass System in jedem Fall verändert wird

Sachbeweis: Festplate, Logs, Gutachten, Fingerabdruck - keine Beweiskraft, nicht zugeordnet, keine Aussagekraft alleine, erst im Kontext, Beweiskraft erst druch Person, die Beweis in Tat-ZSH bringt, Sachbeweis eng mit Personenbeweis verbunden
Beweis rasch Bedeutungslos, weniger Beweiskraft, wenn Person unrichtig darstellt, widerlegbare Behauptungen, Interpretationen, dachliche Darstellung, Integrität Person und Glaubwürdigkeit wichtig, sachliches, fundiertes Gutachten durch unglaubwürdige Darstellung als nichtig betrachtet

\section{Hinweise zum Datenschutz}
Datenschutz bei personenbezogenen Daten auch bei Auswertungen zum Zug
Schweizer Recht??

Vorgängige Klärung wenn Logs personenbezogene Daten beinhalten, Information Datenschutzbeauftrager, Security \& Compliance, IT-leiter, REvision, Vier-Augen-Prinzip wahren

DS bei Ermittlung nicht ausser Kraft, aber kein Täterschutz, Verhältnismässigkeit



\section{Sicherungsebenen}
-Hardware-Ebene
-Software-Ebene
--Betriebssystem-Ebene
--Anwendungssoftware-Ebene

\section{Unterscheidung Daten-Typen}
\begin{itemize}
\item Empfindliche Daten
\begin{itemize}
\item Flüchtige Daten, gehen beim geordneten Shutdown / Ausschalten veroren (Cache, Hauptspeicher, Status NWV, Prozesse,...)
\item Fragile Daten, zwar auf HD, Zustand kann sich beim Zugriff ändern
\item Temporär zugreifbare Daten, auf HD, nur zu Bestimmten Zeitpunkten zugreifbar
\end{itemize}
\end{itemize}
-Flüchtig
-Nicht-Flüchtig

Evtl. Matrix mit Zuordnung Techniken / Themenbereichen zu Typen und Sicherungsebenen

\section{Anti-Forensik und Anti-Detection}
Straftäter und Angreifer auf Computer Systeme werden sich immer mehr bewusst, dass sie Spuren auf dem System hinterlassen. Diese versuchen dann entweder keine oder so wenig Spuren wie möglich zu hinterlassen, Spuren und Beweise zu verändern oder gar zu löschen oder falsche Fährten zu legen. Dies kann entweder manuell oder mit Hilfe von Anti-Forensik und Anti-Detection Tools erfolgen.

Das primäre Ziel dabei ist, zu verhindern, dass das Eindringen oder die verdächtige Handlung entdeckt wird. Dies wird eigentlich eher dem Themenbereich der Anti-Detection, also dem "`Unbemerkt bleiben"', zugeordnet. Bei der Anti-Detection versuchen die Täter unerkannt und unbemerkt zu bleiben. Zusätzlich wird versucht die Ermittler zu behindern, abzulenken oder die Datensammlung zu stören oder zu unterbinden. Zum Teil wird auch versucht den Umstand ausgenutzt, dass für eine Ermittlung nur eine beschränktes Zeitkontingent und Budget vorhanden ist. Dies kann dazu führen, dass der Ermittler nur die Beweise findet, die er soll und sich dann aus zeitlichen und budgettechnischen Gründen damit zufrieden gibt und die Untersuchung abschliesst.

Kennt der Angreifer die eingesetzten Werkzeuge oder kann diese ermitteln, kann er Schwachstellen und Sicherheitslücken in diesen ausnutzen und gezielt angreifen. im schlimmsten Fall kann der Angreifer die Ermittlungen gezielt manipulieren, ohne dass der Ermittler dies bemerkt. Daher sollte die Analyse zum einen in einer geschützten Umgebung durchgeführt werden und die verwendete regelmässig upgedatet werden.

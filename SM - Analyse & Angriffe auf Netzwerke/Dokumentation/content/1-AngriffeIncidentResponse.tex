\chapter{Angriffe, Incident Detection \& Incident Response}
In diesem Kapitel wird erläutert, wie typische Angriffe ablaufen, wie diese erkannt und anschliessend entsprechend reagiert werden kann.


\section{Angriffe}
Kenntniss über Angriffsverfahren, -methoden, -techniken wichtig, da auf diese reagiert, untersucht werden muss

Es kann zwischen generellen und zwischen Massgeschneiderten Attacken und Tools unterschieden werden....

\subsection{Angriffskategorien}
-Un-Targeted attacks, so viele Geräte, Services, User als möglich, Opfer spielt keine Rolle, Ausnutzung der Offenheit des Internets, z.B. Phising, Water holing, ransomware, scanning
-Targeted attacks: organisation / service / user, spezifisches interesse, Basisarbeit: z.T. mehrere Monate, fataler, Maasgeschneidert, z.B. Spear-Phsining, deploying botnet, subverting supply chain
-Targeted Attacks: Heute auch oft spezifische Angriffscodes für einzelne Unternehmen, Codes unbekannt, schwierigere Erkennung
-Advanced Persistent Threats: Solange als mölgich unerkannt bleiben, so viel als möglich, so wenig wie möglich zum unerkannt bleiben, Angriff über mehrere lange Phasen, Kein Vorgängiges Wissen zu Schwachstellen beim Ziel, Monate / Jahre, Schadcode ist so gebaut, dass er nicht gefunden wird, bleibt unbemerkt bestehen

\subsection{Typen von Schwachstellen}
-Flaws: ungbeabsichtigte Funktionalität, Schlechtes Design oder Fehler, oft lange unentdeckt
-Features: Missbrauch vorgesehen Funktion,
-User Error: z.B. unerfahrner Administrator, welcher Schwachstellen "freischaltet", Default-Passowrt, ...


\subsection{Komplexität}
Nimmt laufend zu, Monokulut Komponenten, OS, Anwendungen, Angriffstechniken auch komplexer, Anforderungen an Hacker immer höher,
Bild CF Seite 13


\subsection{Täter}
Vielschichtige Motivationen, Soziale Motivation, Technische Ambitionen, Politische / Finanzielle / staatlich-politische Motivation, Regierung, Gruppen (Anonymous)
Vielfältige Berufsbezeichnungen (Elite, Hacker, Script Kiddies, Cracker, ...), Hacker: neutral, Cracker-: negativ, White (Hackerethik, z.B. für Penetrationstests), Gray, Black-Hats

-Innen- / Aussentäter, Innentäter: Vorteile, Anteil bei Verbrechen relativ hoch, Unternehmen wähnen sich in falscher Sicherheit,


\subsection{Typischer Ablauf}
Ein Angriff kann in die nachfolgenden Phasen gegliedert werden. Diese können je nach Angriff in unterschiedlichen Ausprägungen vorkommen.

Stages:
-Survey: Untersuchen / Analysieren von verfügbaren Informationen, Schwachstellen ermitteln, Social Engeineering, Commodity-Toolkits / -Techniques, Network Scanning, Infos über Org. und IT, user Errors ausnutzen
-Delivery: Soweit kommen, dass die Schwachstelle ausgenutzt werden kann. Angreifer bringt sich in Position (z.B. Zugriff auf Online-Service, Versand Mails mit infizierten Links / Attachment, ifidzierter USB-Stick, Fake Website,), selecting best delivery path to breach defence
-Breach: Ausnutzen der Schwachstelle
-Affect: Auskundschaften der Systeme, Zugriffe erweitern, Hintertüre einrichten, user / admin account,

\todo{CF: Seiten 34-36}
\begin{itemize}
  \item Footprinting
  \item Port- und Protokollscan
  \item Enumerationng / Penetration
  \item Hintertüren einrichten
  \item Spuren verwischen
\end{itemize}

Phasen APT
-Phase 1: Erkennung, Start und Infizieren: Survey, Delivery, + Infect
-Phase 2: Control, Update, Discover, Persist: spread, discover / collect data
-Phase 3: Extract and take action: extract data, take action (sell data, ...)


\section{Incident Detection}
Bevor auf ein Angriff reagiert werden kann, muss zuerst bemerkt werden, dass ein solcher sich anbahn oder bereits in vollem Gange ist.

...

Wichtig ist, dass nach einer Incident Detection das Incident Response Team unverzüglich Informiert wird. Ist kein Incident Response Team vorhanden und gibt es in der Unternehmung keine entpsrechende Anlaufstelle, ist das Vorgehen mit....
Keine übereilten Reaktionen, da der Angreifer allenfalls etwas bemerkt

Grossteil Angriffe nicht verfolgt, viele nicht bemekrt

\subsection{Hinweise Netzwerkseitig}
\begin{itemize*}
  \item Ungewöhnlich hohe Netzwerklast
  \item Ungewähnliche Anzahl Firewall-Regelverstösse
\end{itemize*}

\subsection{Hinweise Serverseitig}
\begin{itemize*}
  \item Unbekannte Prozesse
  \item Unbekannte User
  \item Unbekannte Dateien
  \item Ungewöhnliche Systemlast
  \item Dienste laufen nicht mehr
  \item Ungewöhnliche Systemanmeldungen
\end{itemize*}

\subsection{Hinweise durch Intrusion-Detection-Systeme}
Erkennung von Angriffen / Angriffsmustern, nicht verhindern, wenn gut konfiguriert: Frühzeitig Anzeichen erkennen, Beweise sammeln
, Data Loss Prevention Technology

\subsection{Weitere Hinweise}
-Externe Hinweise: Kunden / Partnern / MA, Strafverfolgungsbehörde, Presse, Intrustion-Mapping-Systeme
-RFC 2196, Abschnitt 5.3

\subsection{Meldung eines Vorfalles}
-Richtige Informationen abfragen (Merkblatt / Chekliste)
--Basisinfos: Aktuelle Uhrzeit, Wer / Welches System berichtet Vorfall, Art und Weise Vorfall, Vermuteter Zeitpunkt Vorfall, mittelbar / unmittelbar betroffene HW / SW, evtl. Auswirkungen, Schaden, Kontakstelle für ISR und Ermittler
--Infos über betroffenes System sammeln (!! möglichst nicht vom System abfragen, Datenklassifizierung? Klassifizierung? Ort?, Physischer Zugang?  allgemeiner Systemzustand,)
--Angreifer: Infos? noch aktiv? Systeme / Daten manipuliert / zerstört, Vermutungen?
--Getroffene Massnahmen / System verändert? Andere Perosnen benachrichtigt?

Beurteilung Vorfall / Störung: Kentnisse aktueller Status, Organisation, Landschaft, MA, nicht zu lange warten, Durchleuchtug / Ausschlussverfahren

Bei Einbruch: erste Risikoabschützung für möglcihe Abschaltzung / Netzdekonnektion, Berücksichtigung weiterer Ermittlungsschritte

Entscheid Abschaltung / Dekonnektion: Management der Systemeigentümer, Basis: Empfehlung ISR

Klassifizierung  des Vorfalls: Probing  / Portscanning, Denial-of--service-Angriff, Unberechtigter Zugriff auf User-Account, ... Admin-Account, Datendiebstahl / -manipulation,
\section{Incident Response Team}
Rasch und richtig handeln
Eingreiftruppe bei Incidents, Häufig durch Situation bedingt, wer wegen Detailkenntnisse in Gruppe gehört,
Augenmerk: Erfahrene Personen für Schlüsselpositionen, Integrität und Zuverlässigkeit MA, passendes Persönlichkeitsprofil (gesunder Menschenverstand,Fähigkeit effiziente und annehmbare Entscheide zu treffen in krit Sit., gute Kommunikationsfähigkeiten, an Regeln / Prozeduren halten, Arbeiten unter Stress, Teamfähig, Vorbild Sicherheitsrelevante Tätigkeiten, Priorisierung utner Stress)

Bei grösseren Organisationen / IT-Abhängige: Evtl. Dauerhaft in Belegschaft, Wahrnehmung anderer Sicherhietsaufgaben, Früherkennung, Personen: Leiter, Kontaktperson bei Verdacht, etc., Spezialist: Erfassung / Behanldung Vorfall, Spezialist: Schwachstellen, Spezialist auf Plattform, Schulungspersonal

Wichtig: Koordinator, direkter Zugang zum Mgmt


\section{Incident Response}
Nach dem der Angriff entdeckt wurde oder Anzeichen für einen zukünftigen Angriff bestehen müssen entsprechende Massnahmen in die Wege geleitet werden. Die einzuleitenden Massnahmen sind dabei von der gewählten Reaktionsart abhängig. Es sind folgende zwei Reaktionen denkbar

Teil der Computer Forensik

Bei Einbruch: in kurzer Zeit: Schaden, Angriffsmethoden und mögliche weitere Auswirkungen für Org. beurteilt werden
Notwendig: guter Beweissicherungsmassnahmen im Prozess etablieren

Guter Incident Response Prozess / erfolgreicher Ablauf Incident Response: Basis für juristische Verfolgung

Wichtig: Ermittlung Ursaxche, grundlage für zukünftige Handlungsempfehlungen

organisatorische Vorarbeit notwendig, um korrekt reagieren zu können. wenn nicht: im Entscheidenden Moment keine Ressourcen
-Incident Awareness: Beteiligte MA, Bewusstsein
-Grobes Konzept Sicherheitsvorfallbehandlung (Eskalations- / Alarmierungsregelung, Weisungskompetenzen)
-Security-Monitoring- und Alarmierungskonzept (Einbezug: Personalvertreter, Datenschutzbeauftragter für Datenauswertung)
-Weiterbildungen: Incident-Detection / Response
-Kontakt zu Security-Spezialisten / Ermittlungsbehörden aufbauen
-....

Strategie, zwei Aspekte berücksichtigen: direkten / indirekten Schaden minimieren, Tathergang möglichst umfassend rekonstruieren zur Identifikation Tatverdächtige, jeder Sicherheitsvorfall erfordert andere Strategie
-Kritkialität System im Bezug auf Unternehmensprozesse
-Kritikalität / Wichtigkeit gestohlene Daten
-Täger-Vermutung?
-Vermutung Fähigkeiten / Wissen beim Täter
-Vorfall an Öffentlichkeit gelangt?
-Wie weit ist der Täter gekommen
-Verkraftbare Downtime?
-Vermuteter finanzieller Gesamtverlust

-Kein unüberlegter Gegenangriff, Angriffer bemerkt, evtl. zerstörung,
-Honeypots

Infos: Business Impact, ...
Vorfall analysieren, Schlüsse ziehen, Lerneffekt, Ermittlungsvorgang analysieren (Optimierung / Verbesserung), Reaktionszeit, Wirksamkeit, Tätermotivation, Kosten

\begin{itemize}
  \item Härtung der Systeme, Abwehr des Angriffes
  \item Abwarten, Beobachten, Informationen sammeln.
\end{itemize}


\subsection{Reaktionsarten}
Bei der Auswahl von geeigneten Massnahmen ist immer auch der Zeitpunkt der Angriffes zu beachten.

\begin{itemize*}
  \item Angriff in der Zukunft
  \item Angriff ist am Laufen
  \item Angriff ist schon vorbei
\end{itemize*}


\subsubsection{Härtung der Systeme, Abwehr des Angriffes}

\subsubsection{Abwarten, Beobachten, Informationen sammeln}

\begin{itemize}
  \item Eigentlicher Angriff noch ausstehend, Härtung und Abwehr
  \item Eigentlicher Angriff noch ausstehend, Abwarten, Beobachten, Informationen sammeln, Backtracing
  \item Angriff am Laufen, Abwehr (Härtung)
  \item Angriff am Laufen, Abwarten, Beobachten, Informationen
\end{itemize}


Evtl. übergreifendes Kapitel
\section{Ablauf}
\begin{itemize}
  \item Systemeinbruch oder normale Betriebsstörung?))
  \item Wahrnehmung / Bemerkung ungewöhnliche Aktivitäten (Administrator, Anwender, ...)
  \item Evtl. weitere Beobachtung
  \item Kurze Analyse / Sammlung von Spuren
  \item Bestätigung Verdacht
  \item Meldung an Incident Response Team
  \item Sicherstellung elektronische Beweise
  \item Beweisspuren identifizieren
  \item Beweisspuren analysieren
  \item Analyseergebnisse interpretieren / verifizieren
  \item Analyseergebnisse in Bericht zusammenfassen / präsentieren.
\end{itemize}

-Identify
-Assess (if is security incident), check, gather information
-Respond: Kick of procedure
--Initial Response: Determine origin, idenitfy compromised systems, evtl. disconnect from nw, untersuchung starten, Anzeige: Sicherung beweise
--Recovery
Report
Review

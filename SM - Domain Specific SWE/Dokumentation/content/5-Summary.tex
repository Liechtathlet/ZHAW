\chapter{Schlusswort \& Fazit}

\section{Fazit}
Bereits heute gibt es viele Standards, Protokolle, Frameworks und komplette \gls{acr:IOT}-Plattformen. Es haben sich jedoch erst wenige übergreifenden Standards etabliert. Auf den gesamten Markt gesehen, konnte sich noch kein einheitlicher Standard durchsetzen Im Bereich der drahtlosen Übertragungsprotokolle kämpfen Z-Wave und ZigBee und in naher Zukunft auch 6LoPWAN um die Vorherrschaft. Z-Wave und ZigBee haben jeweils rund 300 Hersteller auf ihrer Seite, welche entsprechende Produkte herstellen und vertreiben. Die Angebotspalette an Produkten welche einer dieser Standards verwendet ist recht gross und beinhaltet unterschiedlichste Geräte.

Für die Endkonsumenten resultiert das Ganze in einem Wirrwar an verschiedenen Produkten und (In-)Kompatibilitäten. Der Konsument muss sich für eine Seite entscheiden, da die verschiedenen Standards und Produkte nicht miteinander kompatibel sind. 

Im Bereich Standards und Protokolle zur Device-Discovery sind einige Ansätze in \gls{acr:COAP}, Z-Wave und ZigBee vorhanden. Aber auch hier konnte sich noch kein einheitlicher Standard durchsetzen.

Das \gls{acr:IOT} hat ein enormes Potenziell, welches jedoch nur voll ausgeschöpft werden kann, wenn alle Geräte miteinander kompatibel sind.

\section{Reflexion}
Nach dem die anfängliche Euphorie eine Arbeit über ein innovatives Thema wie das Internet der Dinge zu schreiben verflogen war, habe ich rasch gemerkt, dass das Ganze nicht so einfach werden wird. Durch die Vielschichtigkeit und verschiedenen Ausprägungen der einzelnen Elemente waren viele Recherchen notwendig. Auch im Bereich der Standards, Protokolle und Frameworks benötigte ich viel Aufwand für die Recherchen um ein möglichst umfassendes Bild zu erhalten. Da das Thema \gls{acr:IOT} an Sich nicht neu ist, aber doch erst in den letzten Jahren richtig aufgekommen ist, fanden sich sehr unterschiedliche Ansichten, Meinungen und Fakten zum gesamten Themenkomplex. 

Im Endeffekt habe ich viele neue und interessante Sachen gelernt und konnte viel von den getätigten Recherchen profitieren. Ich denke, dass sich das gesammelte Wissen sicherlich bei einer späteren Gelegenheit als nützlich erweisen wird.

Ursprünglich hatte ich geplannt einen Praxistest, beziehungsweise eine Demo eines bestimmten Frameworks (iotivity) zu machen. Dies ist jedoch an technischen Schwierigkeiten und an der fehlenden Zeit zur Umsetzung gescheitert. 

Ich werde mich sicherlich auch in Zukunft ab und zu mit dem Thema \gls{acr:IOT} beschäftigen und das eine oder andere Privat- / Bastelprojekt realisieren.
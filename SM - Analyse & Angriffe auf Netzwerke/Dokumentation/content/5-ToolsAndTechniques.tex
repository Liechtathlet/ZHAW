\chapter{Tools und Techniken} \label{chap:ToolsTechniques}
In diesem Kapitel werden die grundlegenden Tools und Techniken einer forensischen Analyse vorgestellt. Die Tools und Techniken werden dabei nach der Phase, in der diese eingesetzt und angewendet werden, gegliedert.



%---------------------------------------------------------------------------------------------------------------
\section{Readiness}

\subsection{Datenträger löschen}
EnCase, statisches überschreiben (nur einmal), 

Attach to System, Windows, Start EnCase, Tools -> Wipe Drive, Select Device, Defaults as are, Next, HECF; S. 69
Kein WippingUtility, bei spezialfällen / wenn notwendig: Wipeutility: Linux: "`dd if=/dev/random of=/dev/<image drive>"'


%---------------------------------------------------------------------------------------------------------------
\section{Secure}

\subsection{Sicherung flüchtiger Daten}

\todo{CF: Seite 210 / 2011}
\todo{CFH: S. 1 - 35}

\subsection{Forensische Duplikation}
Varianten: Ausbau, Anschluss saubere Platte an System, Kopie via netzwerk
Writeblocker

Versteckte Bereiche auf Datenträgern: Host Protected Area, Device Configuration Overlay --> Verfahren, dass auch diese Daten sichert, Checksummen

\subsection{Verifizierung eines forensischen Duplikates oder eines Beweisstückes}
EnCase, Tools, Verify Single Evidence File
Linux: md5sum "`image file"', compare


\subsubsection{Tool}
EnCase (Seite 72) DOS Boot disk
EnCase (Windows)
Linux: Device Name ermitteln: /proc/partitions oder logs, dann: "`dd if=/dev/<suspect drive> of=/some dir/image name"', "`md5sum /some dir/image name"', "`md5sum /dev/<suspect drive>"'
Midifzierte Version von dd, dcfldd für Forensik: http://dcfldd.sourceforge.net

FreeHelix

FTKImager (Windows)


%---------------------------------------------------------------------------------------------------------------
\section{Analysis}

\subsection{Gelöschte Datenträger}
Datenträger-Löschsoftware, welche wurde eingesetzt? (noch installiert? typische Spuren?) arbeiten nicht immer zufälligen Löschpattern, evtl. Löschsoftware nicht ganz zuverlässig --> Temporäre Dateien, Registry, Protokolle, ...

\subsection{Untersuchung der Shell}
\todo{HECF: S. 171}

\subsection{Untersuchung der Druckerjobs und der Druckerqueue}
\todo{HECF: S. 171}

\subsection{Untersuchung der Dateien / Dateiendungen}
\todo{HECF: S. 172}

\subsection{Untersuchung von User Aktivitäten}

\subsection{Vertiefte Analyse bei Verdacht auf Anti-Forensik-Techniken}


File-Slack: CF S. 109
%---------------------------------------------------------------------------------------------------------------
\section{Reporting}




%-------------------------------------------



Tools: CF: S. 175




http://www.computer-forensik.org/tools/
\section{Kommerzielle Tools}
-SMART
-Helix (ein Teil Freeware)


\section{Tool-Matrix}
\begin{table}[H]
\centering
\caption{Verfügbarkeit der Tools auf verschiedenen Betriebssystemen}
\begin{tabular}{r | c | c | c | c}
Name 					&	\THrot{Windows}		&	\THrot{Linux}		&	\THrot{Mac OSX}		& 	\THrot{Weitere}		\\
\midrule
Test 					&	x			&				&	x			&			\\	
\bottomrule
\end{tabular}
\end{table}

\section{Technik-Matrix}
\begin{table}[H]
\centering
\caption{Einsatz der Techniken auf verschiedenen Betriebssystemen}
\begin{tabular}{r | c | c | c | c}
Name 					&	\THrot{Windows}		&	\THrot{Linux}		&	\THrot{Mac OSX}		& 	\THrot{Weitere}		\\
\midrule
Test 					&	x			&				&	x			&			\\	
\bottomrule
\end{tabular}
\end{table}

\section{Tool-Technik-Matrix}
Nur Tool-Suiten

\begin{table}[H]
\centering
\caption{Tools: Unterstützte Techniken}
\begin{tabular}{r | c | c | c | c}
Name 					&	\THrot{Tool1}		&	\THrot{Tool2}		&	\THrot{Tool3}		& 	\THrot{Tool4}		\\
\midrule
Test 					&	x			&				&	x			&			\\	
\bottomrule
\end{tabular}
\end{table}
% !TeX encoding=utf8
% !TeX spellcheck = de_CH_frami

\chapter{BPM auf dem "`Raspberry PI"' in der Domäne "`Home Automation"'}
In diesem Kapitel wird analysiert wie Business Prozesse auf einem Raspberry PI implementiert, beziehungsweise automatisiert werden können. Dabei werden verschiedene Lösungskategorien aufgezeigt und erläutert.


\section{Der Raspberry PI}
\begin{figure}
  \centering
  \includegraphics[width=8cm]{./images/RaspberryPi2ModelB}
  \captionsource{Raspberry Pi 2 Model B}{\url{https://www.raspberrypi.org/wp-content/uploads/2015/01/Pi2ModB1GB_-comp.jpeg}}
\end{figure}
  
Der Raspberry Pi ist ein Einplatinencomputer, welcher von der Raspberry Pi Foundation entwickelt und vertrieben wird. Er hat ungefähr die Grösse einer Kreditkarte und bietet zahlreiche On-Board Schnittstellen wie USB-, HDMI und Audio Anschlüsse (Abhängig vom konkreten Modell). Zusätzlich stehen eine bestimmte Anzahl an GPIO-Pins (General Purpose Input / Output) zur Verfügung. Mit Hilfe dieser Pins lassen sich zum einen Erweiterungs-Boards anschliessen und zum anderen können auch über ein spezielles Erweiterungsboard eigene Schaltungen, etc. gebaut und verlötet werden. Die Anzahl und genaue Funktion der einzelnen GPIO-Pins ist vom konkreten Raspberry PI Modell abhängig.

\begin{figure}[H]
  \centering
  \includegraphics[width=13cm]{./images/RaspberryPi2ModelBPlusOverview}
  \captionsource{Raspberry Pi 2 Model B Überblick}{\url{https://www.elektronik-kompendium.de/sites/raspberry-pi/bilder/19052512.jpg}}
\end{figure}

\newpage
\begin{landscape}

\subsection{Raspberry Pi Modelle im Überblick}
\begin{table}[H]
\centering
\begin{tabular}{r | c  | c | c | c | c | c | c | c | c | c}
	& \THrot{\textbf{Raspberry Pi Model A}}
	& \THrot{\textbf{Raspberry Pi Model A+}}
	& \THrot{\textbf{Raspberry Pi Model B}}
	& \THrot{\textbf{Raspberry Pi Model B+}}
	& \THrot{\textbf{Raspberry Pi 2 Model B}}
	& \THrot{\textbf{Raspberry Pi 3 Model B}}
	& \THrot{\textbf{Raspberry Pi Compute}}
	& \THrot{\textbf{Raspberry Pi Zero}}\\
\midrule
Gewicht in Gramm
	& 	31
	&	23
	&	40		
	& 	45 
	&	40
	&	40
	&	7
	&	9\\
\midrule
System-on-a-Chip (SoC):
	& 	\multicolumn{4}{|c|}{BCM2835} 
	&	BCM2836
	&	BCM2837
	&	\multicolumn{2}{|c|}{BCM2835} \\
\midrule
CPU Kerne
	& 	1
	&	1
	&	1		
	& 	1 
	&	1
	&	4
	&	1
	&	1\\
\midrule
CPU Takt in MHz
	& 	\multicolumn{4}{|c|}{700} 
	&	900
	&	1200
	&	700
	&	1000\\
\midrule
CPU Architektur
	& 	\multicolumn{4}{|c|}{ARMv6 (32-bit)}  
	&	ARMv7 (32-bit)	
	&	ARMv7 (64-bit)	
	&	\multicolumn{2}{|c|}{ARMv6 (32-bit)}  	\\
\midrule
GPU Takt in MHz
	& 	\multicolumn{5}{|c|}{250} 
	&	300/400
	&	\multicolumn{2}{|c|}{250} \\
\midrule
Arbeitsspeicher in MB
	& 	\multicolumn{2}{|c|}{256}  
	&	256 / 512		
	& 	\multicolumn{2}{|c|}{512}  	
	& 	1024 
	&	\multicolumn{2}{|c|}{512}  	\\
	
\midrule
Pins
	& 	26
	&	40
	& 	26
	& 	\multicolumn{3}{|c|}{40}  	
	&	60
	&	40  	\\
	
\midrule
GPIO-Pins
	& 	17	
	&	26
	& 	17
	& 	\multicolumn{3}{|c|}{26}  	
	&	48
	&	26  	\\
\bottomrule
\end{tabular}
\end{table}

%Evtl.: http://praxistipps.chip.de/welcher-raspberry-pi-alle-modelle-im-vergleich_41923

\end{landscape}
\newpage

\section{Betrachteter Lösungsraum}
Ursprünglich wären folgende Einschränkungen für den Lösungsraum vorgesehen gewesen:
\blockquote {Da der Raspberry Pi eine offene Plattform ist, gibt es unterschiedlichste Möglichkeiten um das betrachtete Problem zu lösen. Im Kontext dieser Seminararbeit erfolgt die Betrachtung spezifisch für ein Raspberry Pi 2 Model B mit einem Raspbian OS (Debian Distribution für den Raspberry Pi). Als zusätzliche Prämisse gilt ebenfalls, dass der Kern der Anwendung auf dem Raspberry Pi lauffähig sein muss und die Lösung muss es in irgendeiner Form ermöglichen einen Ablauf / Prozess im Bereich Home Automation mit \gls{acr:BPMN} oder \gls{acr:BPEL} abzubilden. Alternative Lösungen, bei denen der Raspberry Pi als "`Client"' / "`Agent"' verwendet wird sind nicht im Fokus dieser Arbeit.}

Die ersten intensiven Recherchen haben gezeigt, dass es keine bis sehr wenige Lösungen gibt, welche den Grossteil der Anforderungen aus dem Lösungsraum erfüllen würden. Daher habe ich mich entschieden, den Lösungsraum anzupassen und zwei verschiedene Kategorien von Lösungen zu ermöglichen.

\textbf{Spezifische Home Automation Lösungen}
\begin{itemize}
\item Lauffähig auf dem Raspberry Pi mit Raspbian (32-Bit)
\item Eine einzige Komponente (Keine Kombination von Komponenten)
\item Open Source / Frei verfügbar (allenfalls Demoversion)
\item Bedienbar via Web
\item Fokus: Home-Automation
\item Funktionalität um Abläufe oder Aktionen zu automatisieren
\end{itemize}

\textbf{Lösung mit \gls{acr:BPMN}-Support im Bereich \gls{acr:IOT}}
\begin{itemize}
\item Lauffähig auf dem Raspberry Pi mit Raspbian (32-Bit).
\item Eine einzige Komponente (Keine Kombination von Komponenten)
\item Open Source / Frei verfügbar (allenfalls Demoversion)
\item Bedienbar via Web
\item Abläufe / Prozesse können mit Hilfe von \gls{acr:BPMN} modelliert werden.
\item Möglichkeit zur Anbindung von \gls{acr:IOT}-Geräten aus dem Bereich "`Home Automation"' (z.B. via Plugins oder Custom-Code).
\end{itemize}


\section{Lösungen, Produkte \& Frameworks }
In diesem Abschnitt werden die recherchierten Lösungen, Produkte und Frameworks aufgezeigt. Diese Auflistungen sind nicht abschliessend und repräsentieren den Stand der Dinge zum Zeitpunkt der Recherchen im Q2/2016.

\subsection{Lösungskategorie: Spezifische Home Automation Lösungen}
Die Inhalte dieser Lösungskategorie wurden aus dem Kapitel \ref{sec:Analyse:HA:LPF} \nameref{sec:Analyse:HA:LPF} entnommen und nach folgenden Kriterien gefiltert:

\textbf{Filterkriterien}
\begin{itemize}
\item Web-basiert
\item Ready-To-Use
\item Kein Framework
\item Trigger \& Action oder Workflow  / Prozess oder BPMN / BPEL Unterstützung
\item Open Source / frei verfügbar
\item Lauffähig unter Raspbian 32-Bit
\end{itemize}

\textbf{Lösungsraum (gefiltert nach Filterkriterien)}
\begin{itemize}
\item TriggerHappy
\item HomeAssistant
\item openHAB
\item Domogik
\item HomeGenie
\item Freedomotic
\item Domoticz
\end{itemize}

Aufgrund eines kurzen Antestens und des daraus resultierenden Eindruckes wurde "`openHAB"' für die Realisierung des Demo-Setups ausgewählt. Die Auswahl erfolgte nicht aufgrund bestimmter Kriterien. Der genaue Setup wird im Abschnitt \ref{sec:AnalyseRPI:Beispiel} \nameref{sec:AnalyseRPI:Beispiel} beschrieben. Ein Fazit zur ausgewählten Lösung wird im Kapitel \todo{Ref zu Kapitel.}


\subsection{Lösung mit BPMN-Support im Bereich IOT}
Der definierte Lösungsraum dieser Lösungskategorie ermöglicht ein breites Spektrum an Lösungen. Nachfolgend werden einige der möglichen Lösungen aufgezeigt. Bei der Auswahl wurden folgende Kriterien berücksichtigt:

\textbf{Filterkriterien}
\begin{itemize}
\item Web-basiert
\item Ready-To-Use
\item Kein Framework
\item Integrierte \gls{acr:BPMN}-Engine
\item Open Source / frei verfügbar
\item Lauffähig unter Raspbian 32-Bit
\end{itemize}

\textbf{Lösungsraum (gefiltert nach Filterkriterien)}
\begin{itemize}
\item \hyperlink{http://activiti.org/}{Activiti BPM Platform}
\item \hyperlink{http://www.jbpm.org/}{jBPM}
\item \hyperlink{https://camunda.com/}{Camunda BPM Platform (Community Edition)}
\item \hyperlink{http://www.imixs.org/}{Imixs Workflow}
\end{itemize}



\section{Realisierung eines Beispielhaften Prozesses mit BPMN im Bereich "`Home Automation"'} \label{sec:AnalyseRPI:Beispiel}
Als Beispiel für diese Seminararbeit wird folgender Setup realisiert:

\todo{Abbildung + Beschreibung}

\subsection{Verwendete Hardware}
Für die Realisierung des Beispiel-Setups wurden folgende Hardware-Komponenten verwendet:
\begin{itemize}
\itemBfText{Raspberry Pi 2 Model B}
\itemBfText{Razberry Board}{Das Razberry Board ist ein Raspberry Pi Erweiterungsboard, welches als Z-Wave Controller eingesetzt werden kann.}
\itemBfText{Z-Wave.Me Double Wall Switch}
\itemBfText{Light Bulb}
\itemBfText{WLAN-Dongle}
\end{itemize}

\subsection{Verwendete Software}
Für die Realisierung des Beispiel-Setups wurden folgende Software-Komponenten verwendet.
\begin{itemize}
\itemBfText{NOOBS}

\itemBfText{Raspbian Jessie}

\itemBfText{XMPP-Server: ejabberd}{Die Installation und Konfiguration des Servers erfolgte anhand der Anleitungen von \hyperlink{https://www.digitalocean.com/community/tutorials/how-to-install-ejabberd-xmpp-server-on-ubuntu}{Digitalocean} und \hyperlink{https://box.matto.nl/ejabberdjessie.html}{Matto....?}. Die Verknüpfung mit OpenHab erfolgte gemäss der Dokumentationen im OpenHab-Wik (\hyperlink{https://github.com/openhab/openhab/wiki/Actions\#xmpp-actions}{Action-Bindings}, \hyperlink{https://github.com/openhab/openhab/wiki/Feature-Overview}{UI's})}


\item Apache tomcat 
https://www.digitalocean.com/community/tutorials/how-to-install-apache-tomcat-7-on-ubuntu-14-04-via-apt-get
change port to 8081 in server.xml
\item Activiti
\hyperlink{http://activiti.org/userguide/index.html\#_getting_started}{Anleitung}

\itemBfText{screen}{}
\end{itemize}


\subsection{Komponentenübersicht}

\todo{Image}


\subsection{Szenario}
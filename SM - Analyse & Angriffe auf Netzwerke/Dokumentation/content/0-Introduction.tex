% !TeX encoding=utf8
% !TeX spellcheck = de_CH_frami

\chapter{Einleitung}

\section{Hintergrund}
Im Rahmen meines Bachelor-Studiums in Informatik an der \gls{acr:ZHAW} muss im 6. Semester eine Seminararbeit zu einem vorgegebenen Themenbereich erarbeitet werden. Ich habe mich für den Themenbereich "`Analyse und Angriffe auf Netzwerke"' entschieden.

Es einem Themenkatalog konnte ein spezifisches Thema im Bereich "`Analyse und Angriffe auf Netzwerke"' ausgewählt werden. Ich habe mich für das Thema "`Computer Forensik"' entschieden.

Für die Arbeit sollen circa 50 Arbeitsstunden aufgewendet werden. Dies entspricht etwa einem Umfang von 15 bis 20 Seiten. Zusätzlich gelten die Rahmenbedingungen gemäss dem Reglement zur Verfassung einer Seminararbeit (\cite{ZHAW:2012:Seminararbeit:Reglemente})

\section{Aufgabenstellung}
In dieser Arbeit soll ein Überblick über das Themengebiet der "`Computer Forensik"' erarbeitet werden. Es soll gezeigt werden was für Themenbereiche es gibt und was für Werkzeuge und Tools eingesetzt werden können. Das Ganze soll mit einem Ablauf einer forensischen Untersuchung und entsprechenden Beispielen illustriert werden.

\section{Abgrenzung}
Aufgrund des grossen Themengebietes können nicht alle Detail-Aspekte der Computer Forensik berücksichtigt werden. Daher werden in dieser Arbeit nur die wichtigsten Aspekte der Computer Forensik näher betrachtet.

\todo{Weitere Abgrenzungen}

\section{Motivation}
\subsection{Computerkriminalität}
Stetiger Zuwachs an
Themenkreis: Ausführung von Taten in Kenntnis bzw. unter Einsatz von Computer- bzw. Kommunikationstechnologie, die Verletzung von Eigentum an Sachwerten sowie Verfügungsrechten an immateriellen Gütern und die Beeinträchtigung von Computer- bzw. Kommunikationstechnologien.

Erweiterter Bereich: Sämtliche Straftaten, die mit Hilfe oder Unterstützung von informationsverarbeitenden Systemen vorgenommen werden

Delikte: Computerbetrug, Betrug mit Zugangsberechtigungen zu Kommunikationsdiensten, Betrug mit Konto- oder EC-Karten mit PIN, Private Softwarepiraterie, Gewerbsmässige Softwarepiraterie, Datenveränderung und Computersabotage, Fälschungn beweiserheblicher Daten, Täuschung im Rechtsverkehr bei Datenverarbeitung, Ausspähen von Daten

Angreifer: Cyberkriminelle, Konkurrenten, Nachrichtendienste, Hackers, Hacktivisten, Mitarbeiter

\section{Struktur}
\todo{Struktur erklären}
Diese Arbeit gliedert sich in folgende Hauptteile:
\begin{itemize}
\item Ausgangslage
\item Analyse
\item Evaluation
\item Schlusswort
\end{itemize}

Im ersten Kapitel werden die Details zur Ausgangslage und die Hintergründe der Arbeit aufgezeigt. Im zweiten Kapitel wird mit Hilfe einer Umfrage innerhalb des Turnvereins eine Analyse erstellt. Aus dieser Analyse gehen die Randbedingungen, Ziele und Anforderungen an das Mini \gls{acr:ERP} System hervor. Diese Randbedingungen, Ziele und Anforderungen werden im Kapitel 'Evaluation' als Kriterien für die Vorselektion, Selektion und anschliessenden die Evaluation der Produkte verwendet. Im letzten Kapitel wird ein Fazit gezogen, eine Empfehlung an den \gls{acr:TVT} abgegeben und über die gesamte Arbeit reflektiert.


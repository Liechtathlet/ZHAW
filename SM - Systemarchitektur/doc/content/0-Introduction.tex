% !TeX encoding=utf8
% !TeX spellcheck = de_CH_frami

\chapter{Einleitung}

\section{Hintergrund}
Im geschäftlichen Umfeld wird die Definition, die Modellierung und dadurch auch die Automatisierung von Business-Prozessen immer wichtiger. Mit der zunehmenden Beliebtheit von Home Automation Produkten (Stichwort: \gls{acr:IOT}) gibt es auch im privaten Umfeld Abläufe welche übergreifend koordiniert und gesteuert werden müssen oder können.

\section{Ziel}
Ziel dieser Arbeit ist es, ein praktisches Beispiel einer Implementation einer
End-to-End Prozess-Steuerung im Home Automation Bereich zu realisieren.

\section{Aufgabenstellung} \label{sec:Aufgabenstellung}
Die freigegebene Aufgabenstellung lautet wie folgt:

Es soll aufgezeigt werden, was es heute für Möglichkeiten (Frameworks, Tools,
Produkte, etc. ...) gibt um Abläufe / Prozesse im Bereich Home Automation zu
modellieren und zu automatisieren. Mit Hilfe eines Raspberry PI's soll einer oder
mehrere typische Abläufe aus dem Home Automation-Bereich realisiert werden.
Dabei soll auch geprüft / gezeigt werden, wie die Interaktion mit typischen
Home Automation Protokollen / Stacks gestaltet werden kann.

\newpage
\section{Erwartete Resultate} \label{sec:ErwarteteResultate}
Gemäss freigegebener Aufgabenstellung werden folgende Resultate erwartet:

\begin{itemize}
\item Dokumentation
\item Handout
\item Präsentation
\end{itemize}


\section{Abgrenzung} \label{sec:Abgrenzung}
Aufgrund des Umfanges der Arbeit und der begrenzten Zeitdauer werden folgende Punkte von der Arbeit abgegrenzt:

\begin{itemize}
\itemBfText{Recherche}{Die Recherchen beschränken sich auf die zentralen betrachteten Elemente. Ebenfalls wird der Zeitaufwand eingeschränkt, welcher für Recherchen aufgewendet wird um die Ziele dieser Arbeit zu erreichen.}
\itemBfText{Detailbetrachtung der Lösungen}{Es werden nicht alle recherchierten Lösungen und Möglichkeiten betrachtet. Es wird eine Auswahl von 1 - 2 Lösungen getroffen, welche näher betrachtet und im Test-Setup verwendet werden. Eine zentrale Rolle bei der Auswahl der Lösungen ist die aktuell verfügbare Hardware. Aktuell sind ein Raspberry Pi 2, diverse GrovePi Sensoren, ein Razberry Z-Wave-Modul sowie einige Z-Wave Geräte verfügbar, welche im Projekt verwendet werden können.}
\end{itemize}


\section{Struktur}
Diese Arbeit ist in folgende Teile gegliedert:

\begin{itemize}
\item Einleitung
\item Ausgangslage
\item BPM in der Domäne "`Internet of Things"'
\item BPM in der Domäne "`Home Automation"'
\item BPM auf dem "`Raspberry Pi"' in der Domäne "`Home Automation"'
\item Fazit
\end{itemize}

Im ersten Kapitel werden die Details zur Aufgabenstellung und den Rahmenbedingungen dieser Arbeit aufgezeigt. Anschliessend werden in einem kurzen Kapitel die wichtigsten Aspekte der Ausgangslage aufgezeigt.

Der Kern der Arbeit besteht aus den drei Kapiteln zu Business Prozess Management in den Bereichen "`\gls{acr:IOT}"', "`Home Automation"' und "`Raspberry PI"'. Im ersten dieser drei Kapitel wird die Situation bezüglich Business Prozessen im Bereich \gls{acr:IOT} aufgezeigt und analysiert. Im Anschluss folgt eine Vertiefung der einzelnen Aspekte im Kontext von "`Home Automation"'. Ausgehend von den Erkenntnisse im "`Home Automation"' Teil wird anschliessend geprüft, was für Möglichkeiten es für eine Umsetzung auf dem "`Raspberry Pi"' aktuell gibt.

Am Ende folgt ein Fazit zu den einzelnen Bereichen und eine Reflexion über die gesamten Seminararbeit.

\chapter{Software Engineering in der Domäne "`Internet of Things"'}


Evtl. Struktur nach kategorie: Connectivity, Management, Security, API, ...

Real-Time schwierig, trade off datenverlust, oft TCP als Datenbasis -> messung mit Quality of service besser

http://postscapes.com/internet-of-things-software-guide
https://www.silabs.com/Support%20Documents/TechnicalDocs/bringing-the-internet-of-things-to-life.pdf
http://www.datamation.com/open-source/35-open-source-tools-for-the-internet-of-things-1.html
https://blog.profitbricks.com/top-49-tools-internet-of-things/
http://www.businesscloudnews.com/2015/05/14/samsung-announces-open-internet-of-things-platform/
https://www.rti.com/company/news/iot-connectivity-webinar.html
http://www.iotsworldcongress.com/documents/4643185/4c3cc80c-03b0-41be-baf0-6a1a0fa7db07
https://www.mapr.com/blog/key-requirements-iot-data-platform#.VVuNi3W1FBc


SWE:
https://hal.inria.fr/hal-01064075/document
http://ieeexplore.ieee.org/xpl/articleDetails.jsp?arnumber=7030178
http://www.cio.com/article/2843814/developer/how-to-develop-applications-for-the-internet-of-things.html
http://www.appdevelopersalliance.org/internet-of-things/
http://www.fit.fraunhofer.de/de/fb/ucc/lehre/scrum-based_softwaredevelopmentofinternet-of-thingsapplications.html
http://insights.wired.com/profiles/blogs/winning-strategies-software-development-for-the-internet-of#axzz3anWNtzAE
http://www.intel.com/content/www/us/en/internet-of-things/white-papers/developing-solutions-for-iot.html
http://link.springer.com/chapter/10.1007%2F978-3-642-36818-9_6#page-1

\section{Programmiersprachen}

\subsection{Assembler}
Eine Implementation in Assembler kann unter bestimmten Voraussetzungen die beste Lösung für ein bestimmtes Problem sein. In der Regel wird der Assembler-Ansatz gewählt, wenn das Programm so effizient und sparsam wie möglich ablaufen und mit so wenig Ressourcen als möglich auskommen soll.



\subsection{C / C++}


\subsection{Java / .NET}
Für die Implementation kann es durchaus sinnvoll sein eine Hochsprache, wie Java oder .NET C\# einzusetzen. 

\section{Protokolle und Standards}

Grafik: http://electronicdesign.com/embedded/understanding-protocols-behind-internet-things 

Viele Implementationen:
MQTT: Collect device data, send to server, D2S
CoAP
XBEE
XMPP: Access device data, connect device to people, D2S
DDS: Distribute Device Data, fast bus for integrating intelligent machines (D2D)
HTTP
AMQP (IOT-Client-Seitig?), Queuing system to connect servers to servers (S2S)

https://www.sparkfun.com/news/1705
http://micrium.com/iot/internet-protocols/

\subsection{MQTT}
Message Queue Telemetry Transport, Device Data Collection, Haupfaufgabe: Fernmessung, Remote Monitoring, Datensammeln und an Infrastruktur ausliefern, Anwendungszweck: Grosse Netzwerke von kleinen Geräte, welche überwacht und kontrolliert werden müssen. Basis: TCP (kein Datenverlust)

Hub and Spoke System: Geräte --> Data Verbindung zu Server, System ist designt, um daten an enterprise technologien weiter zu geben (z.B. ESB)

Kein Protokoll für D2D, nicht mehrere Empfänger der Daten, wenige Kontrollmöglichkeiten, muss nicht schnell sein --> kein rela time (sekunden)


Bsp: ÖL Pipeline, Energieverbrauch Überwachung, Licht-Kontrolle

\subsection{XMPP}
Ursprünglich: Jabber, entwickelt für Instant messaging
Extensible Messaging and Presence Protocol

Text-Kommunikation zwischen Punkten

XML, über TCP (oder HTTP over TcP?)

Stärke: Adressierungs-Schema: name@domain.com, Security, Skalierbarkeit 
--> Ideal für Consumer-Oriented applications

Einfacher Weg Gerät zu adressieren, nicht schnell --> Polling oder Check for Updates on demand, 

Einsatz: z.B. Heimautoation --> Thermostat mit Web verbinden


\subsection{DDS}
Data Distribution Service, Geräte welche direkt Geräte-Daten verwenden, Verteilung zu anderen Geräten, Interaktion mit Infrastruktur unterstützt, Daten-zentrierter Middleware-Standard, Wurzeln: High-Performance Verteidigung, Industrie, Embeddedd Applications, effektiv: millionen von nachrichten pro sekunde zu mehreren gleichzeitigen empfängern


Publish, Subscribe Architecture

Unterschied: Daten an Infrastruktur oder an anderes Gerät, Geräte sind schnell, mit vielen Geräten kommunizieren, TCP Point to Point to restriktiv, DDS: Detailierte qualitiy of service conrol, multicast, konfigurierbare verfügbarkeit, Redundanz

Filter und Selektion von Daten, bzw. Bestimmung was wohin geht

Direct Device-to-device bus, relational data model, ähnlich Datenbank, 

Z.B.: Militär, Windparks, Asset-Tracking, Fahrzeug Test und Sicherheit


\subsection{AMQP}
Advanced Message Queuing Protocol, ab und zu: IOT-Protokoll, transaktionsbasierte Nachrichen zwischen Servern, message-centric-middleware, proces thousands of reliable queued transactions, Fokus: kein Nachrichtenverlust, Publishers --> Exchanges, queues to subscribers: TCP, acknowledge acceptance of message, optional transaction mode with formal multiphase commit sequence

Hauptsächlich: Business messaging, device - back-office data centers

IOT: appropriate for control plane oder server-basierte analyse funktionen




\subsection{Thread}
Protokolle: Thread, Netzwerk-Protokoll, Fokus: Security, Low Energy, notwendiger Chip, schon in vielen Geräten vorhanden, gestützt auf 6LoWPAN, IPv6 over Low power Wireless Personal Area Network, 


\subsection{AllJoyn}
Qualcomm entwickelt, anschliessend: Linux Foundation, AllSeen Alliance (Cisco, Microsoft, LG, HTC, ...)
Verbindung, Wartung Geräte in WLAN-Netzwerk, Kontrolle, Benachrichtungs Service, 

\section{Frameworks}
In diesem Kapitel werden einige Frameworks vorgestellt, welche ....

Frameworks --> mobile / web --> Vereinfachung Entwiclung, Abstraktion Implementations-Details, 
apache ISIS?
http://iot.eclipse.org/java/open-iot-stack-for-java.html

\subsection{Kommunikation zwischen \gls{acr:IOT}-Geräten}
AllSeen
IOTIVITY
Z-Wave
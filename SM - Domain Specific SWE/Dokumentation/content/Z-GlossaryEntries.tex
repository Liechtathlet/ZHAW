% !TeX encoding=utf8
% !TeX spellcheck = de_CH_frami

%LOOK: http://tex.stackexchange.com/questions/8946/how-to-combine-acronym-and-glossary

%%% --- Glossary entries

\newglossaryentry{gls:}{name=ERP,
description={Ein klassiches \gls{acr:ERP} System dient dazu interne Prozesse in den Kernbereichen Finanz- und Rechnungswesen, Vertrieb \& Marketing, Personalwesen und Produktion zentral in einem System abzubilden. Heutige \gls{acr:ERP} Systeme bieten neben den Kernbereichen noch zahlreiche weitere Funktionen, welche die klassischen Geschäftsprozesse unterstützen. \cite[S. 486]{Pearson:WirtschaftsInf:LaudonEtAl}}}

\newglossaryentry{gls:JUKO}{name=JUKO,
description={Die Jugendkommission war ursprünglich als organisatorisches Gefäss für alle Jugendlichen, den Nachwuchs, des Turnverein Thalwils gedacht. Per Generalversammlung 2013 hatte die JUKO 121 Mitglieder \cite[S. 296]{TVT:Chronik:150}. Diese setzen sich primär aus Nachwuchs-Leichtathleten und Nachwuchs-Volleyballern zusammen.}}

\newglossaryentry{gls:MERP}{name=Mini-ERP,
description={Ein Mini-ERP-System enthält nur ein subset der Funktionalitäten eines klassichen \gls{acr:ERP}-Systems. Dazu zählen zum Beispiel Module zur Verwaltung von Mitarbeiter- und Organisationsstammdaten, Dokumentenmanagement oder für das Finanz- und Rechnungswesen. }}

\newglossaryentry{gls:TVTInfo}{name=TVT-Info,
description={Vereinsheft des Turnverein Thalwils welches vier Mal pro Jahr erscheint und Bilder, Berichte und Informationen zu aktuellen Themen (Lager, Anlässe, Generalversammlungen, etc.) beinhaltet. (Quelle: Redaktion TVT-Info)}}

\newglossaryentry{gls:STVAdmin}{name=STV-Admin, 
description={Der \gls{acr:STV}-Admin wird vom \gls{acr:STV} allen Vereinen gratis zur Verfügung gestellt und ist eine kleine Vereinssoftware, welche auf die spezifischen Wünsche und Anforderungen von Turnvereinen und insbesondere dem \gls{acr:STV} zugeschnitten sind.}}

%%% --- Acronym definitions
\IfDefined{newacronym}{

\newglossaryentry{acr:ERP}{type=\acronymtype, name={ERP}, description={Enterprise-Resource-Planning}, first={Enterprise-Resource-Planning (ERP) \glsadd{gls:ERP}},see=[Glossary:]{\gls{gls:ERP}}}

\newglossaryentry{acr:JUKO}{type=\acronymtype, name={JUKO}, description={Jugendkommission}, first={Jugendkommission (JUKO) \glsadd{gls:JUKO}},see=[Glossary:]{\gls{gls:JUKO}}}

\newacronym{acr:IOT}{IoT}{Internet of Things}

\newacronym{acr:WSN}{WSN}{Wireless Sensor Network}

\newacronym{acr:DSL}{DSL}{Domain Specific Language}

\newacronym{acr:ZHAW}{ZHAW}{Zürcher Hochschule für Angewandte Wissenschaften}

}






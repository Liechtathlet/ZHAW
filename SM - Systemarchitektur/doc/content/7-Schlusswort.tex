% !TeX encoding=utf8
% !TeX spellcheck = de_CH_frami

\chapter{Schlusswort} \label{chap:Finish}
In diesem Kapitel wird ein Fazit zu dieser Arbeit und den einzelnen betrachteten Bereichen gezogen und über die gesamte Arbeit reflektiert.
%---------------------------------------------------------
\section{Fazit}

\subsection{BPM in der Domäne "`Internet of Things"'}

\subsection{BPM in der Domäne "`Home Automation"'}
Im Bereich "`Home Automation"' gibt es gemäss heutigem Stand nur ...

Die meisten Softwarelösungen und Kombi-Produkte (Software + Hardware) basieren auf dem Konzept von Auslösern (Triggern) und nachfolgenden Ereignissen (Events).

Heimanwender: BPM Schwierig, wenn dann im Bereich der Gebäude-Automatisierung im Bereich von Unternehmen.


OPENHAB:
Version 1: Schwieriger / Komplizierter Setup (insbesondere mit Z-Wave)

Version 2: Einfacherer Setup, viele fehlende Funktionalitäten, nicht intutitiv, Dokumentation für gewisse Grundkonzepte nicht wirklich gut

\subsection{BPM auf dem "`Raspberry Pi"' in der Domäne "`Home Automation"'}
Evtl. Getrennte Fazite für einzelne Bereiche?

Diverse Bindings ausprobiert (XMPP, Telegram, NTP, WOL, Z-Wave, System Info, JPA, MQTT, ), im Beispielssetup konnten nicht alle integriert werden.

\subsection{Allgemein}
Aufgrund der Ergebnisse der Analyse kann gesagt werden, dass aktuell im Bereich der Modellierung und Implementierung von Business Prozessen ein Wandel statt findet. Mit diesem Wandel rückt der Einsatz des \gls{acr:IOT} etwas mehr in den Vordergrund. 

....Bereits erste Produkte / Lösungen mit entsprechendem Support.



%---------------------------------------------------------
\section{Reflexion}




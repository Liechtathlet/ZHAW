\chapter{Schlusswort}

\section{Fazit}
Die Computer Forensik ist ein komplexes und breites Themengebiet. Es sind sowohl Kenntnisse im rechtlichen, juristischen Bereich, als auch sehr detaillierte technische Kenntnisse notwendig. Im technischen Bereich ist ein extrem tiefes Fachwissen in allen Themenbereichen notwendig, um eine umfassende Analyse und Bewertung der Beweise vorzunehmen. Eine juristische Verwertbarkeit erfordert ein exaktes und sehr korrektes Vorgehen, was alleine schon eine Herausforderung darstellt und viel Training, Geduld und Können erfordert. Eine weitere Komplexitätsstufe wird dadurch geschaffen, dass es sich beim zu untersuchenden System un ein sehr altes, ein top modernes oder auch ein eher unbekanntes System handeln kann. Die Analyse muss auf allen Systemen durchgeführt werden können.

\section{Reflexion}
Mit viel Vorfreude habe ich mich an die Recherchen und die Verfassung dieser Arbeit gemacht. Nachdem ich das erste Buch mit entsprechender Fachliteratur durchgelesen hatte und noch zwei weitere Bücher und weitere elektronische Quellen darauf warteten gelesen zu werden, ist mir bewusst geworden wie gross und komplex das Ganze Thema wirklich ist. Ich musste mich dann in gewissen Themenbereichen und auch im praktischen Teil etwas einschränken, um den geforderten Umfang im grossen und ganzen einhalten zu können.

Während meiner Recherchen habe ich viele neue interessante Dinge über die Forensische Analyse, Computer im allgemeinen und Linux im speziellen gelernt. Ich konnte durch diese Arbeit einen Einblick in das spannende Gebiet der Computer Forensik erhalten und habe viele interessante Dinge gelernt, welche mir in Zukunft auch nützlich sein werden.
\chapter{Computer Forensik}
\section{Einbettung und Definition}
\subsection{Forensik}
Forensik, Wissenschaftszweig, Nachweis + Aufklärung strafbare Handlungen,
Begriffs-Ursprung: antikes Rom, Gerichtsverfahren, Untersuchungen, Urteile, Strafvollzug auf dem Marktplatz, öffentlich, lat.: forum: Marktplatz, Forum (Plural: Foren)
Heute: Systematische Analyse, Kriminelle Handlungen identifizieren, analysieren, rekonstruieren, ...

\subsubsection{Teilbereiche}
\begin{itemize}
\item Forensische Phsychiatrie

\item Computer-Forensik
\item Ballistik
\item Rechtsmedizin
\end{itemize}

https://www.mtholyoke.edu/org/forensic/fields.html
http://forensictrak.com/faq.html
http://de.wikipedia.org/wiki/Forensik

\subsection{IT- / Digitale Forensik}
http://de.wikipedia.org/wiki/IT-Forensik
http://www.ehow.com/facts_6733855_difference-computer-forensics-digital-forensics_.html

\subsection{Computer Forensik}
Die Computer Forensik
Einbettung in Forensik, Digitale Forensik


\section{Einführung}
Ziel: nach Systemeinbruch / Sicherheitsvorfall: Methode / Schwachstelle finden, Bezifferung Schaden, Identifikation Angreifer, Beweissicherung für juristische Schritte

\section{Hinweise für die juristische Verwertbarkeit}
-Angewendete Methoden / Techniken / Programme sollten in Fachwelt beschrieben / akzeptiert / gängig sein, neue Verfahren / Tools schwierig, nicht unmöglich
-Glaubwürdigkeit: Schritte / Ergebnisse müssen Nachvollziehbar sein (im Detail), Ermittler muss Ablauf / Hintergrund erklären können, (Programm mit Daten füttern, loslassen)
-Sämtliche Schritte / Methoden müssen reproduzierbar sein (durch Dritte)
-Integrität: sichergestellte Daten dürfen nicht verändert werden, muss jederzeit belegt werden können
-Ursache / Wirkung: Gewählte Methoden: möglichst logische, nachvollziehbare Verbindung person - Ereignis - Beweisspuren
-Angemessene Dokumentation


Wenn am Anfang nicht klar ob jur. Schritte: Trotzdem immer gleichen Prozess verwenden.
Unvoreingenommenheit
\section{Themengebiete und Teilbereiche}

\section{Anwendungsbereich}

\section{Ziele}

\section{Ausbildung \& Zertifizierung}


\section{Sicherungsebenen}
-Hardware-Ebene
-Software-Ebene
--Betriebssystem-Ebene
--Anwendungssoftware-Ebene

\section{Unterscheidung Daten-Typen}
-Flüchtig
-Nicht-Flüchtig

Evtl. Matrix mit Zuordnung Techniken / Themenbereichen zu Typen und Sicherungsebenen

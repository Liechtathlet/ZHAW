\chapter{Computer Forensik}

\section{Einbettung und Definition}
\subsection{Forensik}
\subsubsection{Ursprung}
Der Begriff "`Forensik"' stammt aus den Zeiten des antiken Roms. Damals wurden Gerichtsverfahren, Untersuchungen, Urteilsverkündungen und der Vollzug von Strafen öffentlich auf dem Marktplatz abgehalten. Marktplatz (oder auch Forum) wird im lateinischen mit \it{forum} bezeichnet. Die Plural-Form von \it{forum} ist \it{foren}. Aus dieser Plural-Form hat sich der Begriff "`Forensik"' entwickelt.

Evtl. noch Definitionen aus NIST-Paper Seite ES-1

\subsubsection{Bedeutung}
Die Forensik ist ein Wissenschaftszweig, welche sich mit dem Nachweis, Beweis und der Aufklärung von kriminellen, oder allgemein strafbaren, Handlungen beschäftigt. Die forensische Untersuchung ist eine systematische Analyse mit dem Ziel strafbare Handlungen zu identifizieren, analysieren und rekonstruieren.

\subsubsection{Teilbereiche}
Die Forensik gleidert sich in zahlreiche Unterbereiche. Dazu gehören unter anderem: 
\begin{itemize}
\item Forensische Phsychiatrie

\item Computer-Forensik
\item Ballistik
\item Rechtsmedizin
\end{itemize}

https://www.mtholyoke.edu/org/forensic/fields.html
http://forensictrak.com/faq.html
http://de.wikipedia.org/wiki/Forensik

\subsection{IT- / Digitale Forensik}
%http://de.wikipedia.org/wiki/IT-Forensik
%http://www.ehow.com/facts_6733855_difference-computer-forensics-digital-forensics_.html

\subsection{Computer Forensik}
Die Computer Forensik
Die Computer Forensik
Einbettung in Forensik, Digitale Forensik


\section{Einführung}
Ziel: nach Systemeinbruch / Sicherheitsvorfall: Methode / Schwachstelle finden, Bezifferung Schaden, Identifikation Angreifer, Beweissicherung für juristische Schritte

\section{Hinweise für die juristische Verwertbarkeit}
-Angewendete Methoden / Techniken / Programme sollten in Fachwelt beschrieben / akzeptiert / gängig sein, neue Verfahren / Tools schwierig, nicht unmöglich
-Glaubwürdigkeit: Schritte / Ergebnisse müssen Nachvollziehbar sein (im Detail), Ermittler muss Ablauf / Hintergrund erklären können, (Programm mit Daten füttern, loslassen)
-Sämtliche Schritte / Methoden müssen reproduzierbar sein (durch Dritte)
-Integrität: sichergestellte Daten dürfen nicht verändert werden, muss jederzeit belegt werden können
-Ursache / Wirkung: Gewählte Methoden: möglichst logische, nachvollziehbare Verbindung person - Ereignis - Beweisspuren
-Angemessene Dokumentation


Wenn am Anfang nicht klar ob jur. Schritte: Trotzdem immer gleichen Prozess verwenden.
Unvoreingenommenheit
Gewisse Beweise kurze Halbwertszeit (meist sehr spannend), erfordern besonnenes / koordiniertes erfassen, Bewusstsein, dass System in jedem Fall verändert wird

Sachbeweis: Festplate, Logs, Gutachten, Fingerabdruck - keine Beweiskraft, nicht zugeordnet, keine Aussagekraft alleine, erst im Kontext, Beweiskraft erst druch Person, die Beweis in Tat-ZSH bringt, Sachbeweis eng mit Personenbeweis verbunden
Beweis rasch Bedeutungslos, weniger Beweiskraft, wenn Person unrichtig darstellt, widerlegbare Behauptungen, Interpretationen, dachliche Darstellung, Integrität Person und Glaubwürdigkeit wichtig, sachliches, fundiertes Gutachten durch unglaubwürdige Darstellung als nichtig betrachtet

\section{Hinweise zum Datenschutz}
Datenschutz bei personenbezogenen Daten auch bei Auswertungen zum Zug
Schweizer Recht??

Vorgängige Klärung wenn Logs personenbezogene Daten beinhalten, Information Datenschutzbeauftrager, Security \& Compliance, IT-leiter, REvision, Vier-Augen-Prinzip wahren

DS bei Ermittlung nicht ausser Kraft, aber kein Täterschutz, Verhältnismässigkeit
\section{Themengebiete und Teilbereiche}

\section{Anwendungsbereich}

-Klassisch: Strafuntersuchungen, Computer Security Incident Handling
-Operative Problembehebung
-Log Monitoring
-Data Recovery
-Data Acquisition
-Due Diligence / Regulatory Compliance

\section{Ziele}

\section{Ausbildung \& Zertifizierung}


\section{Sicherungsebenen}
-Hardware-Ebene
-Software-Ebene
--Betriebssystem-Ebene
--Anwendungssoftware-Ebene

\section{Unterscheidung Daten-Typen}
\begin{itemize}
\item Empfindliche Daten
\begin{itemize}
\item Flüchtige Daten, gehen beim geordneten Shutdown / Ausschalten veroren (Cache, Hauptspeicher, Status NWV, Prozesse,...)
\item Fragile Daten, zwar auf HD, Zustand kann sich beim Zugriff ändern
\item Temporär zugreifbare Daten, auf HD, nur zu Bestimmten Zeitpunkten zugreifbar
\end{itemize}
\end{itemize}
-Flüchtig
-Nicht-Flüchtig

Evtl. Matrix mit Zuordnung Techniken / Themenbereichen zu Typen und Sicherungsebenen

\section{Anti-Forensik}
Verschleierung Spuren, Anti-Forensik-Werkzeuge / Technologien, Be- / Verhinderung Analyse, falsche Spuren, Löscheung / Veränderung relevante spuren

Primäres Ziel: Vorgaukeln, dass nichts verdächtiges vorgefallen ist, Anti-Detection: Tat und Täter sollen unterkannt bleiben, Anti-Forensik: Behinderung Ermittler, Datensammlung stören / unterbinden, Ablenken, Zeitkntingent / -budget Ermittler, falsche Fahrte

Angriff auf Werkzeuge: bei komplexen Angriffen / längere Zeit, Ermittlung eingesetzte Forensik-Werkzeuge, Netzbereich der Ermittler, IP-Analysesysteme, Bugs / Sicherheitslücken in Forensik-Werkzeugen: Manipulation Ermittlung, wenn nicht in geschützten Analyseumgebungen: grosser Schaden durch Angreifer, Verschleierung Anti-Forensik-methoden, eher Anti-Detection: Gebrauchsspuren regelmässig löschen / oder Verhinderung entstehung, Arbeit mit virtuellen Umgebungen (via USB-STicks, o.ä.), Datenträger-Löschsoftware, welche wurde eingesetzt? (noch installiert? typische Spuren?) arbeiten nicht immer zufälligen Löschpattern, evtl. Löschsoftware nicht ganz zuverlässig --> Temporäre Dateien, Registry, Protokolle, ...

% !TeX spellcheck = de_DE_frami
\chapter{Ausgangslage}

Mit dem laufenden Fortschritt in der Computer- und Kommunikationstechnik und der damit einhergehenden Miniaturisierung und Mobilisierung eröffnen sich immer wieder neue Bereiche in der Informatik. Eines dieser neuen Gebiete wird als "`Internet of Things"' bezeichnet.

Mit dem Fortschritt in der Computertechnik und der fortlaufenden Miniaturisierung und Mobilisierung kommt das \gls{acr:IOT}

Allgemeins Gadgets, IOT



\section{Internet of Things}
Das Internet of Things (IOT), beziehungsweise das "`Internet der Dinge"', 
Nächste Etappe in der durch das Internet ausgelösten REvolution wie Leute kommunizieren und miteinander arbeiten

Dinge Kommunikation mit realer Welt, Dinge müssen zusammenarbeiten, Geschwindigkeit, Skalierung und Fähigkeiten

Vernetzte Stadt --> Beispiel

Veränderung, womöglich mehr, als das human centric Internet


Anzahl Smartphones aktuell? entwicklung?
Potenziall für IOT?

Device2Device Kommunikation, Server2Server Kommunikation, 

http://de.slideshare.net/RealTimeInnovations/io-34485340

http://www.internet-of-things.eu/


Die "`Things"', Dinge, beziehungsweise Geräte
-Embedded Devices / -Systems

Transformation von isolierten Systemen zu vernetzten dingen, 

Ziel von IOT-Lösungen (z.B. mit Cloud Power): Vernetzung von Millionen Geräten, Teilen, Analyzieren, Schlüsse ziehen (Big Data), 

IOT-Lösungen: Verbesserung medical outcome, schneller und bessere Produkte, Entwicklungskosten reduzieren, besseres Shopping vergüngen, Optimierung Energieerzeugung und -konsum - System: Verbessert für Datensicherheit , -datenschutz, mangement von Geräten, Data analyitcs

End-To-End Lösung: Orchestrierung der einzelnen komponenten,  Grafik Seite 10 (Intel White Paper)

Eingebaute, sichere Kommunikation

IOT: Things - Gateway - Network and cloud, grosses ökosystem, 

Dinge: Autos, Geräte-Sensoren, Wearables, Smartphones,  direkt verbunden über Mobiles Netzwerk mit Zugriff auf das Internet, IOT-Solution: entweder Dinge sind intelligent --> filtern / managed von daten lokal oder verbunden mit gateways, welche diese Fkt bieten.

Gateway: 85 \% der Geräte nicht darauf ausgelegt, direkt mit Internet zu kommunizieren (Quelle: Intel: Developing Solutions for IOT White Paper), Gateway als intermediär zwischen legacy things und cloud und anbietung verbindung, sicherheit, management

Network and Cloud: klassisch, Datenanlyse der Rohdaten


\section{Domain Specific Software Engineering}


\section{Domain Specific Languages}
Eine "`Domain Specific Language (DSL)"' bezeichnet eine Programmier-, Modellierungs- oder Metasprache, welche für eine spezifische Domäne entworfen und entwickelt wurde. Eine DSL addressiert dabei spezifische Schwächen und Problemstellungen der angesprochenen Domäne.




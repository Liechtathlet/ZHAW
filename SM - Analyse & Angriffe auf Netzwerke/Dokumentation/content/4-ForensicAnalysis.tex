\chapter{Forensische Analyse} \label{chap:ForensischeAnalyse}
In diesem Kapitel wird der Ablauf der Computer forensischen Analyse im Detail aufgezeigt und erklärt. Die Techniken und Tools zur Unterstützung dieses Prozesses werden im Kapitel \ref{chap:ToolsTechniques} \nameref{chap:ToolsTechniques} erläutert.

% %\todo{http://de.wikibooks.org/wiki/Disk-Forensik/_Richtlinien/_Reihenfolge_bzw._Vorgehensweise_bei_der_Untersuchung}
\section{Einführung}
Der Prozess der forensischen Analyse lässt sich grundsätzlich in die nachfolgenden Phasen unterteilen werden. Die Erläuterung der einzelnen Phasen erfolgt in den nachfolgenden Kapiteln.

\begin{enumerate}
\item Readiness (Vorbereitung)
\item Secure (Sicherstellung)
\item Analysis (Analyse)
\item Documentation (Dokumentation)
\item Present (Präsentation)
\item Review (Rückblick)
\end{enumerate}

Ziel ist es auf den sichergestellten Datenträgern Beweise zu finden. Dazu werden zuerst die Datenträger gesichert und anschliessend die Daten extrahiert. Aus den extrahierten Daten werden dann Informationen gewonnen, welche allenfalls als Beweise im Zusammenhang mit anderen Informationen verwendet werden können.


\subsection{Ein guter Prozess}
Die Grundlage für eine saubere Beweisaufnahme und eine mögliche juristische Verwertbarkeit dieser Beweise ist ein guter forensischer Prozess. Ein guter Forensischer Analyse Prozess zeichnet sich durch folgende Punkte aus:

\begin{itemize}
\item Kreuzvalidierung von essentiellen Ergebnissen mit anderen Tools
\item Sauberer und Korrekter Umgang mit Beweismaterial
\item Untersuchung wird vollständig durchgeführt (Berücksichtigung aller Aspekte)
\item Case-Management für die Verwaltung der Untersuchung
\item Dokumentation und Archivierung der Beweismittel und der Ergebnisse
\item Dokumentierte und geprüfte Arbeitsprozesse
\item Konformität zu gesetzlichen Vorgaben und Restriktionen
\item Flexibilität
\end{itemize}

\section{Phasen}
\subsection{Readiness (Vorbereitung)}
Um während der Untersuchung Fehler zu verhindern und wertvolle Zeit zu sparen, ist es sinnvoll gewisse Vorbereitungsarbeiten vor jedem Einsatz, beziehungsweise vor jeder Untersuchung, durchzuführen.

\textbf{Vorbereitungsarbeiten}
\begin{itemize}
\item Sterilisieren / Formatieren von Datenträgern für die Sicherung des Beweismaterials
\item Formulare und Protokolle vorbereiten und ausdrucken
\item Vorbereitung und Verpackung der notwendigen technischen Ausrüstung
\begin{itemize}
\item Kleines Werkzeugset
\item Digitalkamera
\item Notizblock und Stifte
\item Wasserfeste Filzstifte und Etiketten
\item Antistatische Beutel
\item Dokumente (Manuals, Anleitungen, Abläufe, etc. )
\item Writeblocker
\item Datenträger
\item (Mobiles) Analysesystem + Zubehör (Adapter, USB-Hubs, Card-Reader, Multi-Card-Reader, CD/DVD/Blue-Ray Leser / Brenner, Drucker)
\item ...
\end{itemize}
\item Vorbereitung und Verpackung der notwendigen Tools und Programme
\begin{itemize}
\item Tool für Datensicherungen
\item Tool zum Auffinden von (versteckten) Dateien
\item Tool zur Sicherung und Auswertung der Internet-History und des Caches der verschiedenen Browser
\item Tool zum Öffnen von Multimediainhalten
\item Tool zum Öffnen von E-Mail-Nachrichten
\item Tool zum Öffnen der gängigen Dokumente (PDF, XML, Microsoft-Office Dokumente, ...)
\item Tool zum Knacken von Passwörtern.
\item Tool zur Untersuchung von mobilen Geräten
\item Tool um grosse Datenmengen zu analysieren
\item ...
\end{itemize}
\end{itemize}

\subsection{Secure (Sicherstellen)}
Die Phase "`Secure"' lässt sich weiter in die Phasen "`Environment (Umgebung)"', "`Identify (Identifizieren)"', "`Asses and Decide (Beurteilen und Entscheiden)"', "`Collect (Sammeln)"' und "`Preserve (Aufbewahren)"' unterteilen.


\subsubsection{Environment (Umgebung)}
Diese Phase muss grundsätzlich nur berücksichtigt werden, wenn es sich bei der Untersuchung um eine Ermittlung im Rahmen einer Incident Response oder einer Tatortssicherung handelt.

Bei Ankunft des Ermittlers am "`Tatort"' sollte er sogleich sicherstellen, dass nur noch berechtigte Personen Zugang zum Tatort und der näheren Umgebung haben. Bevor die Personen den Tatort verlassen, sind zum einen die Kontaktdaten für spätere Rückfragen und zum anderen weitere Informationen (zum Beispiel: Passwörter, Besonderheiten des Systems) zu protokollieren. Sofern noch nicht erfolgt, sollte der Tatort isoliert und dokumentiert werden. Für die Tatortdokumentation sind Fotos und Skizzen sehr gut geeignet.

\subsubsection{Identify (Identifizieren)}
Zuerst müssen sämtliche Datenquellen am Tatort und in der näheren Umgebung identifiziert werden. Dazu zählt zum Beispiel das zu untersuchende System, externe Festplatten, USB-Sticks, Digitalkameras, MP3-Player, Wechseldatenträger, etc. Zu den möglichen Datenquellen zählen auch entfernte Systeme wie Firewalls, Router, Internet-Provider, etc. und physische Dokumente, Zettel, Notizen, etc. 

Falls gestattet, sollte in der näheren Umgebung, zum Beispiel im Aktenschrank oder im Korpus, nach weiteren Datenquellen gesucht werden. Eine mehr oder weniger gut versteckte Passwortliste kann die Arbeit erheblich erleichtern. Wurden nicht an ein System angeschlossene Datenträger, Medien  und Dokumente gefunden ist dies sofort zu dokumentieren und der Beweisgegenstand fachgerecht zu sichern. Mehr dazu im nächsten Kapitel.

\subsubsection{Asses and Decide (Beurteilen und Entscheiden)}
Mit dem aktuellen Wissensstand muss der Ermittler nun eine Lagebeurteilung durchführen und eine Strategie für das weitere Vorgehen auswählen. Je nach Situation muss an dieser Stelle das Incident Response Team, ein Mitglied des Managements oder der Ermittlungsleiter involviert werden, um das weitere Vorgehen abzusprechen und zu genehmigen.

Der Ermittler kann entweder eine Live-Analyse oder eine Post-Mortem-Analyse durchführen. Bei einer Live-Analyse wird das System nicht heruntergefahren und es wird versucht so viele Daten als möglich zu sichern, ohne den Zustand des Systemes allzu stark zu verändern, beziehungsweise eigene Spuren zu hinterlassen. Eine Live-Analyse wird typischerweise durchgeführt, wenn ein unternehmenskritisches System betroffen ist, welches nicht heruntergefahren werden kann. Hier gilt es sorgfältig das Kosten / Nutzen-Verhältnis abzuwägen. Handelt es sich um ein kritisches System, der Schaden ist vergleichsweise klein und die Wahrscheinlichkeit den Täter ausfindig zu machen sehr gering, lohnt sich ein herunterfahren aus wirtschaftlicher Sicht oft nicht. Für eine Live-Analyse muss zwingend ein Zugang zum System bestehen.

Bei einer Post-Mortem-Analyse wird der Grossteil der Sicherungs- und Analyse-Arbeiten erst durchgeführt, wenn das System ausser Betrieb genommen wurde. So ist gewährleistet, dass der Datenstand erhalten bleibt und so wenig Beweise als möglich vernichtet oder beschädigt werden. Einer Post-Mortem-Analyse kann auch eine Live-Analyse vorausgehen. In der Regel wird bei einem laufenden System versucht so viele flüchtige Daten als möglich zu sammeln. Sobald diese gesichert wurden, wird das System ausser Betrieb genommen und die weiteren Sicherungs- und Analyseschritte durchgeführt.

\subsubsection{Collect (Sammeln) and Preserve (Aufbewahren)} \label{subsubsec:FA:Secure:CollectPreserve}
Nachdem sicher der Ermittler für eine Strategie entschieden hat, sollte auf Basis der identifizierten Datenquellen und der gewählten Strategie eine Planung zur Sicherung der Daten ausgearbeitet werden. Dieser Plan sollte die Sicherungsreihenfolge der Datenquellen anhand ihrer Priorität und die dazu zu verwendenden Tools und Techniken beinhalten. Häufig ist es nicht möglich, sämtliche Datenquellen zu sichern, daher müssen diese auf Basis der Halbwertszeit der Daten, der Erfahrung des Ermittlers, der Einschätzung der Situation und des Aufwandes (Zeit / Kosten / Ausrüstung) um die Daten zu sichern,  priorisiert werden.

Bei der Sicherung ist zu beachten, dass auf dem System allenfalls trojanisierte Systemprogramme installiert wurden. Es sollten daher für sämtliche Befehle statisch vorkompilierte  Programme verwendet werden. Viele Forensik-Kits bieten einen guten Grundstock von statisch vorkompilierten Anwendungen für die Sicherung und Analyse.

\textbf{System ist eingeschaltet}\\
Dieser Schritt gilt grundsätzlich für beide Analyse-Strategien. Bei einer Live-Analyse sind die Zustände entsprechend zu dokumentieren, jedoch ohne einen Shutdown durchzuführen. 

\begin{enumerate}
\item Nähere Umgebung und Zustand des Systems dokumentieren
\item Befindet sich das System im Standby? \\
Befindet sich das System im Standby ist abzuwägen, ob das System aufgeweckt oder ein harter Shutdown gemacht werden soll. In dieser Situation ist in der Regel ein harter Shutdown  zu empfehlen.
\item Ist der Screensaver aktiv? \\
Ist auf dem System ein Screensaver ist abzuwägen, ob dieser "`deaktiviert"' werden soll oder ein harter Shutdown gemacht werden soll. Dies ist stark situationsabhängig und sollte von Fall zu Fall entschieden werden. Ist anzunehmen, dass die Freischaltung des Screens mit einem Passwort erfolgt, welches nicht bekannt ist, ist auch hier der harte Shutdown zu empfehlen. Wird der Screensaver "`deaktiviert"' muss dies entsprechend mit der exakten Uhrzeit dokumentiert werden.
\item Ist das System durch ein Passwort geschützt? \\
Ist das System durch ein unbekanntes Passwort geschützt, ist in der Regel ein harter Shutdown zu empfehlen. In Ausnahmefällen kann durchaus auch ein Versuch unternommen werden, dass Passwort mit Hilfe von entsprechenden Werkzeugen zu knacken.
\end{enumerate}

Ist der Zugang zum System hergestellt, kann mit der Sicherung der flüchtigen Daten begonnen werden. 

\begin{enumerate}
\item Festhalten des Bildschirminhalts und der geöffneten Anwendungen
\item Festhalten der aktuellen Systemzeit und einer Referenzzeit, sowie deren Abweichung
\item Erstellen eines Abbildes des Hauptspeichers
\item Sicherung des Hauptspeichers pro Prozess-ID
\item Sicherung der Cache- und Auslagerungsdateien
\item Liste der aktiven Prozesse
\item Pro Prozess: Umgebungsvariablen, Übergabeparameter, geladene Bibliotheken, Offene Dateideskriptoren, etc.
\item Liste der geöffneten Sockets
\item Liste der Anwendungen, die auf geöffnete Sockets hören
\item Liste der geöffneten Ports
\item Liste der angemeldeten User
\item Status und Statistik der Netzwerkverbindungen
\item Informationen über das verwendete Betriebssystem
\end{enumerate}


Ist die Sicherung der flüchtigen Daten abgeschlossen sollte das System mit einem harten Shutdown heruntergefahren werden.


\textbf{System ist nicht eingeschaltet}\\
Ist das System ausgeschaltet oder wurde es heruntergefahren wird als erstes das System von der Stromversorgung getrennt. Anschliessend wird das Innen- und Aussenleben dokumentiert und sämtliche Anschlüsse entfernt. Sämtliche Datenträger werden ausgebaut, beziehungsweise entfernt, und beschriftet.

Im Anschluss wird von sämtlichen Datenträgern ein forensisches Duplikat erzeugt. Ein forensisches Duplikat ist eine exakte Kopie (Bitweise 1:1 Kopie) des Quelldatenträgers und sollte immer auf einen Datenträger gesichert werden, der zuvor formatiert wurde. Die Erstellung eines Duplikates ist grundsätzlich immer sinnvoll da dies bessere Analysemöglichkeiten bietet. Ist die Untersuchung ein Teil einer Strafuntersuchung muss in jedem Fall ein Duplikat angefertig werden.

Sind sämtliche Datenträger entfernt, sollte das System gestartet und direkt das BIOS aufgerufen werden. Der entsprechende Key um ins BIOS zu wechseln sollte vorab ermittelt / recherchiert werden. Anschliessend sind sämtliche BIOS-Informationen zu dokumentieren. Besonders wichtig sind hier die im BIOS eingestellte Systemzeit und das Datum. Besteht eine Diskrepanz zur aktuellen Zeit oder zur notierten Zeit des Betriebssystemes ist dies entsprechend zu notieren und später bei der Analyse zu berücksichtigen.

%--------------------------------------------------
\subsection{Analysis (Analyse)}
Die Analyse Phase unterteilt sich zum einen in die Phase "`Preparation (Vorberietung)"' und zum anderen in die eigentliche Analyse-Phase. Der Übergang zwischen diesen beiden Phasen ist fliessend und lässt sich nicht immer klar trennen. Mit Abschluss der Secure-Phase kann der Ermittler sämtliche weiteren Arbeiten im Labor durchführen. Die Anwesenheit am Tatort ist nicht mehr zwingend, da sämtliches Material eingesammelt und gesichert wurde.

Die anzuwendenden Techniken und durchzuführenden Schritte bei der Analyse sind stark von der Situation abhängig. Bei der Untersuchung einer Cyber-Attacke sind oft andere Aspekte relevant, als bei der Untersuchung einer "`normalen"' Straftat. Im nachfolgenden Abschnitt werden einige Ansatzpunkte für eine Untersuchung aufgezeigt.

\subsubsection{Preparation (Vorbereitung)}
Bevor die Analyse der Daten beginnen kann, muss das forensiche Image gemountet werden. Dafür gibt es verschiedene Vorgehensweisen. In jedem Fall ist jedoch sicherzustellen, dass das Image ReadOnly gemountet wird. Es sollte vor und nach der Analyse jeweils ein Hash des Images erstellt, um zu verifizieren, dass das Image nicht manipuliert wurde.

Nachdem das Image gemountet wurde, wird empfohlen einige weitere Vorbereitungsarbeiten durchzuführen. Zuerst wird versucht so viele Daten wie möglich wiederherzustellen, beziehungsweise sichtbar zu machen, und diese für die anschliessende Analyse zu indexieren.

\begin{enumerate}
\item Wiederherstellung des File-Systems / der Beweisspuren (gelöschte, umbenannte, versteckte, verschlüsselte Dateien)
\item Generierung von Hashes für alle Dateien.
\item Abgleich der generierten Hashes. \\
Die generierten Hashes können mit Datenbanken abgeglichen werden, welche Hashes von zahlreichen System- / Programmdateien von verschiedenen Betriebssystemen und Programmen enthalten. Die so als irrelevant identifizierten Dateien können für die weitere Analyse ausgeblendet werden.
\item Suchindex über sämtliches Material erstellen \\
Es wird empfohlen ein Suchindex aus lesbaren Zeichen zu bilden. Bei der Index-Erstellung ist darauf zu achten, dass auch der File-Slack, alle belegten und unbelegten Bereiche auf dem Datenträger und die Metadaten des Dateisystems indexiert werden.
\item Kategorisierung der Dateien (zum Beispiel nach Typ)
\end{enumerate}



\subsubsection{Analysis (Analyse)}
Bei der Analyse werden die Rohdaten nach Informationen durchsucht, welche den aktuellen Fall unterstützten. Die gefundenen Informationen werden analysiert und in ein der folgenden drei Gruppen eingeteilt.

\begin{itemize}
\item Beweise untermauern eine bestimmte Theorie.
\item Beweise widerlegen eine bestimmte Theorie.
\item Beweise unterstützen keine bestimmte Theorie.
\end{itemize}

Die Schwierigkeit bei der Analyse der Informationen besteht darin, diese in einen kausalen und zeitlichen Zusammenhang zu setzen. Diese Zusammenhänge müssen zum einen plausibel und nachvollziehbar sein und zum anderen mit anderen Ereignissen korrelieren. Die Herstellung eines Zusammenhanges ist nicht immer einfach oder offensichtlich.  Zum Beispiel kann es notwendig sein, die Grammatik eines Verdächtigen zu analysieren, um eine Verbindung zwischen E-Mails, Dokumenten und Chat-Protokollen herzustellen.

Nachfolgend werden einige der gängigen Analysetechniken aufgelistet. Je nach System sind andere Analyse-Techniken erforderlich. Zum Beispiel kann unter Windows, beziehungsweise NTFS-Dateisystemen, eine Analyse des Alternate Datastreams oder der Windows Registry vorgenommen werden.

\begin{itemize}
\item Analyse des File Slacks
\item Timeline-Analyse
\item Analyse der Auslagerungsdateien
\item Analyse der versteckten Dateien
\item Analyse unbekannter Binärdateien
\item Analyse der Systemprotokolle
\item Analyse der Netzwerkschnittstellen
\item Analyse der Shell
\item Analyse der Druckerjobs und der Druckerqueue
\item Analyse der Dateien / Dateiendungen
\item Analyse von User Aktivitäten
\item Analyse der eingerichteten Jobs
\item Bei Verdacht auf Einsatz von Anti-Forensik-Techniken: Vertiefte Analyse
\item Weitere Untersuchungen auf Anwendungsebene (zum Beispiel: E-Mail, Browser)
\item ...
\end{itemize}

\paragraph{Unterstützende Fragestellungen}
Für die Analyse können folgende Fragestellungen hilfreich sein:
\begin{itemize}
\item War ein physischer Zugang zum System notwendig?\\
Ja: Kontrolle der physischen Überwachung und Zutrittskontrolle.
\item Hatten andere Personen Zugang zum System?\\
Ja: Kontrolle der physischen Überwachung und Zutrittskontrolle.
\item Was für Computerkenntnisse hat der Verdächtige? Was für Computerkenntnisse waren notwendig?
\item Was hat der Angreifer für Tools hinterlassen?
\item Wie wurden Tools und Befehle aufterufen? (Manuell / von Hand, via Copy \& Paste,, via Script)
\end{itemize}

%--------------------------------------------------
\subsection{Reporting (Dokumentation)}
Die gesamte forensische Untersuchung muss im Detail protokolliert und dokumentiert werden. Es müssen sämtliche Arbeitsschritte nachvollzogen und gegebenenfalls durch einen anderen Experten reproduziert werden können. Die eingesetzten Tools (inkl. Version) und Techniken sollten kurz beschrieben werden Im Rahmen einer Strafuntersuchung müssen die durchgeführten Schritte und angewandten Techniken so erläutert werden, dass diese von Laien verstanden und nachvollzogen werden können.

Die Dokumentation der Untersuchung sollte soweit als möglich und praktikabel sofort bei der Durchführung erstellt werden, da ansonsten wichtige Informationen, Gedanken und Arbeitsschritte verloren gehen. Die Dokumentation ist mit Screenshots oder gegebenenfalls Fotos, welche mit der Digitalkamera aufgenommen wurden, zu unterlegen.

\begin{itemize}
\item Verwendete Tools (inkl. Versionsnummer)
\item Verwendete Hardware (zum Beispiel FastBloc Write Blocker)
\item Angewendete Techniken
\item Prüfsummen von Dokumenten, Protokollen und Beweisen
\item Erläuterung der Evaluation der Tools und Techniken
\end{itemize}

Klar, einfach, auf Fakten basierend, schwieriger Aspekt
Keine Annahmen

Als Fakten nur: (Rekonstruierbare, Nachstellbare, oder mitGlaubwürdigkeit durch äussere Quelle)n
Nur was Beweise aussagen, keine Wertung / Meinung, wenn dann in der Diskusion, bzw. wenn gefragt, 
Exerten-Meinung basiert auf Fakten von den Beweisen, keine Annahmen

Unterschiedliche Arten von Berichte: Interne Berichte, Eidesstattliche Erklärungen, Erklärungen, Experten-Bericht
Immer: Screenshots, Illustrationen, Visualisationen


Interner Bericht: am häufigsten, nicht formal, trotzdem ernst, Review!, Klare Erklärungen, Details, 
Interner / Externer Anwalt: legal assumptions und risiken welche von Beweis resultieren ist sehr wichtig, evtl. weiterverwendung 
Aufbau: Executive Summary, Summary of Conclusion (Results Section) --> Was gemacht, was gefunden, weitere schritte (Beauftragt?), Alle Beweise aufisten: was, wieso relevant


Declaration: Nach initialem Report --> weitere juristische schritte --> Deklaration erstellen, Unterschied: Ansicht und Verstanden von Richter und Berater, alles sinnvoll / nachvollziehbar für leute ohne tech. hintergrund, vorbereiten für fragen / kritik / gegenargumente, bzw. aussagen auf basis report, worst case: Anzweiflung Beweis (Überprüfung durch Gegenseite), 

Aufbau: Statement, Background (Person, wieso qualifiziert), Informationen / Meinung / Schlussfolgerungen (Hauptteil), nur Beweise aus erster Hand, Conculsion

Eidesstattliche Erklärungen: Gleich Deklaration, Notarielle Beglaubigung

Experten-Bericht: Spitze formalität, Diseratation über Fall + Meinung, Berechtigt zur Meinungsbildung aus Basis der Beweise / Untersuchung, aber vorsichtig, keine spekulationen, auch Review anderer Experten-Berichte, Befragung for Gericht: Referenzierung Bericht wenn immer möglich, Referenzierung auf Untersuchungen, Studien, sTandards ok

Aufbau: Unterschiedlich je nach Experte, Cover Page, Overview, Qualifikationen, Was durchgeführt= (Items Reviewd), Analyse, Conclusion (optionian, state impact)

\subsection{Present (Präsentation)}
Bei der Präsentation werden die Ergebnisse der Untersuchung aufgezeigt und erläutert. Das Zielpublikum kann entweder zum Beispiel der Ermittlungsleiter, der Auftraggeber, das Management der Unternehmung oder ein Gericht sein. Hierbei ist zu beachten, dass auch Laien den Ausführungen folgen können und diese verstehen (analog der Dokumentation).

\subsection{Review (Rückblick)}
In der Review-Phase wird der Ablauf der vergangene Ermittlung betrachtet und analysiert, was gut gelaufen ist, beziehungsweise wo es noch Verbesserungspotenzial gibt. Ziel dieser Phase ist es, dass der eingesetzte Prozess für die forensische Analyse kontinuierlich verbessert und den neusten Erkenntnissen angepasst wird. Gegebenenfalls sind Massnahmen zu treffen, um langfristige Optimierungen und Verbesserungen zu erreichen. Mit dem Aufkommen von neuen Technologien und Trends (zum Beispiel Geräte aus dem Bereich Internet of Things) kann es notwendig sein, entsprechende Analyse-Werkzeuge zu erstellen oder zu evaluieren und im Labor erste Erfahrungen zu sammeln.

\section{Beweiskette und Beweissicherung}
Die korrekte Beweissicherung und die Einhaltung der Beweiskette sind essentiell für eine spätere Verwertung der Beweise bei einer Strafuntersuchung. Wird ein Beweisstück gefunden muss dieses korrekt sichergestellt werden. Jeder Beweis muss beschriftet werden und anschliessend ein Beweiszettel ausgefüllt werden. Ein Beispiel, beziehungsweise eine Vorlage, für einen Beweiszettel ist im Anhang im kapitel \ref{appx:Template:ProofPaper} \nameref{appx:Template:ProofPaper} zu finden. Dadurch wird sichergestellt, dass kein zweifel an der Herkunft und dem Besitztum des Beweises entsteht. Entweder im Beweiszettel oder separat wird ein Protokoll geführt, wer, wann, was mit dem Beweis gemacht hat. Dadurch kann der Weg des Beweises lückenlos zurückverfolgt werden.

Handelt es sich beim Beweis um ein elektronisches Gerät oder Bauteil, ist dieses in einem antistatischen Sack aufzubewahren. Je nach Situation wird bei Datenträgern das Original nicht mitgenommen, da dieses für den Weiterbetrieb benötigt wird. In diesen Fällen ist dies entsprechend zu dokumentieren und sicherzustellen, dass ein entsprechendes forensisches Duplikat erstellt wurde. Bei Datengrägern ist auf dem Beweiszettel zusätzlich der Hash des Datenträgers zu vermerken, sodass die Unversehrtheit jederzeit geprüft werden kann.


\section{Hinweise zur forensischen Analyse}
Bei einer forensischen Analyse sind folgende wichtige Aspekte zu berücksichtigen.

\begin{itemize}
\item \textbf{Zeuge / Zweitperson} \\
Während der Untersuchung sollte eine Zweitperson, bzw. ein Zeuge anwesend sein.
\item \textbf{Protokollierung} \\
Sämtliche durchgeführten Arbeitsschritte müssen protokolliert werden. Am Ende der Untersuchung sollte das Protokoll durch den Zeugen, die Zweitperson abgenommen und von beiden unterschrieben werden.
\item \textbf{Schutz der eigenen Umgebung} \\
Die eigene Analyseumgebung sollte gut gegen Angriffe geschützt sein und nicht direkt mit dem angegriffenen System oder Netzwerk verbunden werden. Sollte sich der Angreifer noch im Netzwerk oder auf dem System befinden, könnte er das Analysesystem angreifen und weiteren Schaden anrichten.
\item \textbf{Schutz der Beweismittel} \\
Sämtliche Beweismittel müssen sichergestellt und anschliessend geschützt werden. Eine Veränderung der Daten nach der Sicherung darf nicht mehr möglich sein beziehungsweise muss zweifelsfrei festgestellt werden können.
\item \textbf{Verwendung von Systembefehlen} \\
Zur Sammlung und Sicherung von Daten sollten niemals Systembefehle verwendet werden. Die Systemprogramme könnten vom Angreifer durch modifizierte Programme ausgetauscht werden sein. Der Ermittler sollte immer statisch vorkompilierte Programme verwenden.
\item \textbf{Einsatz Grafischer Programme} \\
Bei der Untersuchung eines Live-Systems sollte soweit als möglich auf den EInsatz von Programmen mit einer grafischen Oberfläche verzichtet werden. Grund dafür ist, dass diese eine Vielzahl an Binärdateien und Konfigurationen benötigen. Zum einen werden dadurch viele Zeitstempel geändert und zum anderen benötigen diese mehr RAM als Konsolenanwendungen.
\item \textbf{Patches und Updates} \\
Eher nicht, wenn kritisch, nach Rücksprache, Vernichtung Beweise
\item \textbf{Remote Untersuchung} \\
Bei einer forensischen Analyse ist nicht immer ein direkter Zugriff auf das System vorhanden. Gewisse Forensik-Tools erlauben den Einsatz via Netzwerk. Diese benötigt jedoch auf dem zu untersuchenden System einen entsprechenden Forensik-Client, mit welchem sich der Server anschliessend verbinden kann. Dieser sollte bereits vorgängig (vor dem Sicherheitsvorfall) auf dem System installiert worden sein.
\end{itemize}

% !TeX encoding=utf8
% !TeX spellcheck = de_CH_frami

\chapter{Einleitung}

\section{Hintergrund}
Im geschäftlichen Umfeld wird die Definition, die Modellierung und dadurch auch die Automatisierung von Business-Prozessen immer wichtiger. Mit der zunehmenden Beliebtheit von Home-Automation Produkten (Stichwort: \gls{acr:IOT}) gibt es auch im privaten Umfeld Abläufe welche übergreifend koordiniert und gesteuert werden können.

\section{Ziel}
Ziel dieser Arbeit ist es, ein praktisches Beispiel einer Implementation einer
End-to-End Prozess-Steuerung im Home-Automation Bereich zu realisieren.

\section{Aufgabenstellung} \label{sec:Aufgabenstellung}
Die freigegebene Aufgabenstellung lautet wie folgt:

Es soll aufgezeigt werden, was es heute für Möglichkeiten (Frameworks, Tools,
Produkte, ...) gibt um Abläufe / Prozesse im Bereich Home-Automation zu
modellieren und zu automatisieren. Mit Hilfe eines Raspberry PI's soll einer oder
mehrere typische Abläufe aus dem Home-Automation-Bereich realisiert werden.
Dabei soll auch geprüft / gezeigt werden, wie die Interaktion mit typischen
Home-Automation Protokollen / Stacks gestaltet werden kann.


\section{Erwartete Resultate} \label{sec:ErwarteteResultate}
Gemäss freigegebener Aufgabenstellung werden folgende Resultate erwartet:

\begin{itemize}
\item Dokumentation
\item Handout
\item Präsentation
\end{itemize}


\section{Abgrenzung} \label{sec:Abgrenzung}
Aufgrund des Umfanges der Arbeit und der begrenzten Zeitdauer werden folgende Punkte von der Arbeit abgegrenzt:

\todo{Abgrenzung}
\begin{itemize}
\itemBfText{Recherche}{Eingrenzung}
\itemBfText{Detailbetrachtung der Lösungen}{Aufwand, Detaillierungsgrad}
\end{itemize}


\todo{Abgrenzung: Geräte / Umgebung: Verfügbare Hardware: Raspberry Pi + Z-Wave}
\section{Struktur}
Diese Arbeit ist in folgende Teile gegliedert:

\todo{Fertig stellen}
\begin{itemize}
\item Einleitung
\item Ausgangslage
\item ...
\end{itemize}

Im ersten Kapitel werden die Details zur Aufgabenstellung und den Rahmenbedingungen dieser Arbeit aufgezeigt. Anschliessend werden in einem kurzen Kapitel die wichtigsten Aspekte der Ausgangslage aufgezeigt.

Der Kern der Arbeit besteht aus den drei Kapiteln zu den Business Prozessen in den Bereichen "`\gls{acr:IOT}"', "`Home Automation"' und "`Raspberry PI"'. Im ersten dieser drei Kapitel wird die Situation bezüglich Business Prozessen im Bereich \gls{acr:IOT} aufgezeigt und analysiert. Im Anschluss folgt eine Vertiefung der einzelnen Aspekte im Kontext "`Home Automation"' und "`Raspberry PI"'.

%TODO{Fertig stellen - Schlusswort}

\section{Planung} \label{sec:Intro:Planning}

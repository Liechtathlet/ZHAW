\chapter{Forensische Analyse}
\section{Ablauf}
Vorbereitung ("`Readiniess"'): Autorisierung, wichtig bei nicht polizeilichen Ermittlern, keine Aktionen auf eigene Faust, mehr Schaden als Nutzen
Schutz der Beweismittel: Keine Veränderung an Daten möglich, Schutz eigene Umgebung
Imaging und Datensammlung: Bitweise Kopie Datenträer, Sammlung Daten vom "`Lebenden"' System
Untersuchung und Bewertung Informationen: Analyse und Relevanzbewertung
Dokumentation: Alle Phasen, schlüssig sofort dokumentieren

Evaluation 
Collection
Analysis
Presentation
Review

S-A-P-Modell:
\begin{itemize}
\item Secure
\item Analyse
\item Present
\end{itemize}

Analyse:
\begin{itemize}
\item Einbruchsanalyse 
	-(Wer hatte Zugang?) Hinweise zu Täter --> Ermittlung Ausmass / Einschützung Schaden, Insiderwissen
	-Was hat der Angreifer auf Sys gemacht?, Bestimmt weiteres Vorgehen und Gegenmassnahmen, Daten einsicht / Modifikation / Zerstörung, Installation SW, neue User, Hintertüren?
	-Zeitpunkt Vorfall (Wichtig für Daten von anderen Quellen)
	-Weitere betroffene Systeme --> Recovery Planung, Zusatzinfos, mehr Spuren / Beweise
	-Wieso dieses System? Offene Schwachstellen? Besonder Daten?
	-Wie kam der Angreifer rein? Technik und Tools, Hinweise täter, 
	-Ist der täter noch aktiv? ist schon weg? kommt er wieder?
\item Schadensfeststellung
	- Bestimmung auf was für Daten / Informationen der Täter zugriff gehabt hätte
	- Evtl. noch aktive Passwortsniffer o.ä? Aufspüren, Passwörter ändern
\item Analyse der Tools
	- Was würde zurückgelassen? Spuren / Tools, Hinweise auf weitere / eigentliche Ziele?
	- Hinweise herkunft / Täter
	- Wie wurden Tools aufgerufen? Von Hand, via Copy & Paste (Zeilenweise), Scripts (Rasche eingabe)
	- Programmiersprache Tools, Einschränkung Täterkreis, z.T. im Programmcode Hinweise zum Täter (Kommentare, Copyright, ...., Sprache)
	- Querabgleich Binärdateien andere kompromitierte Systeme oder gefundene Dateien bei mutmasslichen Tätern
\item Logdatei-Analyse
	-Logs: netzwerkverbindungen, Firewall, Router, IDS - Wenn nicht: wieso?
	-Was verraten die Logs? Quelle, weitere Ziele ,Muster
	-Sicherung Logs Remote Access Systeme
	-Vermutung Innentäter: Sicherung Daten Zutrittskontrolle / Videoüberwachung
\item Weitere Beweissuche
	-Datenträger Analyse
	-Spuren von verwendeten Applikationen
	-Gelöschte Dateien
	-Versteckte Dateien? Dateimaskierung, versteckte Speicherorte
	-Verschlüsselte Datein vom Angreifer vorhanden? Evtl. Hinweise wenn schlecht gesichert
	-Versteckte Partitionen?
	-Bekannte versteckte Hintertüren / Fernzugriff (Rootkits, trojanisierte Systemprogramme)
\end{itemize}

\section{Techniken}
\section{Ablauf}


\section{Hinweise}
-Zeuge / Zweitperson bei Ermittlungen anwesend
-Protokollierung sämtliche Schritte (Abnahme durch Zeuge)
% !TeX spellcheck = de_DE_frami
\chapter{Ausgangslage}

Mit dem laufenden Fortschritt in der Computer- und Kommunikationstechnik und der damit einhergehenden Miniaturisierung und Mobilisierung eröffnen sich immer wieder neue Bereiche in der Informatik. Eines dieser neuen Gebiete wird als "`Internet of Things"' bezeichnet.

Mit dem Fortschritt in der Computertechnik und der fortlaufenden Miniaturisierung und Mobilisierung kommt das \gls{acr:IOT}

Allgemeins Gadgets, IOT



\section{Internet of Things}
Das Internet of Things (IOT), beziehungsweise das "`Internet der Dinge"', 

http://de.slideshare.net/RealTimeInnovations/io-34485340

http://www.internet-of-things.eu/


Die "`Things"', Dinge, beziehungsweise Geräte
-Embedded Devices / -Systems

\section{Domain Specific Software Engineering}


\section{Domain Specific Languages}
Eine "`Domain Specific Language (DSL)"' bezeichnet eine Programmier-, Modellierungs- oder Metasprache, welche für eine spezifische Domäne entworfen und entwickelt wurde. Eine DSL addressiert dabei spezifische Schwächen und Problemstellungen der angesprochenen Domäne.




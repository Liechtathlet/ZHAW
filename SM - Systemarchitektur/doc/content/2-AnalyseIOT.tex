% !TeX encoding=utf8
% !TeX spellcheck = de_CH_frami

\chapter{Business Prozesse im Bereich "`Internet of Things"'}

%---------------------------------------------------------------------------------
\section{Die Domäne "`Internet of Things"'}

Bereits fürher: RFID, Sensoren, Machine-Conrols, Messaging Integration, Analytics, Dsahboard -> Kosten und Grösser reduziert

Investitionen in IOT nur sinnvoll, wenn zwischen "`Edge"' uns "`Management"' Software, ist, welche Arbeit vereinfacht / Arbeit übernimmt

\subsection{Herausforderungen \& Problemstellungen}

Aktuell: Integration, Impl. Basic Service functions (Lernen von grossen Telekommunikationsfirmen), hoher Effizienzgrad benötigt

%---------------------------------------------------------------------------------
\section{Einfluss von \gls{acr:IOT} auf Business Prozesse}

Evtl. Fragestellung umgekehrt? 
Koordination von Things \& People in der cloud: Erfüllung von Kunden-Needs, Fundamten für IOT: Analyse + Folgeaktion auslösen

	Industrie IOT: nicht IOT, sondern Services bereitstellen, Consumer IOT: mehr auf Geräte fokussiert, anstatt auf Services, BPaas / High volume / on demand / embedded processes -> work with sensors / iot devices -> Eleiminierung repetitiver Prozesse oder Bereitstellung kritischer Prozesse

BPM: Ziel: Vereinfachung der Komplexität, konstant arbeiten, IOT -> Business komplexer gemacht -> BPM

BPM für Kühlschrank oder Waschmaschine: Kein grosser Benefit wenn nur für Home Use angewendet, interessant im Industrie-Bereich: Just-In-Time-Produktion für Ersatzteile (keine Lager)

BPM: Technology of Work, machine-friendly language of work, abstraktion, schnellere organisation der Arbeit, alternative? Software / code, spezifisch,

Meiste IOT-Projekte: Fokus auf Datensammlung, Analyse - Wenn in Betrieb: Porzesse aus effizienzgründen wichtig - z.B. predictive Mainentane - 10'000, triggers dass Komponente x in den nächsten Tagen ausfallen könnte...was wird mit dieser Datenflut angestellt? -> BPM

Auch negativ: BPM keine Relevanz in IOT: Komplexe Ereignisse und Pattern, konstante Veränderung und Interaktion, Keine Chance in IOT vordefinierte Prozesse einzuhalten, IOT: Komplexe Ereignisverarbeitung und Machine Learning, Kein process mining aufgrund grösser diversität der Daten und konstanten Veränderung, grösste Herausforderung: Privacy

BPM: Zukunft: Input IOT data, Prozesse ermöglichen erst die "`Datengewinnung"' aus den ermittelten Daten

IOT braucht BPMN nicht, umgekehrt, Beziehung nicht symmetrisch, BPM ist mehr spezialisiert, IOT ist "`breiter"'

Gemäss einem Bericht von Gartner (\cite{E:Gartner:BPM:2015}) sollten die Investitionen in \gls{acr:iBPMS} im Jahr 2015 um 4.4\% auf 2.7 Milliarden US-Dollar steigen. Im Rahmen der digitalen Transformation überdenken viele Unternehmen ihre Prozesse und Modelle. Einer der 4 genannten Einflussfaktoren ist \gls{acr:IOT}, wobei die "`Dinge"' in die Business Prozesse integriert werden. Dadurch kann sich der Prozess je nach Bedarf den veränderten Bedingungen anpassen. Durch die gemeinsame Orchestrierung mit allen anderen Prozessteilnehmen können Prozessinovationen einfacher umgesetzt werden.

Nach einem anderen Bericht von Gartner aus dem Jahr 2016 (\cite{E:Gartner:BPM:IOT:2020}) werden im Jahr 2020 mehr als die Hälfte aller neuen Business Prozesse und Systeme in irgendeiner Form ein Element von \gls{acr:IOT} beinhalten.

Stark Abhängig vom Needs der Unternehmen
Input - Entscheid - Reaktion
2 Sichten:
	- Unternehmen (z.B. Industrie) als Enabler, IOT als Kern, Unterstütztend
	- Private 
	
Evtl. aufteilen nach
	- Kern
	- Unterstützend
	- Privat
	

z.B. Predictive maintenance (Beispiel AKW), HealthCare
z.B. Formualr wird von Person ausgefüllt, kann das Formular selbst befüllt werden? (Health Care?) oder Ware wird in Truck verschoben, in Warehouse, Ankunft Kunde
	
Vorteile (Benefits) / Nachteile (Gefahren, etc...) in den Bereichen

Fragestellungen:
-Welche kritischen Prozesse zuerst?
-Wie viel Automatisierung notwenidg / erwünscht
-Wann BPM-Tools, wann Code?


Daten werden analysiert, Reaktionen auf Basis dieser Daten, z.b. Alerts oder korrigierende Schritte, Impact auf kritsche Business Przesse, erfordern integration in operative systeme, 

Effekte: Kosteneinsparungen, Effizienzsteigerungen, mehr revenue patterns, viele herausforderungen für gute umsetzung

Fokus BPMN früher auf: Automatisierung von menschhlichen Aufgaben, streamlining Workflows
IOT spielt eine wichtige Rolle zur Verbesserung der Prozesse in Unternehmen


Einsatzmöglichkeiten:
- Lagerhause: Temperatursensor, Threshold erreicht, massnahmen einleiten, wenn nicht erreicht -> benachrichtigung eines menschen

- Auto Versicherung
- Waste Management
- Smart Cities
- Smart Enviornemnts
- Smart Water
- Reatil (Supply Chain control, monitoring storage conditions)
-Industrial Control
-Reduktion Ineffizienten, Energie, Verbesserun glead times, increasing customer services
Beispiel für Prozesse im Bereich IOT
--
Let’s take an example of a fire breaking out at a public parking lot. Someone in the security team would call the building administration who would in turn inform various agencies including fire tenders etc. If someone wants to take stock of the overall status or deliver a coordinated response, it could be days before the information is recorded and reviewed. Look at an alternate reality, the fire sensors in the parking lot detect a fire and immediately cause a fire hazard case to be recorded. This causes immediate email and SMS messages to various agencies including fire tenders and police. The latest pictures and location can be captured by the parking security to make it easier for agencies and administration. The various agencies can provide the latest updates directly via various mobile handhelds. At any point of time, everyone has a clear picture of what’s going on.

A certain flight from London to Dubai gets delayed by 2 hours. Today it would mean 2 frustrating hours of waiting in the lounge or randomly browsing through shops. However IoT could radically change this experience. It could turn into an excellent opportunity for airport retailers and a pleasant couple of hours for passengers. This is how it could work - the information on flight delays is relayed in real time, the retailer then offers a discount sale with details being sent to the passenger mobiles. The Airport Wifi could accurately determine the current location of the passenger (assuming the passenger logs on to the airport app to access free WiFi) and then guide the passengers to the right shop. Based on the passenger profile, the retailer could then offer him a large discount on his favorite brand of cigars.
Today repair and maintenance is one of the most difficult and complex activity in large manufacturing organizations. Consider the advantages of predictive maintenance. When a critical shop floor machine is fitted with sensors, it can know its current condition and wear \& tear and, whenever necessary, initiate its own maintenance process. A combination of sensors and human operators monitor the environment continuously for hazards or damage resulting in reduced risks and maintenance costs.
--

\section{Mögliche Einsatzzwecke von automatisierten Prozessen im Bereich \gls{acr:IOT}}
% z.B. SBB: Stellwerkstörung, Spital



\section{Frameworks, Produkte, ...}
Nachfolgend werden einige Frameworks, beziehungsweise Produkte aufgezeigt, welche im Bereich \gls{acr:IOT} für die Realisierung und Automatisierung von Business Prozessen verwendet werden können.

\begin{itemize}
\itemBfText{IFML}{Plattformunabhängige Beschreibung von UI's, Schwerpunkt auf User-Interaktionen, Interaktions-Optionen, navigations-Pfade, User and System Events, Binding to Business Logic, Binding to Persistance layer, Integration mit BPMN	}
\itemBfText{WSO2}{Middleware, Open Source, Activiti BPM Plattform, Unterstützung: MQTT, OData, Plattform includes: Apache Synapse ESB, Apache Orchestration Director Engine, for operation in DC or cloud, Message Broker suport, trough activiti: BPMN 2.0 support. Details: http://www.infoq.com/news/2015/12/wso2-iot-process-orchestration}
\itemBfText{Pega}{Gartner 2015: Leader in Magic Quadrant for Intelligent Business Process Management Suites}
\itemBfText{Software AG - Digital Business Plattform}{http://www.softwareag.com/corporate/solutions/iot/default.asp}
\itemBfText{OpenIOT}{http://open-platforms.eu/library/openiot-the-open-source-internet-of-things/}
\itemBfText{Oracle Service Bus}
\end{itemize}


% Mögiche Fragestellungen welche in Bezug auf BPMN / IOT zuerst gelöst werden müssen / sollen
%http://enterprise-iot.org/book/enterprise-iot/part-ii-igniteiot-methodology/igniteiot-solution-delivery/building-blocks/iot-technology-profiles/4-middleware/process_efficiency_and_automation/
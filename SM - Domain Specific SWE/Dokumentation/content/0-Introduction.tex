% !TeX encoding=utf8
% !TeX spellcheck = de_CH_frami

\chapter{Einleitung}

\section{Hintergrund}
Im Rahmen meines Bachelor-Studiums in Informatik an der \gls{acr:ZHAW} muss im 5. Semester eine Seminararbeit zu einem vorgegebenen Themenbereich erarbeitet werden. Ich habe mich für den Themenbereich "`Angewandte Wirtschaftsinformatik"' entschieden.

Es soll eine wissenschaftliche Arbeit mit einem Bezug zur Wirtschaftsinformatik, beziehungsweise zu einem Themengebiet aus der Wirtschaftsinformatik, selbstständig erarbeitet werden. 

Für die Arbeit sollen circa 75 Arbeitsstunden aufgewendet werden. Dies entspricht etwa einem Umfang von 25 Seiten. Zusätzlich gelten die Rahmenbedingungen gemäss dem Reglement zur Verfassung einer Seminararbeit (\cite{ZHAW:2012:Seminararbeit:Reglemente})

\section{Aufgabenstellung}
Für den \gls{acr:TVT} soll der Einsatz eines Mini \gls{acr:ERP} Systems geprüft und eine entsprechende Lösung evaluiert werden.

\section{Abgrenzung}
\subsection{Mini ERP System}
Ein klassiches \gls{acr:ERP} System dient dazu interne Prozesse der Kernbereiche Finanz- und Rechnungswesen, Vertrieb \& Marketing, Personalwesen und Produktion, zentral in einem System abzubilden. Heutige \gls{acr:ERP} Systeme bieten neben den Kernbereichen noch zahlreiche weitere Funktionen, welche die klassischen Geschäftsprozesse unterstützen. \cite[S. 486]{Pearson:WirtschaftsInf:LaudonEtAl}

Aufgrund dieses grossen Funktionsspektrums werde ich mich auf ein Mini \gls{acr:ERP} System konzentrieren. Ein Mini \gls{acr:ERP} System ist gegenüber der klassischen \gls{acr:ERP} Lösung in der Bandbreite von Funktionalitäten eingeschränkt und auf die Bedürfnisse von kleinst- bis mittelgrossen Unternehmen, Vereinen, Verbände und \gls{acr:NPO} zugeschnitten. Zu den angebotenen Funktionen in einer Mini \gls{acr:ERP} Lösung gehören unter anderem die Stammdatenverwaltung der Mitglieder und das Finanz- und Rechnungswesen. Die weiteren Funktionalitäten sind wie auch beim klassischen \gls{acr:ERP} System stark von Einsatzzweck und Branche abhängig.

Ein \gls{acr:ERP} System, beziehungsweise ein Mini \gls{acr:ERP} System, kann ein sehr breites Spektrum an Funktionalitäten abdecken. Aus diesem Grund werden im Rahmen dieser Arbeit nur die im Kapitel \ref{sec:PotentialFunctionality} näher beschriebenen Themenbereiche berücksichtigt.

\subsection{Erhebung der Anforderungen}
Die Erhebung der Anforderungen, Randbedingungen und Ziele erfolgt mit Hilfe eines Fragebogens. Diese Erhebung stellt keine empirische Studie dar und dient in dieser Arbeit als Hilfsmittel um Inputs zur Definition der Anforderungen zu ermitteln.

\section{Motivation}
Für mich war von Anfang an klar, dass meine Seminararbeit einen realen Praxisbezug haben muss, damit mir die Arbeit auch Spass macht und ich motiviert arbeiten kann.

Als es an die Ideensammlung ging, bin ich rasch darauf gekommen eine Arbeit über, beziehungsweise für, den \gls{acr:TVT} zu schreiben. 

Durch mein Engagement in der Redaktion des Vereinsheftes (\gls{gls:TVTInfo}), der Mitwirkung bei der Verfassung und Gestaltung unserer Vereinschronik zum 150-Jahre Jubiläum und als Vereinsmitglied sind mir oft Sachen aufgefallen, welche mit dem Einsatz von modernen Anwendungen elegant und einfach gelöst werden könnten (Mehr dazu im Kapitel \ref{sec:Initial:DigitalCentury}).

Der \gls{acr:TVT} liegt mir sehr am Herzen und so setze ich mich immer gerne mit Themen im Zusammenhang mit dem \gls{acr:TVT} auseinander.

\section{Struktur}
Diese Arbeit gliedert sich in folgende Hauptteile:
\begin{itemize}
\item Ausgangslage
\item Analyse
\item Evaluation
\item Schlusswort
\end{itemize}

Im ersten Kapitel werden die Details zur Ausgangslage und die Hintergründe der Arbeit aufgezeigt. Im zweiten Kapitel wird mit Hilfe einer Umfrage innerhalb des Turnvereins eine Analyse erstellt. Aus dieser Analyse gehen die Randbedingungen, Ziele und Anforderungen an das Mini \gls{acr:ERP} System hervor. Diese Randbedingungen, Ziele und Anforderungen werden im Kapitel 'Evaluation' als Kriterien für die Vorselektion, Selektion und anschliessenden die Evaluation der Produkte verwendet. Im letzten Kapitel wird ein Fazit gezogen, eine Empfehlung an den \gls{acr:TVT} abgegeben und über die gesamte Arbeit reflektiert.


\chapter{Die Domäne: Internet of Things}
\section{Übersicht}

\section{Unterschied zu anderen Domänen}

\section{Probleme \& Schwierigkeiten}
%Energieeffizienz, Protokolle

\section{Einfluss auf Software Entwicklung}
\section{Was macht man anders? Was ist Wieso anders? Matrix Tabelle  gemäss Präsentation}

\section{Gegenüberstellung Stndard / Domäne}

\section{DSSE sinnvoll?}


\chapter{Schlusswort}


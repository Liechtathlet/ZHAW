\chapter{Tools und Techniken}

\section{Image-Related / Data-Aquisitation}
\subsection{Laufwerk löschen}
EnCase, statisches überschreiben (nur einmal), 

Attach to System, Windows, Start EnCase, Tools -> Wipe Drive, Select Device, Defaults as are, Next, HECF; S. 69
Kein WippingUtility, bei spezialfällen / wenn notwendig: Wipeutility: Linux: "`dd if=/dev/random of=/dev/<image drive>"'


\subsection{Image erstellen}
EnCase (Seite 72) DOS Boot disk
EnCase (Windows)
Linux: Device Name ermitteln: /proc/partitions oder logs, dann: "`dd if=/dev/<suspect drive> of=/some dir/image name"', "`md5sum /some dir/image name"', "`md5sum /dev/<suspect drive>"'
Midifzierte Version von dd, dcfldd für Forensik: http://dcfldd.sourceforge.net

FreeHelix

FTKImager (Windows)

\subsection{Image verifizieren}
EnCase, Tools, Verify Single Evidence File
Linux: md5sum "`image file"', compare


\section{Gelöschte Datenträger}
Datenträger-Löschsoftware, welche wurde eingesetzt? (noch installiert? typische Spuren?) arbeiten nicht immer zufälligen Löschpattern, evtl. Löschsoftware nicht ganz zuverlässig --> Temporäre Dateien, Registry, Protokolle, ...

\section{Kommerzielle Tools}
-SMART
-Helix (ein Teil Freeware)

Remote Untersuchungen:
-EnCase Enterprise Edition
-Paraben Enterprise
-ProDiscover
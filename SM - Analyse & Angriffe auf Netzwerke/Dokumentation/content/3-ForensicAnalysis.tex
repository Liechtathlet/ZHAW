\chapter{Forensische Analyse}
\section{Ablauf}
Vorbereitung ("`Readiniess"'): Autorisierung, wichtig bei nicht polizeilichen Ermittlern, keine Aktionen auf eigene Faust, mehr Schaden als Nutzen, Incident-Response-Plan
Schutz der Beweismittel: Keine Veränderung an Daten möglich, Schutz eigene Umgebung
Imaging und Datensammlung: Bitweise Kopie Datenträer, Sammlung Daten vom "`Lebenden"' System
Untersuchung und Bewertung Informationen: Analyse und Relevanzbewertung
Dokumentation: Alle Phasen, schlüssig sofort dokumentieren

Evaluation 
Collection
Analysis
Presentation
Review

S-A-P-Modell:
\begin{itemize}
\item Secure
\item Analyse
\item Present
\end{itemize}

--
Collection: Idenify, label, record, acquire data, preserve integrity, in timely manner
Examination: forensically processing data, 
Analysis: analyze result, using legally justifiable methods and techniques to derive useful information zur unterstützung des falles
Reporting: describing actions used, how tools / procedures were selected, , other actions, ....
---
Analyse:
\begin{itemize}
\item Einbruchsanalyse 
	-(Wer hatte Zugang?) Hinweise zu Täter --> Ermittlung Ausmass / Einschützung Schaden, Insiderwissen
	-Was hat der Angreifer auf Sys gemacht?, Bestimmt weiteres Vorgehen und Gegenmassnahmen, Daten einsicht / Modifikation / Zerstörung, Installation SW, neue User, Hintertüren?
	-Zeitpunkt Vorfall (Wichtig für Daten von anderen Quellen)
	-Weitere betroffene Systeme --> Recovery Planung, Zusatzinfos, mehr Spuren / Beweise
	-Wieso dieses System? Offene Schwachstellen? Besonder Daten?
	-Wie kam der Angreifer rein? Technik und Tools, Hinweise täter, 
	-Ist der täter noch aktiv? ist schon weg? kommt er wieder?
\item Schadensfeststellung
	- Bestimmung auf was für Daten / Informationen der Täter zugriff gehabt hätte
	- Evtl. noch aktive Passwortsniffer o.ä? Aufspüren, Passwörter ändern
\item Analyse der Tools
	- Was würde zurückgelassen? Spuren / Tools, Hinweise auf weitere / eigentliche Ziele?
	- Hinweise herkunft / Täter
	- Wie wurden Tools aufgerufen? Von Hand, via Copy \& Paste (Zeilenweise), Scripts (Rasche eingabe)
	- Programmiersprache Tools, Einschränkung Täterkreis, z.T. im Programmcode Hinweise zum Täter (Kommentare, Copyright, ...., Sprache)
	- Querabgleich Binärdateien andere kompromitierte Systeme oder gefundene Dateien bei mutmasslichen Tätern
\item Logdatei-Analyse
	-Logs: netzwerkverbindungen, Firewall, Router, IDS - Wenn nicht: wieso?
	-Was verraten die Logs? Quelle, weitere Ziele ,Muster
	-Sicherung Logs Remote Access Systeme
	-Vermutung Innentäter: Sicherung Daten Zutrittskontrolle / Videoüberwachung
\item Weitere Beweissuche
	-Datenträger Analyse
	-Spuren von verwendeten Applikationen
	-Gelöschte Dateien
	-Versteckte Dateien? Dateimaskierung, versteckte Speicherorte
	-Verschlüsselte Datein vom Angreifer vorhanden? Evtl. Hinweise wenn schlecht gesichert
	-Versteckte Partitionen?
	-Bekannte versteckte Hintertüren / Fernzugriff (Rootkits, trojanisierte Systemprogramme)
\end{itemize}

\section{Techniken}
\subsection{Aktuelle Prozesse}
-z.T. Programm / Prozess gestartet und nachher gelöscht, bei laufenden Prozessen gibt es Binärkopien in /proc, Bei Beendigung: Daten weg

\subsection{Shutdown}
Herunterfahren: Änderung vieler Zeitstempel, etl. Swap gelöscht
Steckerziehen, verkraften nicht alle Dateisysteme, Status laufende Umgebung auf Dateisystemm, Extraktion sehr schwierig, Entscheidung im Kontext
\section{Ablauf}


\section{Hinweise}
-Zeuge / Zweitperson bei Ermittlungen anwesend
-Protokollierung sämtliche Schritte (Abnahme durch Zeuge)

\subsection{Vorbereitung}
-Datenträger sjtjerilisieren, am besten formatierjt
-Dokumente / Formulare / Protokolle

\subsection{Secure}
Sicherung möglichst vieler Daten, ohne eigene Spuren zu hinterlassen
Zuerst Daten, die nach Shutdown oder hartem Shutdown nicht mehr verfügbar sind, nicht Systembefehle verwenden!!
Protokolldateien auf externtes Medium (eigenes), evtl. auch Scripte zum Zusammenzug der Informationen, 


-Beschreibung Physische Umgebung, detailliert, Fotos)
-Beschreibung Umgebung (Computer), detailliert Fotos
-Aktuelle Systemzeit + Referenzzeit, + Abweichung
-Liste der aktiven Prozesse
-Liste der geöffneten Sockets
-Liste der Anwendungen, die auf geöffnete Sockets lauschen
-Liste der User, die angemeldet sind
-Liste der Systeme, die gerade eine Netzverbindung haben oder vor kurzem hatten

Anschliessend, Suche nach:
-Timestamps gehacktes System
-Trojanisierte Systemprogramme
-Versteckte Dateien / Verzeichnissen
-Verdächtige Dateien / Sockets
-Verdächtige Prozesse

-Sicherung Cache- und Auslagerungsdateien
-Sicherung Hauptspeicher pro Prozess-ID
-Offene Netzwerkports
-Verbindungen im Auf-/Abbau
-erfolglose Verbindungsauf- / abbauversuche
-Prozesse: Umgebungsvariablen, Übergabeparameter, geladnee Bibliotheken, offene Dateideskriptoren, etc.
-Wechseldatenträger sicherstellen
-Hauptspeicher: Inhalt + Strukturdaten (Welcher Prozess welche Biblio, Belegung Speicherbereich), Strukturiert und komplett dumpen

-Forensische Duplikation: Bitweise 1:1 Kopie Speichermedium, mehrere Varianten: 1. Ausbau, Anschluss saubere Platte an System, Kopie via Netzwerk, Wichtig: Writeblocker (Verhindert physisch den Schreibzugriff auf das zu sichernde Medium), gewisse Bereiche auf Datenträgern wo Daten versteckt werden können: z.B. Host Protected Area, Device Configuration Overlay --> Verfahren auswählen, dass diese Daten auch sichert, Sicherung durch Checksummen, Hash-Algorithmen

Verwendung Duplikat: Grundsätzlich sinnvoll, bessere Analysemöglichkeiten (Notaufnahme vs. Gerichtsmedizin), Sinnvoll wenn: Strafverfolgung, Sicherstellung Beweise, ...

-Regeulärer Shutdwon? Ja / Nein
Mögliche Fehler Beweissammlung:
-Zeitstempel an Datein werden verändert (z.T. schon durch Ansehen der Datei, oder aufrufen von Systembefehlen)
-Keine Verwendung von grafischen Tools (Zugriff auf Vielzahl von Binärdateien / Konfigurationen --> Änderung Zeitstempel)
-Verdächtige Prozesse nicht beenden
-Alle Befehle protokollieren, sonst Lücke in Beweiskette
-Keine Verwendung vertrauensunwürdiger Programme / Systemtools (evtl. Sys-Dateien ausgetausch, eigene vorkompilierte Dateien)
-Patches / Updates: nur wenn Response-Team das empfiehlt, grundsätzlich offen lassen für Analyse, Abwägen je nach Kritikalität / Zusätzlichem Schaden bei offen lassen
-Software (de)-installieren auf Empfehlung Response-Team, Vernichtung Beweise, Installation Forensik-Tools: Abwägung, Vernichtung Spuren, 
-Protokolle nicht auf zu untersuchende Platte schreiben (Beweise gefährdet, FIle slack...)
-Ordnungsgemässer Shutdown könnte Beweise vernichten 


Sicherstellung:
-Haupteinheit (evtl. nur Datenträger)
-Montor + Tastatur: i.d.R. nicht, nur bei Spezialgeräten
-Dazugehörige Stromkabel
-Externe Festplatten, Disketten, DVD, CD, WORM, Backup-Bänder, USB, Speicherkarten
-Externe Kommunikationssysteme, welche für die Analyse benötigt werden: WLAN-Router, Modem, etc.
-Spezialhardware und -peripherie
-Digitalkameras, MP3-Player, etc.
-PDAs, Mobile-Telefone, ....

Jeweils Abwägung, ob Schaden durch Beeinträchtigung Arbeitsfähigkeit (Mitnahme Gerät / Teile) grösser wird, als das einfache Delikt

\subsection{Erste Schritte}
Einige Allgemeingültige Schritte, je nach System / Gerät etwas anders (Mobiletelefone, etc.)

\subsubsection{System ist ausgeschaltet}
-Fremde Personen vom System und Stromversorgung entfernen
-Umgebung fotografieren / skizzen anfertigen
-Aktive Druckjobs zu Ende laufen lassen
-System nicht einschalten (Achtung Notebooks, Deckel nicht hochklappen)
-Sicherstellung System ausgeschaltet (Bildschirmschoner)
-System evtl. im Standby-Modus? Bei Notebooks: Akku entfernen, sodass Energiesparmodus nicht anspringt (Veränderung Zeitstempel)
-Stromkabel entfernen
-Netzwerkkabel entfernen
-Geräte und Objekte Beschriften (Beweiszettel)
-Umgebung / Notizen / Unterlagen utnersuchen
-Befragung Anwender nach: Besonderheiten System, Passwörter, Konfigurationsspezifika -> Dokumentation & Hinterfragen
-Protokoll


\subsubsection{System ist eingeschaltet}
-Fremde Personen vom System und Stromversorgung entfernen
-Umgebung fotografieren / skizzen anfertigen
-Aktive Druckjobs zu Ende laufen lassen
-Befragung Anwender nach: Besonderheiten System, Passwörter, Konfigurationsspezifika -> Dokumentation & Hinterfragen
-Bildschirminhalte festhalten
-Keyboard / Maus wenn möglich nicht sofort berühren, Bildschirm blank / Bildschirmschoner: Ermittlungsleiter fragen, bezüglich "`aufwecken"', Zeitstempel festhalten)
-Live-Response (Siehe oben)
-Protokoll

\subsection{Analyse}
-Bewertung Beweisspuren: 3 Gruppen: 1: Untermauern bestimmte Theorie, wiederlegen bestimmte Theorie, unterstützen keine best. Theorie
Zuordnung der Informationen

Schwierigkeit: Kausaler und zeitlicher Zusammenhang herstellen, müssen mit anderen Ereignissen überinstimmen, plausiebel, nachvollziehbar

Fragen / Hilfe:
-War physischer Zugang zum System notwendig? Ja --> Zutrittskontrolle / Überwachung prüfen
-Wer hatte noch Zugang zum Computer? Zutrittskontrolle / Überwachung prüfen
-War Cronjob / Scheduler aktiv? --> Handlung ohne Anwesenheit Tatverdächtiger --> Prüfung Logs Betriebssystem / Scheduler
-Digitale Spur durch Fremdeinwirkung / Einwirkung dritter?
-Weitere Beweise Bekräftigung / Widerlegung digitale Spuren?
-Computerkenntnisse Tatverdächtiger? Was war für Know-How notwendig?
-Hardware transportabel? Verschleierung Standort durch Verwendung mehrere Computer?
-Code / Dokument / Mail / .... Verbindung zu Tatverdächtiger? (Stil, Grammatik, Vokabular, ...)
-Verbindung Spuren und besuchte Webseiten in Verbindung?
-Hinweise E-Mail-Clients / Chats zur Identifizierung Mittäter / Mitwisser?

\subsection{Dokumentation}
Wenn elektronisch: Verwendung vn Prüfsummen

Viele Screenshots zum festhalten oder mit Kamera

Untersuchungstools mit Versionsnummer

Unterschrift durch Zeuge: Gesamtes protokoll, und einzeln bei wichtigen Feststellungen / Beweisen

Pro Beweis: Beweiszettel, kein Zweifel an Herkunft, Besitztum, Unversertheit bestehen, Pro Objekt ein Zettel (Festplatte, PDA, Ausdruck, CD-Rom, Notebook)


Collection / Aquisition:
Power Down System and boot into secure environement oder Write Blockers + Anschluss an Analyse-Syste, Runterfahren: Stromquelle entfernen (Kabel ziehen)

1. Runter fahren
2. Sämtliche Laufwerke entfernen --> Dokumentieren und Beweiszettel ausfüllen --> Beweiskette etablieren, Innenleben dokumentieren (schriftlich, Fotos)
3. Sämtliche Datenträger sicherstellen, Beweiskette, wenn berechtigt: Arbeitsumgebung durchsuchen (!Corporate Policies berücksichtigen). 
4. Booten, BIOS-Informationen notieren auf Beweiszettel (Vorabrecherchieren Key für BIOS), Systemdatum -/ Zeit in Beweiszettel, WEnn Zeit anders: bei recover of files: Zeit anpassen
5. Forensisches Image erstellen
-a: Wipe Drive (Vorbereitungsarbeit, wenn möglich nicht vor Ort)
-b: Create Image with EnCase Bootable, Wenn EnCase Boot disk: Kein Write-Blocker notwendig, EnCase verhindert schreiben, Image benötigt in der Regel etwas mehr Platz als das original, immer Hash erstellen, Imaging auf -windows OS-Basis: Immer Write-Blocker, Linux: schreibt nicht auf Geräte, kein mount, Power Down Linux, Attach Drive, Power Up
6.	Hash erzeugen und speichern, am besten auf Beweiszttel
7.	Bag and Tag: Laufwerke auf die, die Images geschrieben wurden beschriften, sicher unterbringen, (anti-statischer Sack, sicherer Ort), Original-Laufwerk: je nach Szenario: auch eintüten oder wieder in Betrieb

Bemerkung zur Remote Untersuchung: nicht immer physischer Zugang möglich, weit entfernt, Einsatz von Forensik-tools via Netzwerk, Voraussetzung: Vorinstallierter Client auf System, voraus testen
% !TeX encoding=utf8
% !TeX spellcheck = de_CH_frami

%
% add files for appendix chapter here

%\input{content/Z-App-Survey.tex} \label{appendix:Survey}

%\input{content/Z-App-SurveyResults.tex} \label{appendix:SurveyResults}

\chapter{Vorlage: Formular Incident-Meldung} \label{appx:Template:IncidentMessage}
-Melder

-Richtige Informationen abfragen (Merkblatt / Chekliste)
--Basisinfos: Aktuelle Uhrzeit, Wer / Welches System berichtet Vorfall, Art und Weise Vorfall, Vermuteter Zeitpunkt Vorfall, mittelbar / unmittelbar betroffene HW / SW, evtl. Auswirkungen, Schaden, Kontakstelle für ISR und Ermittler
--Infos über betroffenes System sammeln (!! möglichst nicht vom System abfragen, Datenklassifizierung? Klassifizierung? Ort?, Physischer Zugang?  allgemeiner Systemzustand,)
--Angreifer: Infos? noch aktiv? Systeme / Daten manipuliert / zerstört, Vermutungen?
--Getroffene Massnahmen / System verändert? Andere Perosnen benachrichtigt?

\chapter{Vorlage Formular Ermittlung}
-Fallnummer
-Datum / Zeit
-Wieso forensische untersuchung (Verdacht?)
-Grund
-Ermittlungsleiter
-Ermittlungsteam
-Betroffene Systeme / Geräte / Anwendungen 8(Seriennummern und interne Bezeichnung)
-Verantwortliche Administratoren
-Protokolle, Incident Meldung, Beweiszettel

Täterprofil
-Was waren / sind mögliche Ziele
-Was ist der Grund für den Angriff / Einbruch?
-(Interne) komplizen?
-Tools / Techniken?
-Spuren?
....
\chapter{Vorlage: Protokoll}


Tabelle mit : Laufnummer, Zeit, Befehl / Aktion, Hash Ergebnisdatei, Kommentar


\chapter{Vorlage: Beweiszettel}
Buch CF: Seite 85

Beweiskette:

Laufwerke: Manufacturer, Model, Serial Number, Evidence Description (Name of suspect, Technologie: SATA, IDE, ...)





\chapter{Ablauf einer forensischen Analyse}

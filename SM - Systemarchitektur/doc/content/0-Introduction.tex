% !TeX encoding=utf8
% !TeX spellcheck = de_CH_frami

\chapter{Einleitung}

\section{Hintergrund}


\section{Ziel}


\section{Aufgabenstellung} \label{sec:Aufgabenstellung}
Die freigegebene Aufgabenstellung lautet wie folgt:

\begin{itemize}
\item 
\end{itemize}


\section{Erwartete Resultate} \label{sec:ErwarteteResultate}
Gemäss freigegebener Aufgabenstellung werden folgende Resultate erwartet:

\begin{itemize}
\item 
\end{itemize}


\section{Abgrenzung} \label{sec:Abgrenzung}
Aufgrund des Umfanges der Arbeit und der begrenzten Zeitdauer werden folgende Punkte von der Arbeit abgegrenzt:

\begin{itemize}
\itemBfText{Schnittstellendokumentation}{In dieser Arbeit werden nicht die Schnittstellendokumentationen und -spezifikationen rekonstruiert. Es werden jeweils die relevanten Aspekte betrachtet und hervorgehoben.}

\end{itemize}


\section{Motivation}

\section{Struktur}
Diese Arbeit ist in folgende Teile gegliedert:

\todo{Fertig stellen}
\begin{itemize}
\item Einleitung
\item Ausgangslage
\item ...
\end{itemize}

Im ersten Kapitel werden die Details zur Aufgabenstellung und den Rahmenbedingungen dieser Arbeit aufgezeigt. Anschliessend werden in einem kurzen Kapitel die wichtigsten Aspekte der Ausgangslage aufgezeigt.

Der Kern der Arbeit besteht aus den drei Kapiteln zu den Business Prozessen in den Bereichen "`\gls{acr:IOT}"', "`Home Automation"' und "`Raspberry PI"'. Im ersten dieser drei Kapitel wird die Situation bezüglich Business Prozessen im Bereich \gls{acr:IOT} aufgezeigt und analysiert. Im Anschluss folgt eine Vertiefung der einzelnen Aspekte im Kontext "`Home Automation"' und "`Raspberry PI"'.

%TODO{Fertig stellen - Schlusswort}

\section{Planung} \label{sec:Intro:Planning}

% !TeX encoding=utf8
% !TeX spellcheck = de_CH_frami

\chapter{Einleitung}

\section{Hintergrund}
Im Rahmen meines Bachelor-Studiums in Informatik an der \gls{acr:ZHAW} muss im 6. Semester eine Seminararbeit zu einem vorgegebenen Themenbereich erarbeitet werden. Ich habe mich für den Themenbereich "`Domain Specific Software Engineering"' entschieden.

Aus einem Themenkatalog konnte ein spezifisches Thema im Bereich "`Domain Specific Software Engineering"' ausgewählt werden. Ich habe mich für das Thema "`Internet of Things"' entschieden.

Für die Arbeit sollen circa 50 Arbeitsstunden aufgewendet werden. Dies entspricht etwa einem Umfang von 15 bis 20 Seiten. Zusätzlich gelten die Rahmenbedingungen gemäss dem Reglement zur Verfassung einer Seminararbeit (\cite{ZHAW:2012:Seminararbeit:Reglemente})

\section{Aufgabenstellung}
Es soll ein Dokument zum Thema Domain Specific Software Engineering im Bereich Internet of Things erstellt werden. Das Papier soll die Schwierigkeiten der Software-Entwicklung in diesem Bereich aufzeigen und einen groben Überblick über das Thema eben.

\section{Abgrenzung}
\todo{Abgrenzung}

\section{Motivation}
In den letzten Monaten bin ich immer wieder auf das Thema "`Internet of Things"' aufmerksam geworden. Seit meiner Teilnahme an der Microsoft Konferenz in Barcelona Ende 2014 (TechEd Europe) war meine Neugier definitiv geweckt und ich hatte mir bereits einige Projekte überlegt, welche ich allenfalls als Semesterarbeit, Seminararbeit oder in einem anderen Schulprojekt realisieren könnte. Mit dem Seminar "`Domain Specific Software Engineering"' hat sich mir nun eine erste Gelegenheit geboten, um mich in dieses Thema zu vertiefen. An dem Thema fasziniert mich vor allem das Breite Spektrum an Innovations- und Kombinationsmöglichkeiten.


\section{Struktur}
Diese Arbeit gliedert sich in folgende Hauptteile:

\begin{itemize}
\item Ausgangslage
\item Die Domäne "`Internet of Things"'
\item Software Engineering in der Domäne "`Internet of Tings"'
\item Case Study "`iotivity"'
\end{itemize}

\todo{ Struktur abschliessend verifizieren}
Im ersten Kapitel werden die Details zur Ausgangslage und die Hintergründe der Arbeit aufgezeigt. Im zweiten Kapitel wird die Ausgangslage in Bezug auf die betrachteten Themen "`Internet of Things"', "`Domain Specific Software Engineering"' und "`Domain Specific Languages"' kurz erläutert. im darauffolgenden Kapitel wird die Domäne "`Internet of Things"' im Detail beschrieben und aufgezeigt, was für Anforderungen an das Software Engineering bestehen. Im Kapitel 4 werden nachher die in der Domäne eingesetzten Standards, Protokolle und Frameworks beschrieben. Im Kapitel 5 wird anschliessend eine kleine Fallstudio, beziehungsweise ein Praxistest mit dem Framework "'iotivity"' durchgeführt. Im letzten Kapitel wird ein Fazit gezogen, eine Empfehlung an den  abgegeben und über die gesamte Arbeit reflektiert.
\chapter{Software Engineering in der Domäne "`Internet of Things"'}


http://postscapes.com/internet-of-things-software-guide
https://www.silabs.com/Support%20Documents/TechnicalDocs/bringing-the-internet-of-things-to-life.pdf
http://www.datamation.com/open-source/35-open-source-tools-for-the-internet-of-things-1.html
https://blog.profitbricks.com/top-49-tools-internet-of-things/
http://www.businesscloudnews.com/2015/05/14/samsung-announces-open-internet-of-things-platform/
https://www.rti.com/company/news/iot-connectivity-webinar.html
http://www.iotsworldcongress.com/documents/4643185/4c3cc80c-03b0-41be-baf0-6a1a0fa7db07
https://www.mapr.com/blog/key-requirements-iot-data-platform#.VVuNi3W1FBc


SWE:
https://hal.inria.fr/hal-01064075/document
http://ieeexplore.ieee.org/xpl/articleDetails.jsp?arnumber=7030178
http://www.cio.com/article/2843814/developer/how-to-develop-applications-for-the-internet-of-things.html
http://www.appdevelopersalliance.org/internet-of-things/
http://www.fit.fraunhofer.de/de/fb/ucc/lehre/scrum-based_softwaredevelopmentofinternet-of-thingsapplications.html
http://insights.wired.com/profiles/blogs/winning-strategies-software-development-for-the-internet-of#axzz3anWNtzAE
http://www.intel.com/content/www/us/en/internet-of-things/white-papers/developing-solutions-for-iot.html
http://link.springer.com/chapter/10.1007%2F978-3-642-36818-9_6#page-1

\section{Programmiersprachen}

\subsection{Assembler}
Eine Implementation in Assembler kann unter bestimmten Voraussetzungen die beste Lösung für ein bestimmtes Problem sein. In der Regel wird der Assembler-Ansatz gewählt, wenn das Programm so effizient und sparsam wie möglich ablaufen und mit so wenig Ressourcen als möglich auskommen soll.



\subsection{C / C++}


\subsection{Java / .NET}
Für die Implementation kann es durchaus sinnvoll sein eine Hochsprache, wie Java oder .NET C\# einzusetzen. 

\section{Protokolle und Standards}

\subsection{Thread}
Protokolle: Thread, Netzwerk-Protokoll, Fokus: Security, Low Energy, notwendiger Chip, schon in vielen Geräten vorhanden, gestützt auf 6LoWPAN, IPv6 over Low power Wireless Personal Area Network, 


\subsection{AllJoyn}
Qualcomm entwickelt, anschliessend: Linux Foundation, AllSeen Alliance (Cisco, Microsoft, LG, HTC, ...)
Verbindung, Wartung Geräte in WLAN-Netzwerk, Kontrolle, Benachrichtungs Service, 

MQTT: Collect device data
CoAP
XBEE
XMPP: Access device data
DDS: Distribute Device Data
HTTP
AMQP (IOT-Client-Seitig?)

https://www.sparkfun.com/news/1705
http://micrium.com/iot/internet-protocols/
http://electronicdesign.com/embedded/understanding-protocols-behind-internet-things

\section{Frameworks}
In diesem Kapitel werden einige Frameworks vorgestellt, welche ....

apache ISIS?
http://iot.eclipse.org/java/open-iot-stack-for-java.html

\subsection{Kommunikation zwischen \gls{acr:IOT}-Geräten}
AllSeen
IOTIVITY
Z-Wave
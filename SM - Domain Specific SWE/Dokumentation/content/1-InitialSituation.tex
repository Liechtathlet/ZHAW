% !TeX spellcheck = de_DE_frami
\chapter{Ausgangslage}

Mit dem laufenden Fortschritt in der Computer- und Kommunikationstechnik und der damit einhergehenden Miniaturisierung und Mobilisierung entwickeln sich laufend neue Themenbereiche und Sparten in der Informatik. Das "`Internet der Dinge"' ("`Internet of Things"') ist eines dieser Gebiete und hat in den letzten Jahren und Monaten für viele Diskussionen und viel Ideen und Innovationsmöglichkeiten gesorgt.

\section{Internet of Things}
Nach dem Aufkommen der Personal Computer, der Etablierung des Internets und Trend im Bereich Smartphones, Tablets und Wearables der letzten Jahre, stellt das \gls{acr:IOT} oder "`Internet der Dinge"' die nächste Etappe in der Entwicklung des Internets und der \gls{acr:M2M} Kommunikation dar. Diese Etappe wird die Art und Weise wie Leute und vor allem Dinge miteinander kommunizieren und interagieren grundlegend verändern.

Dem \gls{acr:IOT}-Markt wird grosses Entwicklungspotenzial zugeschrieben, ein grösseres sogar als vor einigen Jahren dem Smartphone und Tablet Markt. Gab es im Jahr 2012 weltweit bereits 20 Milliarden "`Connectable Things"' (Dinge die mit dem Internet verbunden sind, beziehungsweise verbunden werden könnten), werden es im Jahr 2020 bereits 32 Milliarden sein. \cite[S. 5]{E:EMC:DigitalUniverseOfOpportunities}

Mit der Etablierung des \gls{acr:IOT} wird eine Transformation von isolierten Geräten hin zu vernetzten Geräten statt finden. In Zukunft wird vermehrt die Kommunikation von Geräten untereinander und mit dem Internet im Vordergrund stehen. Im Privatbereich wird sich vieles, wenn nicht sogar alles, um die Heimautomation drehen. Wird von der Wetterstation ein Sturm gemeldet, werden automatisch die Sonnenstoren eingefahren und die Fenster und Rollläden geschlossen. Wenn man nach Hause kommt wird automatisch beim Aufschliessen der Türe das Licht der Radio eingeschaltet. Die Heimautomation kann auch als Einbruchsschutz während längeren Abwesenheiten dienen. Durch den Einsatz von Machine-Learning-Algorithmen lernt das System die Gewohnheiten der Bewohner und kann dieses bei längerer Abwesenheit simulieren. Kommt es dennoch zu einem Einbruch oder einem Einbruchsversuch kann das System selbstständig reagieren und zum Beispiel die Fenster und Türen schliessen, die Polizei und die Bewohner alarmieren und allenfalls Videoaufnahmen und Fotos anfertigen.

In der Industrie wird das \gls{acr:IOT} zum Beispiel zur Überwachung, Kontrolle und Steuerung von Produktionsanlagen oder Fertigungsprozessen eingesetzt. 

Durch das \gls{acr:IOT} ergeben sich viele Möglichkeiten und Chancen. Es ist jedoch nicht ausser Acht zu lassen, dass auch neue Risiken und Gefahren entstehen. So müssen die entstehenden Netzwerke gegen Angriffe und Manipulationen abgesichert und die transportierten Daten vor fremden Augen geschützt werden. Kann ein Heimautomationssystem oder ein medizinisches Überwachungssystem in einem Spital von aussen manipuliert werden, kann dies schwerwiegende Folgen haben. 

Im Kapitel \ref{chap:DomainIot} \nameref{chap:DomainIot} wird die Domäne des \gls{acr:IOT} näher beschrieben.

\section{Domain Specific Software Engineering}
Domain Specific Software Engineering, beziehungsweise das domänenspezifische entwickeln von Software, beschreibt den Entwicklungsprozess in einem bestimmten Kontext oder Problembereich. Der Entwicklungsprozess für Software von medizinischen Geräten unterscheidet sich an vielen Stellen vom Entwicklungsprozess einer Unternehmung, welche Web-Seiten designt und entwickelt. Auch unterscheidet sich die Entwicklung von Web-Seiten von der Entwicklung von Mobile-Apps oder Fat-Client-Anwendungen. Jede Domäne hat ihre eigenen Anforderungen und Besonderheiten. Diesem Umstand wird durch den Einsatz von \gls{acr:DSL}'s und spezifischen Entwicklungswerkzeugen und -Methodiken Rechnung getragen.

\section{Domain Specific Languages}
Eine \gls{acr:DSL} bezeichnet eine Programmier-, Modellierungs- oder Metasprache, welche für eine spezifische Problemdomäne entworfen und entwickelt wurde. Eine \gls{acr:DSL} addressiert dabei spezifische Schwächen und Problemstellungen der angesprochenen Domäne.

Eine gute \gls{acr:DSL} zeichnet sich durch ihre Einfachheit aus. Die Sprache sollte sich so nah als möglich am Problembereich befinden und auch die entsprechenden Ausdrücke und Begriffe verwenden. Dadurch wird die Eintrittshürde zur Verwendung der \gls{acr:DSL} heruntergesetzt und der Einarbeitungsaufwand reduziert. Die Semantik und Syntax sollte so gewählt werden, das diese im Kontext Sinn ergibt und einfach lesbar und verständlich ist. \gls{acr:DSL}'s werden oft auch von nicht Programmierern verwendet, da sich die Sprache sehr nahe an der fachlichen Sprache der Domäne befindet.

Es können zwei Arten von \gls{acr:DSL}'s unterschieden werden. Eine externe, oder auch unabhängige, \gls{acr:DSL} ist so ausgelegt, dass diese nicht von einer bestimmten Sprache abhängig ist. Die Festlegung der Syntax und Grammatik liegt komplett in der Verantwortung und Entscheidungsfreiheit des Autors. Für die Implementation kann eine beliebige Programmiersprache verwendet werden.

Im Gegensatz dazu basiert die interne oder eingebettete \gls{acr:DSL} auf einer spezifischen Sprache (Wirtssprache). Diese Sprache gibt dabei die Einschränkungen für die zu implementierende \gls{acr:DSL} vor. Im Gegenzug muss sich der Autor nicht mehr um die Grammatik, Parser und Tools kümmern, da dies bereits von der Sprache zur Verfügung gestellt wird.

Externe \gls{acr:DSL}'s sind flexibler, erfordern aber einen viel höheren Implementierungsaufwands als eine interne \gls{acr:DSL}.

\textbf{Beispiele}
\begin{itemize}
\item Externe \gls{acr:DSL}
\begin{itemize}
\item SQL
\item Reguläre Ausdrücke
\item CSS
\item Sass
\end{itemize}
\item Interne \gls{acr:DSL}
\begin{itemize}
\item Rake (Ruby)
\end{itemize}
\end{itemize}




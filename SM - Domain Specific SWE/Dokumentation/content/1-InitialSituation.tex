% !TeX spellcheck = de_DE_frami
\chapter{Ausgangslage}

Mit dem laufenden Fortschritt in der Computer- und Kommunikationstechnik und der damit einhergehenden Miniaturisierung und Mobilisierung eröffnen sich immer wieder neue Bereiche in der Informatik. Eines dieser neuen Gebiete wird als "`Internet of Things"' bezeichnet.


\section{Internet of Things}
Nach dem Aufkommen der Personal Computer, der Etablierung des Internets und dem Mobile-Trend der letzten Jahre, stellt das \gls{acr:IOT} oder "`Internet der Dinge"' die nächste Etappe in der Entwicklung des Internets dar. Diese Etappe wird wiederum die Art und Weise wie Leute und vor allem Dinge miteinander kommunizieren und interagieren grundlegend verändern.

Der \gls{acr:IOT}-Markt wird grosses Entwicklungspotenzial zugeschrieben, ein grösses sogar als vor einigen Jahren dem Smartphone Markt. \todo{Prognose / Zahlen --> Aus PDF von LIEBH}

Mit der Etablierung des \gls{acr:IOT} wird eine Transformation von isolierten Geräten hin zu vernetzten Geräten statt finden. In Zukunft wird vermehrt die Kommunikation von Geräten untereinander und mit dem Internet im Vordergrund stehen. Im Privatbereich wird sich vieles, wenn nicht sogar alles, um die Heimautomation drehen. Wird von der Wetterstation ein Sturm gemeldet, können automatisch die Sonnenstoren eingefahren, die Fenster geschlossen und die Fensterläden heruntergelassen werden. Ein anders Beispiel wäre, dass wenn man nach Hause kommt automatisch beim Aufschliessen der Türe das Licht angeht und sich der Radio einschaltet. Die Heimautomation kann auch als Einbruchsschutz während längeren Abwesenheiten genutzt wurden. Durch den Einsatz von Machine-Learning kann das System die Gewohnheiten der Bewohner erlernen und das Verhalten bei Abwesenheit zu simulieren. Kommt es dennoch zu einem Einbruch oder einem Einbruchsversuch kann das System selbstständig reagieren und zum Beispiel die Fenster und Türen schliessen, die Fensterläden herunterlassen, die Polizei und die Bewohner alarmieren und allenfalls Videos und Fotos anfertigen.

Im Industriebereich kann das \gls{acr:IOT} zum Beispiel zur Überwachung, Kontrolle und Steuerung von Produktionsanalgen oder Fertigungsprozessen eingesetzt werden. 

Durch das \gls{acr:IOT} ergeben sich viele Möglichkeiten und Chancen. Es ist jedoch nicht ausser acht zu lassen, dass auch neue Risiken und Gefahren entstehen. So müssen die entstehenden Netzwerke gegen Angriffe und Manipulationan abgesichert und die transportierten Daten vor fremden Augen geschützt werden. Kann ein Heimautomationssystem oder ein Überwachungssystem von aussen manipuliert werden, kann dies schwerwiegende folgen haben. Unter Umständen kann der Angreifer die Wohnungstüre öffnen und anschliessend so umprogrammieren, dass sie der Bewohner nicht mehr öffnen kann.

\section{Domain Specific Software Engineering}
Domain Specific Software Engineering, beziehungsweise das domänenspezifische entwickeln von Software, beschreibt den Entwicklungsprozess in einem bestimmten Kontext, beziehungsweise Problembereich. Der Entwicklungsprozess für medizinische Geräten und Software unterscheidet sich an vielen Stellen vom Entwicklungsprozess einer Unternehmung, welche Web-Seiten designt. Auch unterscheidet sich die Entwicklung von Web-Seiten von der Entwicklung von Mobile-Apps oder Fat-Client-Anwendungen.

\section{Domain Specific Languages}
Eine \gls{acr:DSL} bezeichnet eine Programmier-, Modellierungs- oder Metasprache, welche für eine spezifische Problemdomäne entworfen und entwickelt wurde. Eine \gls{acr:DSL} addressiert dabei spezifische Schwächen und Problemstellungen der angesprochenen Domäne.

Eine gute \gls{acr:DSL} zeichnet sich durch ihre Einfachheit aus. Die Sprache sollte sich so nah als möglich am Problembereich befinden und auch die entsprechenden Ausdrücke und Begriffe verwenden. Dadurch wird die Eintrittshürde zur Verwendung der \gls{acr:DSL} heruntergesetzt und der Einarbeitungsaufwand reduziert. Auch sollte die Semantik und Syntax so gewählt werden, das diese im Kontext Sinn ergibt und einfach lesbar und verständlich ist.

Es können zwei Arten von \gls{acr:DSL}'s unterschieden werden. Eine externe, oder auch unabhängige, \gls{acr:DSL} ist so ausgelegt, dass diese nicht von einer bestimmten Sprache abhängig ist. Die Festlegung der Syntax und Grammatik liegt komplett in der Verantwortung und Entscheidungsfreiheit des Autors. Für die Umsetzung kann eine beliebige Programmiersprache verwendet werden.

Im Gegensatz dazu basiert die interne oder eingebettete \gls{acr:DSL} auf einer spezifischen Sprache (Wirtssprache). Diese Sprache gibt dabei die Einschränkungen für die zu implementierende \gls{acr:DSL} vor. Im Gegenzug muss sich der Autor nicht mehr um die Grammatik, Parser und Tools kümmern, da dies bereits von der Sprache zur Verfügung gestellt wird.

Externe \gls{acr:DSL}'s sind flexibler, erfordern aber einen viel höheren Implementierungsaufwands als eine interne \gls{acr:DSL}

\textbf{Beispiele}
\begin{itemize}
\item Externe \gls{acr:DSL}
\begin{itemize}
\item SQL
\item Reguläre Ausdrücke
\item CSS
\item Sass
\end{itemize}
\item Interne \gls{acr:DSL}
\begin{itemize}
\item Rake (Ruby)
\end{itemize}
\end{itemize}




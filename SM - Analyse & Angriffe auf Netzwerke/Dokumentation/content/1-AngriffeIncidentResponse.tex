\chapter{Angriffe, Incident Detection \& Incident Response} \label{chap:AIDIR}
In diesem Kapitel wird erläutert, wie typische Angriffe ablaufen, wie diese erkannt und anschliessend entsprechend reagiert werden kann.


\section{Angriffe}
Für die Reaktion auf Angriffe und die Untersuchung von Angriffen ist es wichtig die grundlegenden Angriffsverfahren, -methoden und -techniken zu verstehen. In diesem Kapitel werden die wichtigsten Punkte erläutert.


\subsection{Angriffstypen}
Grundsätzlich können zwei Angriffstypen unterschieden werden. Auf der einen Seite stehen Massenangriffe, so genannte "`un-targeted attacks"', deren Ziel es ist so viele Geräte oder Services als möglich zu treffen. Das einzelne Opfer spielt dabei eine untergeordnete Rolle. Phising und Malware sind zwei Beispiele für solche Massenangriffe. Ausgenutzt wird hier grundsätzlich immer die Offenheit des Internets.

Auf der anderen Seite stehen gezielte Angriffe, so genannte "`targeted attacks"'. Diese Attacken sind in der Regel auf das Ziel oder das spezifische Szenario, massgeschneidert. Solche Angriffe werden über mehrere Monate hinweg geplant und vorbereitet. Oft sind diese Codes spezifisch entwickelt worden und können somit von Intrusion-Detection-Systemen und Anti-Viren-Software nicht oder nur sehr schwer erkannt werden. Ein Beispielt für eine solche Attacke wäre Spear-Phising.

Bei den gezielten Angriffen hat sich in den letzten Jahren eine neue Unterkategorie, die Kategorie der "`advanced persistent threats"'. Ziel dieser Angriffe ist es, möglichst lange unerkannt zu bleiben und den Einbruch zu vertuschen. Dabei werden gerade so viele Daten gesammelt, bzw. Aktionen durchgeführt, dass der Täter noch unerkannt bleibt. Ein solcher Angriff wird über mehrere Monate, wenn nicht sogar Jahre, hinweg vorbereitet und anschliessend Schritt für Schritt umgesetzt. Auch der eingesetzte Schadcode wird so gebaut, dass dieser möglichst lange unterkannt bliebt, aber trotzdem so viel Nutzen als möglich erbringen kann.


\subsection{Kategorien von Schwachstellen}
Bei einem Angriff werden immer vorhandene Schwachstellen ausgenutzt. Diese Schwachstellen können in drei Kategorien unterteilt werden.

\begin{itemize}
\item \textbf{Flaws (Fehler / Mängel)} \\
Bei einem Flaw handelt es sich um eine unbeabsichtigte Funktionalität der Anwendung. Dieser kann entweder durch schlechtes Design oder simpel und einfach durch einen Implementierungsfehler entstehen.
\item \textbf{Features (Funktionalitäten)} \\
Hier wird eine vorhandene Funktionalität für andere Zwecke missbraucht. Dabei handelt es sich um keinen Fehler in der Anwendungen, sondern um eine Funktionalität, welche entsprechend spezifiziert wurde.
\item \textbf{User Errors (Benutzer Fehler)} \\
User Errors werden durch den Benutzer verursacht. Zum Beispiel könnte ein unerfahrener Systemadministrator unwissentlich Schwachstellen im System freischalten.
\end{itemize}


\subsection{Komplexität}
Durch die vorherrschende Monokultur von Betriebsystemen, Anwendungen und Komponenten werden die Anforderungen an Hacker immer grösser. Mit den steigenden Anforderungen werden auch die Angriffe und die Angriffstechniken immer komplexer.

\todo{Bild CF Seite 13}


\subsection{Täter}
Die Motivation von Tätern sind sehr unterschiedlich. Dies reicht von Sozielen, politischen, finanziellen, staatlich-politischen Motivationen über technische Ambitionen bis hin zu Regierungen oder Gruppierungen wie Anonymous. Neben der Motivation können die Täter nach Aussen- und Innentätern unterschieden werden. Innentäter verfügen über Insider-Wissen und arbeiten in der Regel für das angegriffene Unternehmen oder die angegriffene Organisation. Der Anteil an Innentätern am gesamten Tätervolumen ist sehr hoch und wächst stetig. Unternehmen und Organisationen sind sich dessen aber nicht immer bewusst und wähnen sich in falscher Sicherheit.

Die "`Berufsbezeichnungen"' der Täter sind sehr unterschiedlich und vielfältig. Nachfolgend sind einige der gängigsten Bezeichnungen und deren Bedeutung aufgelistet.

\todo{Vervollständigen}
\begin{itemize}
\item \textbf{Elite} \\
\item \textbf{Hacker} \\
Neutraler Begriff
\item \textbf{Cracker} \\
Negativ
\item \textbf{Script Kiddy} \\
\item \textbf{White-Hat} \\
Berücksichtigung Hackerethik, Penetrationstests
\item \textbf{Gray-Hats} \\
\item \textbf{Black-Hats} \\
...
\end{itemize}


\subsection{Typischer Ablauf}
Ein Angriff kann in die nachfolgenden Phasen gegliedert werden. Diese können je nach Angriff in unterschiedlichen Ausprägungen vorkommen.

\subsubsection{Survey (Untersuchung)}
In dieser Phase werden so viele Informationen wie möglich gesammelt. Dazu gehören Informationen über die Organisation, die eingesetzte Hard- und Software und Prozesse. Anschliessend wird versucht so viele Schwachstellen wie möglich zu ermitteln. Zum einen wird ein Footprinting durchgeführt, welches so viele Informationen wie möglich über die Systeme zu Tage befördern soll. Zum Footprinting gehören unter anderem Port- und Protokollscanns und DNS- und WHOIS-Abfragen. Zum anderen werden mit Hilfe von Social Engeineering und Commodity-Toolkits und -Techniken weitere Schwachstellen ermittelt.


\subsubsection{Delivery (Positionierung)}
Diese Phase beschäftigt sich mit den expliziten Vorbereitungen für die Ausnutzung der Schwachstellen. Der Angreifer versucht das für dieses Szenario am besten geeignete Vorgehen zu ermitteln und bringt sich anschliessend in Position um die Schwachstellen auszunutzen. Eine typische Aktion in dieser Phase wäre zum Beispiel der Versand einer infizierten E-Mail oder das Unterjubeln eines inifizierten USB-Sticks.


\subsubsection{Breach (Ausnutzung)}
In dieser Phase wird die Schwachstelle ausgenutzt, um dem Angreifer Zugang zum gewünschten System zu verschaffen.

\subsubsection{Affect (Beeinträchtigung / Infizierung)}
Nach dem der Angreifer Zugang zum System erlangt hat, unternimmt er weitere Schritte um sein eigentliches Ziel zu erreichen. Dies kann zum Beispiel die Erweiterung seiner Zugriffsrechte, die Einrichtung von Hintertüren, die Sammlung von Daten oder der Angriff eines weiteren Systemes sein.


\subsubsection{Clean Up (Aufräumen)}
Je nach Ziel und Zweck des Angreifers verwischt er seine Spuren und räumt auf, damit er unerkannt bleibt oder allenfalls zu einem späteren Zeitpunkt nochmals zurückkehren kann.

\section{Incident Detection (Erkennung eines Vorfalls)}
Bevor auf einen Angriff, beziehungsweise auf einen Sicherheitsvorfall, reagiert werden kann muss dieser zuerst bemerkt werden. Bleibt der Vorfall unterkannt, wird es nie zu einer Untersuchung kommen.
Ein Angriff kann durch verschiedenste Indikatoren erkannt und zum Teil sogar vorausgesagt werden. Nachfolgend werden einige dieser Indikatoren aufgelistet.

\subsection{Hinweise Netzwerkseitig}
\begin{itemize}
  \item Ungewöhnlich hohe Netzwerklast
  \item Ungewähnliche Anzahl Firewall-Regelverstösse
\end{itemize}

\subsection{Hinweise Serverseitig}
\begin{itemize}
  \item Unbekannte Prozesse
  \item Unbekannte / Neue User
  \item Unbekannte Dateien
  \item Ungewöhnliche Systemlast
  \item Dienste laufen nicht mehr
  \item Ungewöhnliche Systemanmeldungen
  \item Systemabsturz
  \item Kleiner werdende Log-Files
  \item Bestehende Dateien werden grösser (Beispiel: Ausführbare Datei wächst um mehrere kB)
  \item Versuch Berechtigungen zu verändern
  \item Schlechte Performance
\end{itemize}

\subsection{Hinweise durch Intrusion-Detection-Systeme}
Intrusion-Detection-Systeme sind dazu da Angriffe möglichst früh zu erkennen und die entsprechenden Stellen zu informieren. Ist das Intrusion-Detection-System gut konfiguriert, kann dieses Angriffe anhand von Strategien und Mustern erkennen.

\todo{Data Loss Prevention Technology}


\subsection{Weitere Hinweise}
Weitere Hinweise können durch Kunden, Partner, Mitarbeiter, Strafverfolgungsbehörden oder die Presse erfolgen.
\todo{Intrustion-Mapping-Systeme}

\subsection{Meldung eines Vorfalles}

-Richtige Informationen abfragen (Merkblatt / Chekliste)
--Basisinfos: Aktuelle Uhrzeit, Wer / Welches System berichtet Vorfall, Art und Weise Vorfall, Vermuteter Zeitpunkt Vorfall, mittelbar / unmittelbar betroffene HW / SW, evtl. Auswirkungen, Schaden, Kontakstelle für ISR und Ermittler
--Infos über betroffenes System sammeln (!! möglichst nicht vom System abfragen, Datenklassifizierung? Klassifizierung? Ort?, Physischer Zugang?  allgemeiner Systemzustand,)
--Angreifer: Infos? noch aktiv? Systeme / Daten manipuliert / zerstört, Vermutungen?
--Getroffene Massnahmen / System verändert? Andere Perosnen benachrichtigt?

Beurteilung Vorfall / Störung: Kentnisse aktueller Status, Organisation, Landschaft, MA, nicht zu lange warten, Durchleuchtug / Ausschlussverfahren

Bei Einbruch: erste Risikoabschützung für möglcihe Abschaltzung / Netzdekonnektion, Berücksichtigung weiterer Ermittlungsschritte

Entscheid Abschaltung / Dekonnektion: Management der Systemeigentümer, Basis: Empfehlung ISR

Klassifizierung  des Vorfalls: Probing  / Portscanning, Denial-of--service-Angriff, Unberechtigter Zugriff auf User-Account, ... Admin-Account, Datendiebstahl / -manipulation,

...

Wichtig ist, dass nach einer Incident Detection das Incident Response Team unverzüglich Informiert wird. Ist kein Incident Response Team vorhanden und gibt es in der Unternehmung keine entpsrechende Anlaufstelle, ist das Vorgehen mit....
Keine übereilten Reaktionen, da der Angreifer allenfalls etwas bemerkt



\section{Incident Response Team}
Rasch und richtig handeln
Eingreiftruppe bei Incidents, Häufig durch Situation bedingt, wer wegen Detailkenntnisse in Gruppe gehört,
Augenmerk: Erfahrene Personen für Schlüsselpositionen, Integrität und Zuverlässigkeit MA, passendes Persönlichkeitsprofil (gesunder Menschenverstand,Fähigkeit effiziente und annehmbare Entscheide zu treffen in krit Sit., gute Kommunikationsfähigkeiten, an Regeln / Prozeduren halten, Arbeiten unter Stress, Teamfähig, Vorbild Sicherheitsrelevante Tätigkeiten, Priorisierung utner Stress)

Bei grösseren Organisationen / IT-Abhängige: Evtl. Dauerhaft in Belegschaft, Wahrnehmung anderer Sicherhietsaufgaben, Früherkennung, Personen: Leiter, Kontaktperson bei Verdacht, etc., Spezialist: Erfassung / Behanldung Vorfall, Spezialist: Schwachstellen, Spezialist auf Plattform, Schulungspersonal

Wichtig: Koordinator, direkter Zugang zum Mgmt

-Ermittler, IT-Professionals (Technical support staff, system, network, security admin): small number of forensic tools according to their area of expertise, 
-Incident-Handlers (respond to variaty of computer security incidents, wide variety of forensiq technique and tools, knowledge of forensic principles, guidelines, procedures, ...)

-Legal advisors, hr, auditors, physical security staff

\section{Incident Response}
Nach dem der Angriff entdeckt wurde oder Anzeichen für einen zukünftigen Angriff bestehen müssen entsprechende Massnahmen in die Wege geleitet werden. Die einzuleitenden Massnahmen sind dabei von der gewählten Reaktionsart abhängig. Es sind folgende zwei Reaktionen denkbar

Teil der Computer Forensik

Bei Einbruch: in kurzer Zeit: Schaden, Angriffsmethoden und mögliche weitere Auswirkungen für Org. beurteilt werden
Notwendig: guter Beweissicherungsmassnahmen im Prozess etablieren

Guter Incident Response Prozess / erfolgreicher Ablauf Incident Response: Basis für juristische Verfolgung

Wichtig: Ermittlung Ursaxche, grundlage für zukünftige Handlungsempfehlungen

organisatorische Vorarbeit notwendig, um korrekt reagieren zu können. wenn nicht: im Entscheidenden Moment keine Ressourcen
-Incident Awareness: Beteiligte MA, Bewusstsein
-Grobes Konzept Sicherheitsvorfallbehandlung (Eskalations- / Alarmierungsregelung, Weisungskompetenzen)
-Security-Monitoring- und Alarmierungskonzept (Einbezug: Personalvertreter, Datenschutzbeauftragter für Datenauswertung)
-Weiterbildungen: Incident-Detection / Response
-Kontakt zu Security-Spezialisten / Ermittlungsbehörden aufbauen
-....

Strategie, zwei Aspekte berücksichtigen: direkten / indirekten Schaden minimieren, Tathergang möglichst umfassend rekonstruieren zur Identifikation Tatverdächtige, jeder Sicherheitsvorfall erfordert andere Strategie
-Kritkialität System im Bezug auf Unternehmensprozesse
-Kritikalität / Wichtigkeit gestohlene Daten
-Täger-Vermutung?
-Vermutung Fähigkeiten / Wissen beim Täter
-Vorfall an Öffentlichkeit gelangt?
-Wie weit ist der Täter gekommen
-Verkraftbare Downtime?
-Vermuteter finanzieller Gesamtverlust

-Kein unüberlegter Gegenangriff, Angriffer bemerkt, evtl. zerstörung,
-Honeypots

Infos: Business Impact, ...
Vorfall analysieren, Schlüsse ziehen, Lerneffekt, Ermittlungsvorgang analysieren (Optimierung / Verbesserung), Reaktionszeit, Wirksamkeit, Tätermotivation, Kosten

\begin{itemize}
  \item Härtung der Systeme, Abwehr des Angriffes
  \item Abwarten, Beobachten, Informationen sammeln.
\end{itemize}


\subsection{Reaktionsarten}
Bei der Auswahl von geeigneten Massnahmen ist immer auch der Zeitpunkt der Angriffes zu beachten.

\begin{itemize}
  \item Angriff in der Zukunft
  \item Angriff ist am Laufen
  \item Angriff ist schon vorbei
\end{itemize}


\subsubsection{Härtung der Systeme, Abwehr des Angriffes}

\subsubsection{Abwarten, Beobachten, Informationen sammeln}

\begin{itemize}
  \item Eigentlicher Angriff noch ausstehend, Härtung und Abwehr
  \item Eigentlicher Angriff noch ausstehend, Abwarten, Beobachten, Informationen sammeln, Backtracing
  \item Angriff am Laufen, Abwehr (Härtung)
  \item Angriff am Laufen, Abwarten, Beobachten, Informationen
\end{itemize}


Evtl. übergreifendes Kapitel
\section{Ablauf}
\begin{itemize}
  \item Systemeinbruch oder normale Betriebsstörung?))
  \item Wahrnehmung / Bemerkung ungewöhnliche Aktivitäten (Administrator, Anwender, ...)
  \item Evtl. weitere Beobachtung
  \item Kurze Analyse / Sammlung von Spuren
  \item Bestätigung Verdacht
  \item Meldung an Incident Response Team
  \item Sicherstellung elektronische Beweise
  \item Beweisspuren identifizieren
  \item Beweisspuren analysieren
  \item Analyseergebnisse interpretieren / verifizieren
  \item Analyseergebnisse in Bericht zusammenfassen / präsentieren.
\end{itemize}

-Identify
-Assess (if is security incident), check, gather information
-Respond: Kick of procedure
--Initial Response: Determine origin, idenitfy compromised systems, evtl. disconnect from nw, untersuchung starten, Anzeige: Sicherung beweise
--Recovery
Report
Review


Organisatorische Vorbereitung: Rollen, Verantowrtlichkeiten, Policies, Vorbereitende / Unterstützende Massnahmen im Rahmen System Life Cycle (Zentralisierte Logs, Auditing für Server, Arbeitsplatzcomputer, ..., file hashes für verbreitete Betriebssysteme und Installationen, File integrity checking software, data retention policies, etc. ), Guidelines, Step-By-Step- Procedures
% !TeX encoding=utf8
% !TeX spellcheck = de_CH_frami

\chapter{Schlusswort} \label{chap:Finish}

%---------------------------------------------------------
\section{Fazit}

\subsection{Internet of Things}
\subsection{Home Automation}
Im Bereich "`Home Automation"' gibt es gemäss heutigem Stand nur ...

Die meisten Softwarelösungen und Kombi-Produkte (Software + Hardware) basieren auf dem Konzept von Auslösern (Triggern) und nachfolgenden Ereignissen (Events).

Heimanwender: BPM Schwierig, wenn dann im Bereich der Gebäude-Automatisierung im Bereich von Unternehmen.


OPENHAB:
Version 1: Schwieriger / Komplizierter Setup (insbesondere mit Z-Wave)

Version 2: Einfacherer Setup, viele fehlende Funktionalitäten, nicht intutitiv, Dokumentation für gewisse Grundkonzepte nicht wirklich gut

\subsection{Home Automation + Raspberry Pi}
Evtl. Getrennte Fazite für einzelne Bereiche?


\subsection{Allgemein}
Aufgrund der Ergebnisse der Analyse kann gesagt werden, dass aktuell im Bereich der Modellierung und Implementierung von Business Prozessen ein Wandel statt findet. Mit diesem Wandel rückt der Einsatz des \gls{acr:IOT} etwas mehr in den Vordergrund. 

....Bereits erste Produkte / Lösungen mit entsprechendem Support.



%---------------------------------------------------------




%---------------------------------------------------------

\section{Dank}

\chapter{Tools und Techniken} \label{chap:ToolsTechniques}
In diesem Kapitel werden die grundlegenden Tools und Techniken einer forensischen Analyse vorgestellt. Die Tools und Techniken werden dabei nach der Phase, in der diese eingesetzt und angewendet werden, gegliedert.



%---------------------------------------------------------------------------------------------------------------
\section{Readiness}

\subsection{Datenträger löschen}
Digitale Beweise sollten immer auf leere, beziehungsweise komplett gelöschte, Datenträger gesichert werden.

\textbf{Linux (native)} \\
\begin{verbatim}
dd if=/dev/random of=/dev/<DriveToWipe>
\end{verbatim}

\textbf{Tool-Suiten}\\
\begin{itemize}
\item EnCase
\end{itemize}




%---------------------------------------------------------------------------------------------------------------
\section{Secure}

\subsection{Auslesen der Zeitkonfiguration}
Die Zeitkonfiguration des Systemes wird je nach Betriebssystem und zum Teil sogar je nach Distribution an einem anderen Ort gespeichert.

Unter Ubuntu, beziehungsweise Debian-Systemen ist diese unter \textit{/etc/timezone} zu finden. Bei Red-Hat-Distributionen unter \textit{/etc/sysconfig/clock}.
Beim Auslesen der Zeitkonfiguration ist auch zu notieren, ob das System automatisch zwischen Sommer- und Winterzeit umstellt.

\subsection{Sicherung flüchtiger Daten}

\todo{CF: Seite 210 / 2011}
\todo{CFH: S. 1 - 35}

\subsection{Forensische Duplikation}
Varianten: Ausbau, Anschluss saubere Platte an System, Kopie via netzwerk
Writeblocker

Versteckte Bereiche auf Datenträgern: Host Protected Area, Device Configuration Overlay --> Verfahren, dass auch diese Daten sichert, Checksummen

\subsection{Verifizierung eines forensischen Duplikates oder eines Beweisstückes}
EnCase, Tools, Verify Single Evidence File
Linux: md5sum "`image file"', compare

\subsection{Harter vs. Normaler Shutdown}
Bei einem normalen Shutdown wird das System durch den Benutzer regulär heruntergefahren, sämtliche Daten werden gespeichert und das System befindet sich im Anschluss in einem sauberen / lauffähigen Zustand. Während dem Shutdown werden jedoch die Zeitstempel von zahlreichen Dateien verändert und temporäre Dateien des Betriebssystems gelöscht. Dies kann unter Umständen die Analysearbeiten erschweren oder im schlimmsten Fall wichtige Beweise vernichten. 

Bei einem harten Shutdown wird das System von der Stromversorgung getrennt, ohne das dieses vorher heruntergefahren wurde. Mit diesem Vorgehen wird sichergestellt, dass keine Zeitstempel von Dateien verändert werden. Auch ist die Wahrscheinlichkeit da, dass auf der Festplatte eine Auslagerungsdatei vorhanden ist, welche anschliessend analysiert werden kann. Die Extraktion und anschliessende Analyse dieser Daten ist jedoch sehr aufwändig. Diese Methode des Shutdowns kann bei gewissen Dateisystemen zu irreparablen Schäden führen. Daher ist vorgängig abzuwägen, ob ein harter Shutdown sinnvoll und verkraftbar ist.

Bei einem Shutdown gehen in der Regel immer viele Daten verloren. Es wäre zum Beispiel möglich, dass der Angreifer ein Schadprogramm installiert hat, welches nur noch im RAM verfügbar ist. Nach einem weichen Shutdown ist das Programm auf dem System nicht mehr auffindbar. 

Bei beiden Varianten ist die Zeit der Durchführung und die Art des Shutdowns zu protokollieren.

\subsection{Sicherung des RAM-Inhaltes}
Nachdem herunterfahren des Systems sind die Daten im RAM noch einige Sekunden verfügbar. Dies reicht in der Regel jedoch nicht um eine Datensicherung durchzuführen. Einige neuere Studien und Experimente haben gezeigt, dass es durchaus Mittel und Wege gibt, um den Inhalt des RAMS zu sichern. Eine Möglichkeit besteht darin, die Raumbausteine mit einem Stickstoffspray auf -50 Grad Celsius herunter gekühlt. Anschliessend wird der Rechner ausgeschaltet, der RAM ausgebaut und in ein anderes System eingebaut. Das System wird mit einer Spezialsoftware gestartet, welches einen Memory-Dump erstellt.

Kernel-Level Sicherung (Nicht User-Mode)
Small-Footprint
Firewire-Attacke
\subsection{Writeblocker}
\todo{Writeblocker}
HW und SW

\subsubsection{Tool}
EnCase (Seite 72) DOS Boot disk
EnCase (Windows)
Linux: Device Name ermitteln: /proc/partitions oder logs, dann: "`dd if=/dev/<suspect drive> of=/some dir/image name"', "`md5sum /some dir/image name"', "`md5sum /dev/<suspect drive>"'
Midifzierte Version von dd, dcfldd für Forensik: http://dcfldd.sourceforge.net

FreeHelix

FTKImager (Windows)




%---------------------------------------------------------------------------------------------------------------
\section{Analysis}

\subsection{Gelöschte Datenträger}
Datenträger-Löschsoftware, welche wurde eingesetzt? (noch installiert? typische Spuren?) arbeiten nicht immer zufälligen Löschpattern, evtl. Löschsoftware nicht ganz zuverlässig --> Temporäre Dateien, Registry, Protokolle, ...

\subsection{Analyse des File Slacks}
Aufgrund der Besonderheiten einiger Dateisysteme können auf Datenträgern mit einem solchen Dateisystem Dateien versteckt werden. Der Grund dafür liegt, darin das eine Datei auf dem Datenträger in sogenannten Dateiblöcken mit einer festen Länge gespeichert werden. Füllt eine Datei nicht den gesamten Dateiblock aus entsteht ein ungenutzter Bereich. Dieser Bereich wird File Slack genannt und wird vom Betriebssystem mit zufälligen Daten aufgefüllt. Unter Windows werden für das Auffüllen des Slacks zum Beispiel Inhalte aus dem Arbeitsspeicher verwendet. Dies wird dann als RAM-Slack bezeichnet.

\todo{Dateisysteme mit FileSlack}

Die File Slack Analyse ermöglicht dem Ermittler unter Umständen auf Inhalte zuzugreifen, welcher der Täter gelöscht hat.

\subsection{Timeline-Analyse}
Bei der Timeline-Analyse werden die letzten Aktivitäten, welche auf dem System durchgeführt wurden, in ein zeitlich logische Abfolge gebracht. Bei der Herstellung einer Verbindung zu anderen Beweisen ist eine allfällige Abweichung der Systemzeit von der Referenzzeit zu beachten. Die Timeline-Analyse ist häufig der erste Ansatzpunkt einer Ermittlung. Durch die Analyse kann festgestellt werden, was undder Täter auf dem System gemacht / verändert / installiert hat. Ausgehend von dem entstehenden Zeitstrahl können weitere, vertiefte Analysen durchgeführt werden. Ein wichtiger Baustein der Timeline-Analyse ist die Auswertung der MAC-Time der Dateien und Ordner. Die MAC-Time setzt sich aus folgenden Elementen zusammen:

\begin{itemize}
\item Modification Time\\
Zeitpunkt der letzten Modifikation (Schreiben), Der Zeitstempel ändert sich bei folgenden Aktionen nicht: Kopieren, Verschieben, Umbenennen, Veränderung Dateiattribute
\item Access-Time\\
Zeitpunkt des letzten Zugriffes (Lesen / Ausführen), Der Zeitstempel wird auch verändert, wenn Metadaten oder Dateiinhalte angezeigt werden
\item Creation-Time (Windows)\\
Zeitpunkt der Erstellung der Datei, Der Zeitstempel wird bei Erstellung einer Kopie aktualisiert. Beim Verschieben einer Datei / Ordner wird der Zeitstempel nicht aktualisiert.
\item Change-Time (Unix)\\
Zeitpunkt der Veränderung bestimmter Metadaten der Datei
\end{itemize}

Nicht jede Aktion löst auf der Datei selbst eine Veränderung der MAC-Time aus. Unter Linux bewirkt die Verschiebung einer Datei eine Veränderung der MAC-Time des Verzeichnises, aber nicht der Datei selbst. Die MAC-Time kann jedoch relativ einfach durch den Angreifer, beziehungsweise durch Anti-Forensik-Tools manipuliert oder unbrauchbar gemacht werden. Eine weitere Schwierigkeit besteht darin, dass die MAC-Time sich je nach Betriebssystem anders verhält.


\subsection{Analyse von Auslagerungsdateien}
Fast alle OS: Erweiterung physisch nutzbarer Speicherbereich, oder Auslagerung kurzfristig nicht benötiger Speicher
Auslagerungs- oder Swap-Dateien, Datei oder virtuelles / physisches Dateisystem
Win: Swap / Page-File: nicht gelöscht bei Shutdown, 

Sonderfälle: Inhalt Hauptspeicher vor Eintritt in Ruheszustand / Suspend-To-Disk-Modus, nach Wake-Up wieder gleich, Hilfreich: RAM-Analyse


\subsection{Analyse von versteckten Dateien}
Klassische Festplatte:
Section Gaps: Sektoren überall gleich --> am Rand Gaps
Paritions Gaps: Mehrere Partitionen, Gaps --> spuren früherer Partitionen, Daten vestecken, auch in unpartitionierten Bereich

Bad Blocks: Vom OS als unbrauchbar markiert, missbrauch möglich
Hidden-Attribut: Möglichkeit auf OS Ebene Dateien zu verstecken


Rootkits: Können gut Dateien, Verzeichnis, Prozesse, Netzwerkverbindungen verbergen --> trojanisierte Systemprogramme
Kernel-Level-Rootkits: Prozesse, Verbindungen, Dateien verstecken + Hintertüren bereitstellen, Austausch von System-Calls, Aufspüren schwierig: Analyse von Strukturen, Beobachtungen, Vergleich Syscall-tAbelle, z.T. auch Syscall-Code direkt verändert

Bei modernen: Reduktion durch Mapping-Technologien

\subsection{Dateien oder Fragemente wiederherstellen}
File Carving: Zusammensetzen von Dateifragementen, Extraktion wesentlicher Informationen, Wiederherstellung aus File-Slack oder unalloziertem Bereich, 
Temp-Bereich: gelöschte Reste compilerlauf vom kompilieren von Codes, Wiedeherstellung nur am Duplikat, wenn Live: Auf sparatem System speichern

Anti-Forensik: Mehrfaches überschreiben mit echten zufälligen Bitmustern oder Nullen

\subsection{Unbekannte Binärdateien analysieren}
Fund Binärdateien (z.B. Rootkits, Tools) --> Analyse --> Hinweise / Absichten Täter, Vergleich Prüfsumme (Originaldatei?), Analyse auf Analysesystem --> Ausführung in isolierter Testumgebung

Ablauf gemäss CF: Seite 141

-DAteityp ermitteln (file), Virusscan, 

Tools: PEiD: Analyse Windows-Datei, kann getäuscht werden.
String-Analyse: Suche aller lesbaren Zeichen in Binärdatei --> Hilfetexte, Infos Autor, Copyright
Tools: Unix/Cygwin: strings, BinText, WinHex, IDA Pro

Laufzeitanalyse in isolierter Umgebung (Prozess, netzwerk-, Dateisystemaktivitäten), Analyse Hauptspeicher, Virtuelle Umgebung

Unix: strace, truss --> Zugriff auf welche Ressourcen

Auffinden von nicht angepassten STandard Kits / Tools relativ einfach

weitere Möglichkeit: Analyse dynamisch eingebundener Bibliotheken --> Grundfunktion nachvollziehen
ldd

\subsection{Systemprotokolle}
Anzeichen vor Angriff, oder vom Angrif, wenn nicht gesäubert --> anhaltspunkte, nicht typsich für Post-Mortem, häufig im Betrieb --> Feststellung Angriff

3 Formen: Normale Meldungen (Tagesbetrieb), ANzeichen Angriffe / kritische Meldungen, Unbekannte Meldungen
--> Fehlgeschalgende Anmeldeversuche, Erfolgreiche Anmeldeversiche im Vorfeld von laken Angriffen oder SU-vErsuchen
--Aktionen welche erweiterte Rechte erfoern, Shutdeon / Restor

Details im Kapitel Detection


\subsection{Untersuchung der Shell}
MAC-Time
\todo{HECF: S. 171}

\subsection{Untersuchung der Druckerjobs und der Druckerqueue}
\todo{HECF: S. 171}

\subsection{Untersuchung der Dateien / Dateiendungen}
\todo{HECF: S. 172}

\subsection{Untersuchung von User Aktivitäten}

\subsection{Vertiefte Analyse bei Verdacht auf Anti-Forensik-Techniken}

\subsection{Weitere Analyse-Möglichkeiten}
\begin{itemize}
\item Analyse der Master File Table
\item Analyse von NTFS-Streams
\item Analyse von NTFS TxF
\item NTFS-Volumen-Schattenkopien
\item Analyse der Windows-Registry
\item Analyse der Windows UserAssist Keys
\item Analyse der Windows Prefetch-Dateien
\item Analyse von Netzwerkmitschnitten
\end{itemize}

%---------------------------------------------------------------------------------------------------------------
\section{Reporting}




%-------------------------------------------


\section{Tool-Suiten}
\begin{itemize}
\item Encase
\end{itemize}


Tools: CF: S. 175




http://www.computer-forensik.org/tools/
\section{Kommerzielle Tools}
-SMART
-Helix (ein Teil Freeware)


\section{Tool-Matrix}
\begin{table}[H]
\centering
\caption{Verfügbarkeit der Tools auf verschiedenen Betriebssystemen}
\begin{tabular}{r | c | c | c | c}
Name 					&	\THrot{Windows}		&	\THrot{Linux}		&	\THrot{Mac OSX}		& 	\THrot{Weitere}		\\
\midrule
Test 					&	x			&				&	x			&			\\	
\bottomrule
\end{tabular}
\end{table}

\section{Technik-Matrix}
\begin{table}[H]
\centering
\caption{Einsatz der Techniken auf verschiedenen Betriebssystemen}
\begin{tabular}{r | c | c | c | c}
Name 					&	\THrot{Windows}		&	\THrot{Linux}		&	\THrot{Mac OSX}		& 	\THrot{Weitere}		\\
\midrule
Test 					&	x			&				&	x			&			\\	
\bottomrule
\end{tabular}
\end{table}

\section{Tool-Technik-Matrix}
Nur Tool-Suiten

\begin{table}[H]
\centering
\caption{Tools: Unterstützte Techniken}
\begin{tabular}{r | c | c | c | c}
Name 					&	\THrot{Tool1}		&	\THrot{Tool2}		&	\THrot{Tool3}		& 	\THrot{Tool4}		\\
\midrule
Test 					&	x			&				&	x			&			\\	
\bottomrule
\end{tabular}
\end{table}
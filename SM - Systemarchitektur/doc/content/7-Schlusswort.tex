% !TeX encoding=utf8
% !TeX spellcheck = de_CH_frami

\chapter{Schlusswort} \label{chap:Finish}
In diesem Kapitel wird ein Fazit zu dieser Arbeit und den einzelnen betrachteten Bereichen gezogen und über die gesamte Arbeit reflektiert.

%---------------------------------------------------------
\section{Fazit}
\subsection{BPM in der Domäne "`Internet of Things"'}
BPM ist im \gls{acr:IOT} definitiv ein Thema und es gibt auch bereits erste Hersteller, welche mit Ihren BPM Lösungen Möglichkeiten anbieten um Aspekte des \gls{acr:IOT} abzudecken, beziehungsweise zu integrieren. Auch gibt es zahlreiche \gls{acr:IOT} Plattformen, welche entsprechende Mechanismen zur Abbildung von Prozessen und Workflows Out-of-the-Box anbieten.

Gerade für Unternehmen, welche das \gls{acr:IOT} für das Kerngeschäft relevant ist oder relevant werden könnte, ist die Integration in (bestehende) \gls{acr:BPM} Lösungen ein wichtiger Aspekt. Grund dafür ist die dadurch erreichten Effizienzsteigerungen und Kostenreduktionen. Ebenfalls stellt die \gls{acr:BPM} Integration sicher, dass auch grosse Volumen korrekt und effizient verarbeitet werden können und dadurch ein Teil der in Kapitel \ref{sec:AnalyseIot:ChallangesAndProblems} \nameref{sec:AnalyseIot:ChallangesAndProblems} aufgezeigten Punkte abgedeckt, beziehungsweise entschärft werden können.

\subsection{BPM in der Domäne "`Home Automation"'}
Die durchgeführten Recherchen haben gezeigt, dass es im Bereich "`Home Automation"' heute eigentlich keine Lösung gibt, bei welcher Prozesse über \gls{acr:BPMN} oder \gls{acr:BPEL} modelliert werden können.



Die meisten Softwarelösungen und Kombi-Produkte (Software + Hardware) basieren auf dem Konzept von Auslösern (Triggern) und nachfolgenden Ereignissen (Events).

Heimanwender: BPM Schwierig, wenn dann im Bereich der Gebäude-Automatisierung im Bereich von Unternehmen.


OPENHAB:
Version 1: Schwieriger / Komplizierter Setup (insbesondere mit Z-Wave)

Version 2: Einfacherer Setup, viele fehlende Funktionalitäten, nicht intutitiv, Dokumentation für gewisse Grundkonzepte nicht wirklich gut

\subsection{BPM auf dem "`Raspberry Pi"' in der Domäne "`Home Automation"'}\label{subsec:Fazit:BPMN:RPI:HA}
Evtl. Getrennte Fazite für einzelne Bereiche?

Diverse Bindings ausprobiert (XMPP, Telegram, NTP, WOL, Z-Wave, System Info, JPA, MQTT, ), im Beispielssetup konnten nicht alle integriert werden.

\subsection{Allgemein}
Aufgrund der Ergebnisse der Analyse kann gesagt werden, dass aktuell im Bereich der Modellierung und Implementierung von Business Prozessen ein Wandel statt findet. Mit diesem Wandel rückt der Einsatz des \gls{acr:IOT} etwas mehr in den Vordergrund. 

....Bereits erste Produkte / Lösungen mit entsprechendem Support.



%---------------------------------------------------------
\section{Reflexion}




% !TeX encoding=utf8
% !TeX spellcheck = de_CH_frami

\chapter{Einleitung}


\section{Hintergrund}
Im Rahmen meines Bachelor-Studiums in Informatik an der \gls{acr:ZHAW} muss im 6. Semester eine Seminararbeit zu einem vorgegebenen Themenbereich erarbeitet werden. Ich habe mich für den Themenbereich "`Analyse und Angriffe auf Netzwerke"' entschieden.

Aus einem Themenkatalog konnte ein spezifisches Thema im Bereich "`Analyse und Angriffe auf Netzwerke"' ausgewählt werden. Ich habe mich für das Thema "`Computer Forensik"' entschieden.

Für die Arbeit sollen circa 50 Arbeitsstunden aufgewendet werden. Dies entspricht etwa einem Umfang von 15 bis 20 Seiten. Zusätzlich gelten die Rahmenbedingungen gemäss dem Reglement zur Verfassung einer Seminararbeit (\cite{ZHAW:2012:Seminararbeit:Reglemente})

\section{Aufgabenstellung}
In dieser Arbeit soll ein Überblick über das Themengebiet der "`Computer Forensik"' erarbeitet werden. Es soll gezeigt werden was für Themenbereiche es gibt und was für Werkzeuge und Tools eingesetzt werden können. Das Ganze soll mit einem Ablauf einer forensischen Untersuchung und entsprechenden Beispielen illustriert werden.

\section{Abgrenzung}
Aufgrund des grossen Themengebietes können nicht alle Detail-Aspekte der Computer Forensik berücksichtigt werden. Daher werden in dieser Arbeit nur einige Kernaspekte betrachtet. 

Folgende Themengebiete werden im Detail erläutert:
\begin{itemize}
\item Analyse von normalen Einzelplatz Unix-Systemen
\end{itemize}

Explizit ausgeschlossen werden folgende Themenbereiche:

\begin{itemize}
\item Detaillierte rechtliche Aspekte (zum Beispiel Strafrechtliches Vorgehen, Strafantrag, Tatortprinzip, etc.)
\item Remote-Analyse
\item Analyse von RAID-Systemen
\item Analyse von Windows und Mac OS X Systemen.
\end{itemize}

\section{Motivation}
Die forensischen Wissenschaften haben mich seit jeher fasziniert. Zusammen mit meinem berufsbedingten Interesse für Informatik, Computer und andere elektronische Geräte hat sich mit der Zeit das Interesse an der Computer Forensik herauskristallisiert. Ich hatte bereits vor längerer Zeit ein Buch zu diesem Thema gekauft, bin jedoch nie dazu gekommen, mich vertieft damit auseinanderzusetzen. Dieses Seminar hat mir nun ermöglicht, mich vertieft mit diesem Themenkomplex auseinanderzusetzen und erste Einblicke zu erhalten und Erfahrungen zu sammeln.

\subsection{Computerkriminalität}
Unter Computerkriminalität (auch als Cybercrime oder e-Crime bezeichnet) werden heute alle Straftaten zusammengefasst, welche mit Hilfe oder mit Unterstützung von informationsverarbeitenden Systemen durchgeführt wurden. Dazu zählen zum Beispiel: Betrug mit Zugangsberechtigungen, Betrug mit Konto- oder EC-karten mit PIN, Softwarepiraterie, Datenveränderung und Computersabotage oder Ausspähen von Daten. Angreifer können entweder Cyberkriminelle, Konkurrenten, Nachrichtendienste, Hacker, Hacktivisten oder auch Mitarbeiter sein.

Durch die starke Zunahme an Computerkriminalität in den letzten Jahren und die zunehmende Verbreitung von Informationstechnologien werden ich immer mehr Fachkräfte benötigt, welche in der Lage sind entsprechende Untersuchungen durchzuführen.

\section{Struktur}
Diese Arbeit gliedert sich in folgende Hauptteile:
\begin{itemize}
\item Einleitung
\item Angriffe
\item Incident Detection \& Incident Response
\item Computer Forensik
\item Forensische Analyse
\item Tools und Techniken
\item Schlusswort
\end{itemize}

Im ersten Kapitel werden die Details zur Ausgangslage und die Hintergründe der Arbeit aufgezeigt. Im darauffolgenden Kapitel wird zum besseren Verständnis die Kategorien und Phasen eines Angriffes aufgezeigt. Anschliessend wird der Ablauf einer Incident Response erklärt. Die darauffolgenden Kapitel beschäftigen sich mit dem Kernbereich der Arbeit, der Computer Forensik. Zuerst werden allgemeine Informationen zur Computer Forensik vermittelt, bevor die Forensische Analyse im Detail betrachtet wird. Im Kapitel 6 werden dann verschiedene Tools und Techniken vorgestellt, welche im Rahmen der forensischen Analyse eingesetzt werden können. Am Ende folgt noch das Schlusswort mit einem Fazit und einer Reflexion über die gesamte Arbeit.


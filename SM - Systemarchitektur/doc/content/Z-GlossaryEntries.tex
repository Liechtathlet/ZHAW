% !TeX encoding=utf8
% !TeX spellcheck = de_CH_frami

%LOOK: http://tex.stackexchange.com/questions/8946/how-to-combine-acronym-and-glossary

%%% --- Glossary entries


\newglossaryentry{gls:ERP}{name=ERP,
description={Ein klassiches \gls{acr:ERP} System dient dazu interne Prozesse in den Kernbereichen Finanz- und Rechnungswesen, Vertrieb \& Marketing, Personalwesen und Produktion zentral in einem System abzubilden. Heutige \gls{acr:ERP} Systeme bieten neben den Kernbereichen noch zahlreiche weitere Funktionen, welche die klassischen Geschäftsprozesse unterstützen. \cite[S. 486]{Pearson:WirtschaftsInf:LaudonEtAl}}}

\newglossaryentry{gls:CEP}{name=CEP,
description={Bei \gls{acr:CPE} ist ein Werkzeug aus dem Bereich BigData. Dabei steht die Erkennung, Analyse, Gruppierung und Verarbeitung von abhängigen Ereignissen im Vordergrund.}}


%%% --- Acronym definitions
\IfDefined{newacronym}{

\newacronym{acr:ZHAW}{ZHAW}{Zürcher Hochschule für Angewandte Wissenschaften}

\newacronym{acr:EBS}{EBS}{Einschreibe- und Bewertungssystem}

\newacronym{acr:IOT}{IoT}{Internet of Things}

\newacronym{acr:BPM}{BPM}{Business Process Management}

\newacronym{acr:BPMS}{BPMS}{Business Process Management Suite}

\newacronym{acr:BPMN}{BPMN}{Business Process Management Notation}

\newacronym{acr:BPEL}{BPEL}{Business Process Execution Language}

\newacronym{acr:BMA}{BMA}{Business Activity Monitoring}

\newacronym{acr:iBPMS}{iBPMS}{Intelligent Business Process Management Suite}

\newacronym{acr:CPE}{CPE}{Complex Event Processing}

\newglossaryentry{acr:ERP}{type=\acronymtype, name={ERP}, description={Enterprise-Resource-Planning}, first={Enterprise-Resource-Planning (ERP) \glsadd{gls:ERP}},see=[Glossary:]{\gls{gls:ERP}}}


}





